\documentclass[11pt,a4paper]{article}
\usepackage{isabelle,isabellesym}

\usepackage{amssymb}
\usepackage[english]{babel}
\usepackage[utf8]{inputenc}
\usepackage[only,bigsqcap]{stmaryrd}
\usepackage{eufrak}
\usepackage{textcomp}

% this should be the last package used
\usepackage{pdfsetup}

% urls in roman style, theory text in math-similar italics
\urlstyle{rm}

% for rule output in LaTeXsugar 
\usepackage{mathpartir}
\usepackage{graphicx}
\isabellestyle{it}

% this should be the last package used
\usepackage{pdfsetup}

\renewcommand{\isasymacute}{\isatext{\'\relax\hspace{-0.20em}}}
\DeclareRobustCommand{\isactrlesup}{\egroup\egroup\endmath\egroup\relax\hspace{-0.15em}}

\begin{document}

\title{--- \textbf{Simpl} --- \\
       A Sequential Imperative Programming Language\\  
       Syntax, Semantics, Hoare Logics and Verification Environment}
\author{Norbert W. Schirmer}

\begin{abstract}
We present the theory of Simpl, a sequential imperative programming language.
We introduce its syntax, its semantics (big and small-step operational
semantics) and Hoare logics for both partial as well as total correctness. 
We prove soundness and completeness of the Hoare logic. We
integrate and automate the Hoare logic in Isabelle/HOL to obtain a
practically usable verification environment for imperative programs.

Simpl is independent of a concrete programming language but expressive
enough to cover all common language features: mutually recursive
procedures, abrupt termination and exceptions, runtime faults, local
and global variables, pointers and heap, expressions with side
effects, pointers to procedures, partial application and closures,
dynamic method invocation and also unbounded nondeterminism.  
\end{abstract}

\maketitle

\tableofcontents
\parindent 0pt\parskip 0.5ex

\pagebreak

\begin{center}
  \makebox[0pt]{\includegraphics[width=\paperwidth=\textheight,keepaspectratio]{session_graph}
}\end{center}

\pagebreak

\section{Introduction}

The work presented in these theories was developed within the German Verisoft 
project\footnote{\url{http://www.verisoft.de}}. A thorough description of the core
parts can be found in my PhD thesis~\cite{Schirmer-PhD}. A tutorial-like user guide
is in Section~\ref{sec:UserGuide}.

Applications so far include BDD-normalisation~\cite{Ortner-Schirmer-TPHOL05},
a C0 compiler~\cite{Leinenbach:SSV08-??}, a page fault handler~\cite{Alkassar:TACAS08-??}
and extensions towards separation logic~\cite{Tuch:separation-logic:2007}.


% generated text of all theories
\input{Rtrancl_On.tex}

\input{Nat_Bijection.tex}

\input{Countable.tex}

\input{Extended_Nat.tex}

\input{Liminf_Limsup.tex}

\input{Extended_Real.tex}

\input{Stuff.tex}

\input{Digraph.tex}

\input{Arc_Walk.tex}

\input{Pair_Digraph.tex}

\input{Digraph_Component.tex}

\input{Vertex_Walk.tex}

\input{Digraph_Component_Vwalk.tex}

\input{Digraph_Isomorphism.tex}

\input{Euler.tex}

\input{Kuratowski.tex}

\input{Weighted_Graph.tex}

\input{Shortest_Path.tex}

\input{Graph_Theory.tex}

%
\begin{isabellebody}%
\def\isabellecontext{Connected{\isacharunderscore}Components}%
%
\isadelimtheory
%
\endisadelimtheory
%
\isatagtheory
\isacommand{theory}\isamarkupfalse%
\ Connected{\isacharunderscore}Components\isanewline
\isakeyword{imports}\ {\isachardoublequoteopen}{\isachardot}{\isachardot}{\isacharslash}Graph{\isacharunderscore}Theory{\isacharslash}Graph{\isacharunderscore}Theory{\isachardoublequoteclose}\isanewline
\isakeyword{begin}%
\endisatagtheory
{\isafoldtheory}%
%
\isadelimtheory
\isanewline
%
\endisadelimtheory
\isanewline
\isacommand{locale}\isamarkupfalse%
\ connected{\isacharunderscore}components{\isacharunderscore}locale\ {\isacharequal}\isanewline
\ \ fin{\isacharunderscore}digraph\ {\isacharplus}\isanewline
\ \ \isakeyword{fixes}\ num\ {\isacharcolon}{\isacharcolon}\ {\isachardoublequoteopen}{\isacharprime}a\ {\isasymRightarrow}\ nat{\isachardoublequoteclose}\isanewline
\ \ \isakeyword{fixes}\ parent{\isacharunderscore}edge\ {\isacharcolon}{\isacharcolon}\ {\isachardoublequoteopen}{\isacharprime}a\ {\isasymRightarrow}\ {\isacharprime}b\ option{\isachardoublequoteclose}\isanewline
\ \ \isakeyword{fixes}\ r\ {\isacharcolon}{\isacharcolon}\ {\isacharprime}a\isanewline
\ \ \isakeyword{assumes}\ r{\isacharunderscore}assms{\isacharcolon}\ {\isachardoublequoteopen}r\ {\isasymin}\ verts\ G\ {\isasymand}\ parent{\isacharunderscore}edge\ r\ {\isacharequal}\ None\ {\isasymand}\ num\ r\ {\isacharequal}\ {\isadigit{0}}{\isachardoublequoteclose}\isanewline
\ \ \isakeyword{assumes}\ parent{\isacharunderscore}num{\isacharunderscore}assms{\isacharcolon}\ \isanewline
\ \ \ \ {\isachardoublequoteopen}{\isasymAnd}v{\isachardot}\ v\ {\isasymin}\ verts\ G\ {\isasymand}\ v\ {\isasymnoteq}\ r\ {\isasymLongrightarrow}\isanewline
\ \ \ \ \ \ \ {\isasymexists}e\ {\isasymin}\ arcs\ G{\isachardot}\isanewline
\ \ \ \ \ \ \ \ \ parent{\isacharunderscore}edge\ v\ {\isacharequal}\ Some\ e\ {\isasymand}\isanewline
\ \ \ \ \ \ \ \ \ head\ G\ e\ {\isacharequal}\ v\ {\isasymand}\isanewline
\ \ \ \ \ \ \ \ \ num\ v\ {\isacharequal}\ \ num\ {\isacharparenleft}tail\ G\ e{\isacharparenright}\ {\isacharplus}\ {\isadigit{1}}{\isachardoublequoteclose}\isanewline
\isanewline
\isacommand{sublocale}\isamarkupfalse%
\ connected{\isacharunderscore}components{\isacharunderscore}locale\ {\isasymsubseteq}\ fin{\isacharunderscore}digraph\ G\isanewline
%
\isadelimproof
\ \ %
\endisadelimproof
%
\isatagproof
\isacommand{by}\isamarkupfalse%
\ auto%
\endisatagproof
{\isafoldproof}%
%
\isadelimproof
\isanewline
%
\endisadelimproof
\isanewline
\isacommand{context}\isamarkupfalse%
\ connected{\isacharunderscore}components{\isacharunderscore}locale\isanewline
\isakeyword{begin}\isanewline
\isanewline
\isacommand{lemma}\isamarkupfalse%
\ ccl{\isacharunderscore}wellformed{\isacharcolon}\ {\isachardoublequoteopen}wf{\isacharunderscore}digraph\ G{\isachardoublequoteclose}\isanewline
%
\isadelimproof
\ \ %
\endisadelimproof
%
\isatagproof
\isacommand{by}\isamarkupfalse%
\ unfold{\isacharunderscore}locales%
\endisatagproof
{\isafoldproof}%
%
\isadelimproof
\isanewline
%
\endisadelimproof
\isanewline
\isacommand{lemma}\isamarkupfalse%
\ num{\isacharunderscore}r{\isacharunderscore}is{\isacharunderscore}min{\isacharcolon}\isanewline
\ \ \isakeyword{assumes}\ {\isachardoublequoteopen}v\ {\isasymin}\ verts\ G{\isachardoublequoteclose}\ \isanewline
\ \ \isakeyword{assumes}\ {\isachardoublequoteopen}v\ {\isasymnoteq}\ r{\isachardoublequoteclose}\isanewline
\ \ \isakeyword{shows}\ {\isachardoublequoteopen}num\ v\ {\isachargreater}\ {\isadigit{0}}{\isachardoublequoteclose}\isanewline
%
\isadelimproof
\ \ %
\endisadelimproof
%
\isatagproof
\isacommand{using}\isamarkupfalse%
\ parent{\isacharunderscore}num{\isacharunderscore}assms\ assms\isanewline
\ \ \isacommand{by}\isamarkupfalse%
\ fastforce%
\endisatagproof
{\isafoldproof}%
%
\isadelimproof
\isanewline
%
\endisadelimproof
\isanewline
\isacommand{lemma}\isamarkupfalse%
\ path{\isacharunderscore}from{\isacharunderscore}root{\isacharcolon}\isanewline
\ \ \isakeyword{fixes}\ v\ {\isacharcolon}{\isacharcolon}\ {\isacharprime}a\isanewline
\ \ \isakeyword{assumes}\ {\isachardoublequoteopen}v\ {\isasymin}\ verts\ G{\isachardoublequoteclose}\isanewline
\ \ \isakeyword{shows}\ {\isachardoublequoteopen}r\ {\isasymrightarrow}\isactrlsup {\isacharasterisk}\ v{\isachardoublequoteclose}\isanewline
%
\isadelimproof
\ \ %
\endisadelimproof
%
\isatagproof
\isacommand{using}\isamarkupfalse%
\ assms\isanewline
\isacommand{proof}\isamarkupfalse%
\ {\isacharparenleft}induct\ {\isachardoublequoteopen}num\ v{\isachardoublequoteclose}\ arbitrary{\isacharcolon}\ v{\isacharparenright}\isanewline
\ \ \isacommand{case}\isamarkupfalse%
\ {\isadigit{0}}\isanewline
\ \ \isacommand{hence}\isamarkupfalse%
\ {\isachardoublequoteopen}v\ {\isacharequal}\ r{\isachardoublequoteclose}\ \isacommand{using}\isamarkupfalse%
\ num{\isacharunderscore}r{\isacharunderscore}is{\isacharunderscore}min\ \isacommand{by}\isamarkupfalse%
\ fastforce\isanewline
\ \ \isacommand{with}\isamarkupfalse%
\ {\isacharbackquoteopen}v\ {\isasymin}\ verts\ G{\isacharbackquoteclose}\ \isacommand{show}\isamarkupfalse%
\ {\isacharquery}case\ \isacommand{by}\isamarkupfalse%
\ auto\isanewline
\isacommand{next}\isamarkupfalse%
\isanewline
\ \ \isacommand{case}\isamarkupfalse%
\ {\isacharparenleft}Suc\ n{\isacharprime}{\isacharparenright}\isanewline
\ \ \isacommand{hence}\isamarkupfalse%
\ {\isachardoublequoteopen}v\ {\isasymnoteq}\ r{\isachardoublequoteclose}\ \isacommand{using}\isamarkupfalse%
\ r{\isacharunderscore}assms\ \isacommand{by}\isamarkupfalse%
\ auto\isanewline
\ \ \isacommand{then}\isamarkupfalse%
\ \isacommand{obtain}\isamarkupfalse%
\ e\ \isakeyword{where}\ ee{\isacharcolon}\isanewline
\ \ \ \ {\isachardoublequoteopen}e\ {\isasymin}\ arcs\ G{\isachardoublequoteclose}\isanewline
\ \ \ \ {\isachardoublequoteopen}head\ G\ e\ {\isacharequal}\ v\ {\isasymand}\ num\ v\ {\isacharequal}\ num\ {\isacharparenleft}tail\ G\ e{\isacharparenright}\ {\isacharplus}\ {\isadigit{1}}{\isachardoublequoteclose}\ \ \isanewline
\ \ \ \ \isacommand{using}\isamarkupfalse%
\ Suc\ parent{\isacharunderscore}num{\isacharunderscore}assms\ \isacommand{by}\isamarkupfalse%
\ blast\isanewline
\ \ \isacommand{with}\isamarkupfalse%
\ {\isacharbackquoteopen}v\ {\isasymin}\ verts\ G{\isacharbackquoteclose}\ Suc{\isacharparenleft}{\isadigit{1}}{\isacharcomma}{\isadigit{2}}{\isacharparenright}\ tail{\isacharunderscore}in{\isacharunderscore}verts\isanewline
\ \ \isacommand{have}\isamarkupfalse%
\ {\isachardoublequoteopen}r\ {\isasymrightarrow}\isactrlsup {\isacharasterisk}\ {\isacharparenleft}tail\ G\ e{\isacharparenright}{\isachardoublequoteclose}\ {\isachardoublequoteopen}tail\ G\ e\ {\isasymrightarrow}\ v{\isachardoublequoteclose}\isanewline
\ \ \ \ \isacommand{by}\isamarkupfalse%
\ {\isacharparenleft}auto\ intro{\isacharcolon}\ in{\isacharunderscore}arcs{\isacharunderscore}imp{\isacharunderscore}in{\isacharunderscore}arcs{\isacharunderscore}ends{\isacharparenright}\isanewline
\ \ \isacommand{then}\isamarkupfalse%
\ \isacommand{show}\isamarkupfalse%
\ {\isacharquery}case\ \isacommand{by}\isamarkupfalse%
\ {\isacharparenleft}rule\ reachable{\isacharunderscore}adj{\isacharunderscore}trans{\isacharparenright}\isanewline
\isacommand{qed}\isamarkupfalse%
%
\endisatagproof
{\isafoldproof}%
%
\isadelimproof
%
\endisadelimproof
%
\begin{isamarkuptext}%
The underlying undirected, simple graph is connected%
\end{isamarkuptext}%
\isamarkuptrue%
\isacommand{lemma}\isamarkupfalse%
\ connectedG{\isacharcolon}\ {\isachardoublequoteopen}connected\ G{\isachardoublequoteclose}\isanewline
%
\isadelimproof
%
\endisadelimproof
%
\isatagproof
\isacommand{proof}\isamarkupfalse%
\ {\isacharparenleft}unfold\ connected{\isacharunderscore}def{\isacharcomma}\ intro\ strongly{\isacharunderscore}connectedI{\isacharparenright}\isanewline
\ \ \ \ \isacommand{show}\isamarkupfalse%
\ {\isachardoublequoteopen}verts\ {\isacharparenleft}with{\isacharunderscore}proj\ {\isacharparenleft}mk{\isacharunderscore}symmetric\ G{\isacharparenright}{\isacharparenright}\ {\isasymnoteq}\ {\isacharbraceleft}{\isacharbraceright}{\isachardoublequoteclose}\ \isanewline
\ \ \ \ \ \ \isacommand{by}\isamarkupfalse%
\ {\isacharparenleft}metis\ equals{\isadigit{0}}D\ r{\isacharunderscore}assms\ reachable{\isacharunderscore}in{\isacharunderscore}vertsE\ reachable{\isacharunderscore}mk{\isacharunderscore}symmetricI\ reachable{\isacharunderscore}refl{\isacharparenright}\isanewline
\ \ \isacommand{next}\isamarkupfalse%
\isanewline
\ \ \isacommand{let}\isamarkupfalse%
\ {\isacharquery}SG\ {\isacharequal}\ {\isachardoublequoteopen}mk{\isacharunderscore}symmetric\ G{\isachardoublequoteclose}\isanewline
\ \ \isacommand{interpret}\isamarkupfalse%
\ S{\isacharcolon}\ pair{\isacharunderscore}fin{\isacharunderscore}digraph\ {\isachardoublequoteopen}{\isacharquery}SG{\isachardoublequoteclose}\ \isacommand{{\isachardot}{\isachardot}}\isamarkupfalse%
\isanewline
\ \ \isacommand{fix}\isamarkupfalse%
\ u\ v\ \isacommand{assume}\isamarkupfalse%
\ uv{\isacharunderscore}sG{\isacharcolon}\ {\isachardoublequoteopen}u\ {\isasymin}\ verts\ {\isacharquery}SG{\isachardoublequoteclose}\ {\isachardoublequoteopen}v\ {\isasymin}\ verts\ {\isacharquery}SG{\isachardoublequoteclose}\isanewline
\ \ \isacommand{from}\isamarkupfalse%
\ uv{\isacharunderscore}sG\ \isacommand{have}\isamarkupfalse%
\ {\isachardoublequoteopen}u\ {\isasymin}\ verts\ G{\isachardoublequoteclose}\ {\isachardoublequoteopen}v\ {\isasymin}\ verts\ G{\isachardoublequoteclose}\ \isacommand{by}\isamarkupfalse%
\ auto\isanewline
\ \ \isacommand{then}\isamarkupfalse%
\ \isacommand{have}\isamarkupfalse%
\ {\isachardoublequoteopen}u\ {\isasymrightarrow}\isactrlsup {\isacharasterisk}\isactrlbsub {\isacharquery}SG\isactrlesub \ r{\isachardoublequoteclose}\ {\isachardoublequoteopen}r\ {\isasymrightarrow}\isactrlsup {\isacharasterisk}\isactrlbsub {\isacharquery}SG\isactrlesub \ v{\isachardoublequoteclose}\ \isanewline
\ \ \ \ \isacommand{by}\isamarkupfalse%
\ {\isacharparenleft}auto\ intro{\isacharcolon}\ reachable{\isacharunderscore}mk{\isacharunderscore}symmetricI\ path{\isacharunderscore}from{\isacharunderscore}root\ \ symmetric{\isacharunderscore}reachable\isanewline
\ \ \ \ \ \ symmetric{\isacharunderscore}mk{\isacharunderscore}symmetric\ simp\ del{\isacharcolon}\ pverts{\isacharunderscore}mk{\isacharunderscore}symmetric{\isacharparenright}\isanewline
\ \ \isacommand{then}\isamarkupfalse%
\ \isacommand{show}\isamarkupfalse%
\ {\isachardoublequoteopen}u\ {\isasymrightarrow}\isactrlsup {\isacharasterisk}\isactrlbsub {\isacharquery}SG\isactrlesub \ v{\isachardoublequoteclose}\isanewline
\ \ \ \ \isacommand{by}\isamarkupfalse%
\ {\isacharparenleft}rule\ S{\isachardot}reachable{\isacharunderscore}trans{\isacharparenright}\isanewline
\isacommand{qed}\isamarkupfalse%
%
\endisatagproof
{\isafoldproof}%
%
\isadelimproof
\isanewline
%
\endisadelimproof
\isanewline
\isacommand{theorem}\isamarkupfalse%
\ connected{\isacharunderscore}by{\isacharunderscore}path{\isacharcolon}\isanewline
\ \ \isakeyword{fixes}\ u\ v\ {\isacharcolon}{\isacharcolon}\ {\isacharprime}a\isanewline
\ \ \isakeyword{assumes}\ {\isachardoublequoteopen}u\ {\isasymin}\ pverts\ {\isacharparenleft}mk{\isacharunderscore}symmetric\ G{\isacharparenright}{\isachardoublequoteclose}\isanewline
\ \ \isakeyword{assumes}\ {\isachardoublequoteopen}v\ {\isasymin}\ pverts\ {\isacharparenleft}mk{\isacharunderscore}symmetric\ G{\isacharparenright}{\isachardoublequoteclose}\isanewline
\ \ \isakeyword{shows}\ {\isachardoublequoteopen}u\ {\isasymrightarrow}\isactrlsup {\isacharasterisk}\isactrlbsub mk{\isacharunderscore}symmetric\ G\isactrlesub \ v{\isachardoublequoteclose}\isanewline
%
\isadelimproof
%
\endisadelimproof
%
\isatagproof
\isacommand{using}\isamarkupfalse%
\ connectedG\ \ wellformed{\isacharunderscore}mk{\isacharunderscore}symmetric\ assms\isanewline
\isacommand{unfolding}\isamarkupfalse%
\ connected{\isacharunderscore}def\ strongly{\isacharunderscore}connected{\isacharunderscore}def\ \isacommand{by}\isamarkupfalse%
\ fastforce%
\endisatagproof
{\isafoldproof}%
%
\isadelimproof
\isanewline
%
\endisadelimproof
\isacommand{end}\isamarkupfalse%
\isanewline
\isanewline
\isacommand{corollary}\isamarkupfalse%
\ {\isacharparenleft}\isakeyword{in}\ connected{\isacharunderscore}components{\isacharunderscore}locale{\isacharparenright}\ connected{\isacharunderscore}graph{\isacharcolon}\isanewline
\ \ \isakeyword{assumes}\ {\isachardoublequoteopen}u\ {\isasymin}\ verts\ G{\isachardoublequoteclose}\ \isakeyword{and}\ {\isachardoublequoteopen}v\ {\isasymin}\ verts\ G{\isachardoublequoteclose}\isanewline
\ \ \isakeyword{shows}\ {\isachardoublequoteopen}{\isasymexists}p{\isachardot}\ vpath\ p\ {\isacharparenleft}mk{\isacharunderscore}symmetric\ G{\isacharparenright}\ {\isasymand}\ hd\ p\ {\isacharequal}\ u\ {\isasymand}\ last\ p\ {\isacharequal}\ v{\isachardoublequoteclose}\isanewline
%
\isadelimproof
%
\endisadelimproof
%
\isatagproof
\isacommand{proof}\isamarkupfalse%
\ {\isacharminus}\isanewline
\ \ \isacommand{interpret}\isamarkupfalse%
\ S{\isacharcolon}\ pair{\isacharunderscore}fin{\isacharunderscore}digraph\ {\isachardoublequoteopen}mk{\isacharunderscore}symmetric\ G{\isachardoublequoteclose}\ \isacommand{{\isachardot}{\isachardot}}\isamarkupfalse%
\isanewline
\ \ \isacommand{show}\isamarkupfalse%
\ {\isachardoublequoteopen}{\isacharquery}thesis{\isachardoublequoteclose}\ \isacommand{unfolding}\isamarkupfalse%
\ S{\isachardot}reachable{\isacharunderscore}vpath{\isacharunderscore}conv{\isacharbrackleft}symmetric{\isacharbrackright}\isanewline
\ \ \ \ \isacommand{using}\isamarkupfalse%
\ assms\ \isacommand{by}\isamarkupfalse%
\ {\isacharparenleft}auto\ intro{\isacharcolon}\ connected{\isacharunderscore}by{\isacharunderscore}path{\isacharparenright}\isanewline
\isacommand{qed}\isamarkupfalse%
%
\endisatagproof
{\isafoldproof}%
%
\isadelimproof
\isanewline
%
\endisadelimproof
%
\isadelimtheory
\isanewline
%
\endisadelimtheory
%
\isatagtheory
\isacommand{end}\isamarkupfalse%
%
\endisatagtheory
{\isafoldtheory}%
%
\isadelimtheory
%
\endisadelimtheory
\end{isabellebody}%
%%% Local Variables:
%%% mode: latex
%%% TeX-master: "root"
%%% End:


%
\begin{isabellebody}%
\def\isabellecontext{Shortest{\isacharunderscore}Path{\isacharunderscore}Theory}%
%
\isadelimtheory
%
\endisadelimtheory
%
\isatagtheory
\isacommand{theory}\isamarkupfalse%
\ Shortest{\isacharunderscore}Path{\isacharunderscore}Theory\isanewline
\isakeyword{imports}\isanewline
\ \ Complex\ \ \ \isanewline
\ \ {\isachardoublequoteopen}{\isachardot}{\isachardot}{\isacharslash}Graph{\isacharunderscore}Theory{\isacharslash}Graph{\isacharunderscore}Theory{\isachardoublequoteclose}\isanewline
\isakeyword{begin}%
\endisatagtheory
{\isafoldtheory}%
%
\isadelimtheory
\isanewline
%
\endisadelimtheory
\isanewline
\isanewline
\isacommand{locale}\isamarkupfalse%
\ basic{\isacharunderscore}sp\ {\isacharequal}\ \isanewline
\ \ fin{\isacharunderscore}digraph\ {\isacharplus}\isanewline
\ \ \isakeyword{fixes}\ dist\ {\isacharcolon}{\isacharcolon}\ {\isachardoublequoteopen}{\isacharprime}a\ {\isasymRightarrow}\ ereal{\isachardoublequoteclose}\isanewline
\ \ \isakeyword{fixes}\ c\ {\isacharcolon}{\isacharcolon}\ {\isachardoublequoteopen}{\isacharprime}b\ {\isasymRightarrow}\ real{\isachardoublequoteclose}\isanewline
\ \ \isakeyword{fixes}\ s\ {\isacharcolon}{\isacharcolon}\ {\isachardoublequoteopen}{\isacharprime}a{\isachardoublequoteclose}\isanewline
\ \ \isakeyword{assumes}\ general{\isacharunderscore}source{\isacharunderscore}val{\isacharcolon}\ {\isachardoublequoteopen}dist\ s\ {\isasymle}\ {\isadigit{0}}{\isachardoublequoteclose}\isanewline
\ \ \isakeyword{assumes}\ trian{\isacharcolon}\isanewline
\ \ \ \ {\isachardoublequoteopen}{\isasymAnd}e{\isachardot}\ e\ {\isasymin}\ arcs\ G\ {\isasymLongrightarrow}\ \isanewline
\ \ \ \ \ \ dist\ {\isacharparenleft}head\ G\ e{\isacharparenright}\ {\isasymle}\ dist\ {\isacharparenleft}tail\ G\ e{\isacharparenright}\ {\isacharplus}\ c\ e{\isachardoublequoteclose}\isanewline
\isanewline
\isacommand{locale}\isamarkupfalse%
\ basic{\isacharunderscore}just{\isacharunderscore}sp\ {\isacharequal}\ \isanewline
\ \ basic{\isacharunderscore}sp\ {\isacharplus}\isanewline
\ \ \isakeyword{fixes}\ enum\ {\isacharcolon}{\isacharcolon}\ {\isachardoublequoteopen}{\isacharprime}a\ {\isasymRightarrow}\ enat{\isachardoublequoteclose}\isanewline
\ \ \isakeyword{assumes}\ just{\isacharcolon}\isanewline
\ \ \ \ {\isachardoublequoteopen}{\isasymAnd}v{\isachardot}\ {\isasymlbrakk}v\ {\isasymin}\ verts\ G{\isacharsemicolon}\ v\ {\isasymnoteq}\ s{\isacharsemicolon}\ enum\ v\ {\isasymnoteq}\ {\isasyminfinity}{\isasymrbrakk}\ {\isasymLongrightarrow}\isanewline
\ \ \ \ \ \ {\isasymexists}\ e\ {\isasymin}\ arcs\ G{\isachardot}\ v\ {\isacharequal}\ head\ G\ e\ {\isasymand}\isanewline
\ \ \ \ \ \ \ \ dist\ v\ {\isacharequal}\ dist\ {\isacharparenleft}tail\ G\ e{\isacharparenright}\ {\isacharplus}\ c\ e\ \ {\isasymand}\isanewline
\ \ \ \ \ \ \ \ enum\ v\ {\isacharequal}\ enum\ {\isacharparenleft}tail\ G\ e{\isacharparenright}\ {\isacharplus}\ {\isacharparenleft}enat\ {\isadigit{1}}{\isacharparenright}{\isachardoublequoteclose}\isanewline
\isanewline
\isacommand{locale}\isamarkupfalse%
\ shortest{\isacharunderscore}path{\isacharunderscore}non{\isacharunderscore}neg{\isacharunderscore}cost\ {\isacharequal}\isanewline
\ \ basic{\isacharunderscore}just{\isacharunderscore}sp\ {\isacharplus}\isanewline
\ \ \isakeyword{assumes}\ s{\isacharunderscore}in{\isacharunderscore}G{\isacharcolon}\ {\isachardoublequoteopen}s\ {\isasymin}\ verts\ G{\isachardoublequoteclose}\isanewline
\ \ \isakeyword{assumes}\ source{\isacharunderscore}val{\isacharcolon}\ {\isachardoublequoteopen}dist\ s\ {\isacharequal}\ {\isadigit{0}}{\isachardoublequoteclose}\isanewline
\ \ \isakeyword{assumes}\ no{\isacharunderscore}path{\isacharcolon}\ {\isachardoublequoteopen}{\isasymAnd}v{\isachardot}\ v\ {\isasymin}\ verts\ G\ {\isasymLongrightarrow}\ dist\ v\ {\isacharequal}\ {\isasyminfinity}\ {\isasymlongleftrightarrow}\ enum\ v\ {\isacharequal}\ {\isasyminfinity}{\isachardoublequoteclose}\isanewline
\ \ \isakeyword{assumes}\ non{\isacharunderscore}neg{\isacharunderscore}cost{\isacharcolon}\ {\isachardoublequoteopen}{\isasymAnd}e{\isachardot}\ e\ {\isasymin}\ arcs\ G\ {\isasymLongrightarrow}\ {\isadigit{0}}\ {\isasymle}\ c\ e{\isachardoublequoteclose}\isanewline
\isanewline
\isacommand{locale}\isamarkupfalse%
\ basic{\isacharunderscore}just{\isacharunderscore}sp{\isacharunderscore}pred\ {\isacharequal}\ \isanewline
\ \ basic{\isacharunderscore}sp\ {\isacharplus}\isanewline
\ \ \isakeyword{fixes}\ enum\ {\isacharcolon}{\isacharcolon}\ {\isachardoublequoteopen}{\isacharprime}a\ {\isasymRightarrow}\ enat{\isachardoublequoteclose}\isanewline
\ \ \isakeyword{fixes}\ pred\ {\isacharcolon}{\isacharcolon}\ {\isachardoublequoteopen}{\isacharprime}a\ {\isasymRightarrow}\ {\isacharprime}b\ option{\isachardoublequoteclose}\isanewline
\ \ \isakeyword{assumes}\ just{\isacharcolon}\isanewline
\ \ \ \ {\isachardoublequoteopen}{\isasymAnd}v{\isachardot}\ {\isasymlbrakk}v\ {\isasymin}\ verts\ G{\isacharsemicolon}\ v\ {\isasymnoteq}\ s{\isacharsemicolon}\ enum\ v\ {\isasymnoteq}\ {\isasyminfinity}{\isasymrbrakk}\ {\isasymLongrightarrow}\isanewline
\ \ \ \ \ \ {\isasymexists}\ e\ {\isasymin}\ arcs\ G{\isachardot}\ \isanewline
\ \ \ \ \ \ \ \ e\ {\isacharequal}\ the\ {\isacharparenleft}pred\ v{\isacharparenright}\ {\isasymand}\ \isanewline
\ \ \ \ \ \ \ \ v\ {\isacharequal}\ head\ G\ e\ {\isasymand}\isanewline
\ \ \ \ \ \ \ \ dist\ v\ {\isacharequal}\ dist\ {\isacharparenleft}tail\ G\ e{\isacharparenright}\ {\isacharplus}\ c\ e\ \ {\isasymand}\isanewline
\ \ \ \ \ \ \ \ enum\ v\ {\isacharequal}\ enum\ {\isacharparenleft}tail\ G\ e{\isacharparenright}\ {\isacharplus}\ {\isacharparenleft}enat\ {\isadigit{1}}{\isacharparenright}{\isachardoublequoteclose}\isanewline
\isanewline
\isacommand{sublocale}\isamarkupfalse%
\ basic{\isacharunderscore}just{\isacharunderscore}sp{\isacharunderscore}pred\ {\isasymsubseteq}\ basic{\isacharunderscore}just{\isacharunderscore}sp\ \ \isanewline
%
\isadelimproof
%
\endisadelimproof
%
\isatagproof
\isacommand{using}\isamarkupfalse%
\ basic{\isacharunderscore}just{\isacharunderscore}sp{\isacharunderscore}pred{\isacharunderscore}axioms\ \isanewline
\isacommand{unfolding}\isamarkupfalse%
\ basic{\isacharunderscore}just{\isacharunderscore}sp{\isacharunderscore}pred{\isacharunderscore}def\isanewline
\ \ \ basic{\isacharunderscore}just{\isacharunderscore}sp{\isacharunderscore}pred{\isacharunderscore}axioms{\isacharunderscore}def\isanewline
\isacommand{by}\isamarkupfalse%
\ unfold{\isacharunderscore}locales\ {\isacharparenleft}blast{\isacharparenright}%
\endisatagproof
{\isafoldproof}%
%
\isadelimproof
\isanewline
%
\endisadelimproof
\isanewline
\isacommand{locale}\isamarkupfalse%
\ shortest{\isacharunderscore}path{\isacharunderscore}non{\isacharunderscore}neg{\isacharunderscore}cost{\isacharunderscore}pred\ {\isacharequal}\isanewline
\ \ basic{\isacharunderscore}just{\isacharunderscore}sp{\isacharunderscore}pred\ {\isacharplus}\isanewline
\ \ \isakeyword{assumes}\ s{\isacharunderscore}in{\isacharunderscore}G{\isacharcolon}\ {\isachardoublequoteopen}s\ {\isasymin}\ verts\ G{\isachardoublequoteclose}\isanewline
\ \ \isakeyword{assumes}\ source{\isacharunderscore}val{\isacharcolon}\ {\isachardoublequoteopen}dist\ s\ {\isacharequal}\ {\isadigit{0}}{\isachardoublequoteclose}\isanewline
\ \ \isakeyword{assumes}\ no{\isacharunderscore}path{\isacharcolon}\ {\isachardoublequoteopen}{\isasymAnd}v{\isachardot}\ v\ {\isasymin}\ verts\ G\ {\isasymLongrightarrow}\ dist\ v\ {\isacharequal}\ {\isasyminfinity}\ {\isasymlongleftrightarrow}\ enum\ v\ {\isacharequal}\ {\isasyminfinity}{\isachardoublequoteclose}\isanewline
\ \ \isakeyword{assumes}\ non{\isacharunderscore}neg{\isacharunderscore}cost{\isacharcolon}\ {\isachardoublequoteopen}{\isasymAnd}e{\isachardot}\ e\ {\isasymin}\ arcs\ G\ {\isasymLongrightarrow}\ {\isadigit{0}}\ {\isasymle}\ c\ e{\isachardoublequoteclose}\isanewline
\isanewline
\isacommand{sublocale}\isamarkupfalse%
\ shortest{\isacharunderscore}path{\isacharunderscore}non{\isacharunderscore}neg{\isacharunderscore}cost{\isacharunderscore}pred\ {\isasymsubseteq}\ shortest{\isacharunderscore}path{\isacharunderscore}non{\isacharunderscore}neg{\isacharunderscore}cost\isanewline
%
\isadelimproof
%
\endisadelimproof
%
\isatagproof
\isacommand{using}\isamarkupfalse%
\ shortest{\isacharunderscore}path{\isacharunderscore}non{\isacharunderscore}neg{\isacharunderscore}cost{\isacharunderscore}pred{\isacharunderscore}axioms\ \isanewline
\isacommand{by}\isamarkupfalse%
\ unfold{\isacharunderscore}locales\ \isanewline
\ \ \ {\isacharparenleft}auto\ simp{\isacharcolon}\ shortest{\isacharunderscore}path{\isacharunderscore}non{\isacharunderscore}neg{\isacharunderscore}cost{\isacharunderscore}pred{\isacharunderscore}def\ \isanewline
\ \ \ shortest{\isacharunderscore}path{\isacharunderscore}non{\isacharunderscore}neg{\isacharunderscore}cost{\isacharunderscore}pred{\isacharunderscore}axioms{\isacharunderscore}def{\isacharparenright}%
\endisatagproof
{\isafoldproof}%
%
\isadelimproof
\isanewline
%
\endisadelimproof
\isanewline
\isacommand{lemma}\isamarkupfalse%
\ tail{\isacharunderscore}value{\isacharunderscore}helper{\isacharcolon}\isanewline
\ \ \isakeyword{assumes}\ {\isachardoublequoteopen}hd\ p\ {\isacharequal}\ last\ p{\isachardoublequoteclose}\isanewline
\ \ \isakeyword{assumes}\ {\isachardoublequoteopen}distinct\ p{\isachardoublequoteclose}\isanewline
\ \ \isakeyword{assumes}\ {\isachardoublequoteopen}p\ {\isasymnoteq}\ {\isacharbrackleft}{\isacharbrackright}{\isachardoublequoteclose}\isanewline
\ \ \isakeyword{shows}\ {\isachardoublequoteopen}p\ {\isacharequal}\ {\isacharbrackleft}hd\ p{\isacharbrackright}{\isachardoublequoteclose}\isanewline
%
\isadelimproof
%
\endisadelimproof
%
\isatagproof
\isacommand{by}\isamarkupfalse%
\ {\isacharparenleft}metis\ assms\ distinct{\isachardot}simps{\isacharparenleft}{\isadigit{2}}{\isacharparenright}\ append{\isacharunderscore}butlast{\isacharunderscore}last{\isacharunderscore}id\ hd{\isacharunderscore}append\isanewline
\ \ append{\isacharunderscore}self{\isacharunderscore}conv{\isadigit{2}}\ distinct{\isacharunderscore}butlast\ hd{\isacharunderscore}in{\isacharunderscore}set\ not{\isacharunderscore}distinct{\isacharunderscore}conv{\isacharunderscore}prefix{\isacharparenright}%
\endisatagproof
{\isafoldproof}%
%
\isadelimproof
\isanewline
%
\endisadelimproof
\isanewline
\isacommand{lemma}\isamarkupfalse%
\ {\isacharparenleft}\isakeyword{in}\ basic{\isacharunderscore}sp{\isacharparenright}\ dist{\isacharunderscore}le{\isacharunderscore}cost{\isacharcolon}\isanewline
\ \ \isakeyword{fixes}\ v\ {\isacharcolon}{\isacharcolon}\ {\isacharprime}a\isanewline
\ \ \isakeyword{fixes}\ p\ {\isacharcolon}{\isacharcolon}\ {\isachardoublequoteopen}{\isacharprime}b\ list{\isachardoublequoteclose}\ \isanewline
\ \ \isakeyword{assumes}\ {\isachardoublequoteopen}awalk\ s\ p\ v{\isachardoublequoteclose}\isanewline
\ \ \isakeyword{shows}\ {\isachardoublequoteopen}dist\ v\ {\isasymle}\ awalk{\isacharunderscore}cost\ c\ p{\isachardoublequoteclose}\isanewline
%
\isadelimproof
\ \ %
\endisadelimproof
%
\isatagproof
\isacommand{using}\isamarkupfalse%
\ assms\isanewline
\ \ \isacommand{proof}\isamarkupfalse%
\ {\isacharparenleft}induct\ {\isachardoublequoteopen}length\ p{\isachardoublequoteclose}\ arbitrary{\isacharcolon}\ p\ v{\isacharparenright}\isanewline
\ \ \isacommand{case}\isamarkupfalse%
\ {\isadigit{0}}\isanewline
\ \ \ \ \isacommand{hence}\isamarkupfalse%
\ {\isachardoublequoteopen}s\ {\isacharequal}\ v{\isachardoublequoteclose}\ \isacommand{by}\isamarkupfalse%
\ auto\isanewline
\ \ \ \ \isacommand{thus}\isamarkupfalse%
\ {\isacharquery}case\ \isacommand{using}\isamarkupfalse%
\ {\isadigit{0}}{\isacharparenleft}{\isadigit{1}}{\isacharparenright}\ general{\isacharunderscore}source{\isacharunderscore}val\isanewline
\ \ \ \ \ \ \isacommand{by}\isamarkupfalse%
\ {\isacharparenleft}metis\ awalk{\isacharunderscore}cost{\isacharunderscore}Nil\ length{\isacharunderscore}{\isadigit{0}}{\isacharunderscore}conv\ zero{\isacharunderscore}ereal{\isacharunderscore}def{\isacharparenright}\isanewline
\ \ \isacommand{next}\isamarkupfalse%
\isanewline
\ \ \isacommand{case}\isamarkupfalse%
\ {\isacharparenleft}Suc\ n{\isacharparenright}\isanewline
\ \ \ \ \isacommand{then}\isamarkupfalse%
\ \isacommand{obtain}\isamarkupfalse%
\ p{\isacharprime}\ e\ \isakeyword{where}\ p{\isacharprime}e{\isacharcolon}\ {\isachardoublequoteopen}p\ {\isacharequal}\ p{\isacharprime}\ {\isacharat}\ {\isacharbrackleft}e{\isacharbrackright}{\isachardoublequoteclose}\isanewline
\ \ \ \ \ \ \isacommand{by}\isamarkupfalse%
\ {\isacharparenleft}cases\ p\ rule{\isacharcolon}\ rev{\isacharunderscore}cases{\isacharparenright}\ auto\isanewline
\ \ \ \ \isacommand{then}\isamarkupfalse%
\ \isacommand{obtain}\isamarkupfalse%
\ u\ \isakeyword{where}\ ewu{\isacharcolon}\ {\isachardoublequoteopen}awalk\ s\ p{\isacharprime}\ u\ {\isasymand}\ awalk\ u\ {\isacharbrackleft}e{\isacharbrackright}\ v{\isachardoublequoteclose}\ \isanewline
\ \ \ \ \ \ \isacommand{using}\isamarkupfalse%
\ awalk{\isacharunderscore}append{\isacharunderscore}iff\ Suc{\isacharparenleft}{\isadigit{3}}{\isacharparenright}\ \isacommand{by}\isamarkupfalse%
\ simp\isanewline
\ \ \ \ \isacommand{then}\isamarkupfalse%
\ \isacommand{have}\isamarkupfalse%
\ du{\isacharcolon}\ {\isachardoublequoteopen}dist\ u\ {\isasymle}\ ereal\ {\isacharparenleft}awalk{\isacharunderscore}cost\ c\ p{\isacharprime}{\isacharparenright}{\isachardoublequoteclose}\isanewline
\ \ \ \ \ \ \isacommand{using}\isamarkupfalse%
\ Suc\ p{\isacharprime}e\ \isacommand{by}\isamarkupfalse%
\ simp\isanewline
\ \ \ \ \isacommand{from}\isamarkupfalse%
\ ewu\ \isacommand{have}\isamarkupfalse%
\ ust{\isacharcolon}\ {\isachardoublequoteopen}u\ {\isacharequal}\ tail\ G\ e{\isachardoublequoteclose}\ \isakeyword{and}\ vta{\isacharcolon}\ {\isachardoublequoteopen}v\ {\isacharequal}\ head\ G\ e{\isachardoublequoteclose}\isanewline
\ \ \ \ \ \ \isacommand{by}\isamarkupfalse%
\ auto\isanewline
\ \ \ \ \isacommand{then}\isamarkupfalse%
\ \isacommand{have}\isamarkupfalse%
\ {\isachardoublequoteopen}dist\ v\ {\isasymle}\ dist\ u\ {\isacharplus}\ c\ e{\isachardoublequoteclose}\isanewline
\ \ \ \ \ \ \isacommand{using}\isamarkupfalse%
\ ewu\ du\ ust\ trian{\isacharbrackleft}\isakeyword{where}\ e{\isacharequal}e{\isacharbrackright}\ \isacommand{by}\isamarkupfalse%
\ force\isanewline
\ \ \ \ \isacommand{with}\isamarkupfalse%
\ du\ \isacommand{have}\isamarkupfalse%
\ {\isachardoublequoteopen}dist\ v\ {\isasymle}\ ereal\ {\isacharparenleft}awalk{\isacharunderscore}cost\ c\ p{\isacharprime}{\isacharparenright}\ {\isacharplus}\ c\ e{\isachardoublequoteclose}\isanewline
\ \ \ \ \ \ \isacommand{by}\isamarkupfalse%
\ {\isacharparenleft}metis\ add{\isacharunderscore}right{\isacharunderscore}mono\ order{\isacharunderscore}trans{\isacharparenright}\isanewline
\ \ \ \ \isacommand{thus}\isamarkupfalse%
\ {\isachardoublequoteopen}dist\ v\ {\isasymle}\ awalk{\isacharunderscore}cost\ c\ p{\isachardoublequoteclose}\ \isanewline
\ \ \ \ \ \ \isacommand{using}\isamarkupfalse%
\ awalk{\isacharunderscore}cost{\isacharunderscore}append\ p{\isacharprime}e\ \isacommand{by}\isamarkupfalse%
\ simp\isanewline
\ \ \isacommand{qed}\isamarkupfalse%
%
\endisatagproof
{\isafoldproof}%
%
\isadelimproof
\isanewline
%
\endisadelimproof
\isanewline
\isacommand{lemma}\isamarkupfalse%
\ {\isacharparenleft}\isakeyword{in}\ fin{\isacharunderscore}digraph{\isacharparenright}\ witness{\isacharunderscore}path{\isacharcolon}\isanewline
\ \ \isakeyword{assumes}\ {\isachardoublequoteopen}{\isasymmu}\ c\ s\ v\ {\isacharequal}\ ereal\ r{\isachardoublequoteclose}\isanewline
\ \ \isakeyword{shows}\ {\isachardoublequoteopen}{\isasymexists}\ p{\isachardot}\ apath\ s\ p\ v\ {\isasymand}\ {\isasymmu}\ c\ s\ v\ {\isacharequal}\ awalk{\isacharunderscore}cost\ c\ p{\isachardoublequoteclose}\isanewline
%
\isadelimproof
%
\endisadelimproof
%
\isatagproof
\isacommand{proof}\isamarkupfalse%
\ {\isacharminus}\isanewline
\ \ \isacommand{have}\isamarkupfalse%
\ sv{\isacharcolon}\ {\isachardoublequoteopen}s\ {\isasymrightarrow}\isactrlsup {\isacharasterisk}\ v{\isachardoublequoteclose}\ \isanewline
\ \ \ \ \isacommand{using}\isamarkupfalse%
\ shortest{\isacharunderscore}path{\isacharunderscore}inf{\isacharbrackleft}of\ s\ v\ c{\isacharbrackright}\ assms\ \isacommand{by}\isamarkupfalse%
\ fastforce\isanewline
\ \ \isacommand{{\isacharbraceleft}}\isamarkupfalse%
\ \isanewline
\ \ \ \ \isacommand{fix}\isamarkupfalse%
\ p\ \isacommand{assume}\isamarkupfalse%
\ {\isachardoublequoteopen}awalk\ s\ p\ v{\isachardoublequoteclose}\isanewline
\ \ \ \ \isacommand{then}\isamarkupfalse%
\ \isacommand{have}\isamarkupfalse%
\ no{\isacharunderscore}neg{\isacharunderscore}cyc{\isacharcolon}\ \isanewline
\ \ \ \ {\isachardoublequoteopen}{\isasymnot}\ {\isacharparenleft}{\isasymexists}w\ q{\isachardot}\ awalk\ w\ q\ w\ {\isasymand}\ w\ {\isasymin}\ set\ {\isacharparenleft}awalk{\isacharunderscore}verts\ s\ p{\isacharparenright}\ {\isasymand}\ awalk{\isacharunderscore}cost\ c\ q\ {\isacharless}\ {\isadigit{0}}{\isacharparenright}{\isachardoublequoteclose}\isanewline
\ \ \ \ \ \ \isacommand{using}\isamarkupfalse%
\ neg{\isacharunderscore}cycle{\isacharunderscore}imp{\isacharunderscore}inf{\isacharunderscore}{\isasymmu}\ assms\ \isacommand{by}\isamarkupfalse%
\ force\isanewline
\ \ \isacommand{{\isacharbraceright}}\isamarkupfalse%
\isanewline
\ \ \isacommand{thus}\isamarkupfalse%
\ {\isacharquery}thesis\ \isacommand{using}\isamarkupfalse%
\ no{\isacharunderscore}neg{\isacharunderscore}cyc{\isacharunderscore}reach{\isacharunderscore}imp{\isacharunderscore}path{\isacharbrackleft}OF\ sv{\isacharbrackright}\ \isacommand{by}\isamarkupfalse%
\ presburger\isanewline
\isacommand{qed}\isamarkupfalse%
%
\endisatagproof
{\isafoldproof}%
%
\isadelimproof
\isanewline
%
\endisadelimproof
\isanewline
\isacommand{lemma}\isamarkupfalse%
\ {\isacharparenleft}\isakeyword{in}\ basic{\isacharunderscore}sp{\isacharparenright}\ \ dist{\isacharunderscore}le{\isacharunderscore}{\isasymmu}{\isacharcolon}\isanewline
\ \ \isakeyword{fixes}\ v\ {\isacharcolon}{\isacharcolon}\ {\isacharprime}a\isanewline
\ \ \isakeyword{assumes}\ {\isachardoublequoteopen}v\ {\isasymin}\ verts\ G{\isachardoublequoteclose}\isanewline
\ \ \isakeyword{shows}\ {\isachardoublequoteopen}dist\ v\ {\isasymle}\ {\isasymmu}\ c\ s\ v{\isachardoublequoteclose}\ \isanewline
%
\isadelimproof
%
\endisadelimproof
%
\isatagproof
\isacommand{proof}\isamarkupfalse%
\ {\isacharparenleft}rule\ ccontr{\isacharparenright}\isanewline
\ \ \isacommand{assume}\isamarkupfalse%
\ nt{\isacharcolon}\ {\isachardoublequoteopen}{\isasymnot}\ {\isacharquery}thesis{\isachardoublequoteclose}\isanewline
\ \ \isacommand{show}\isamarkupfalse%
\ False\isanewline
\ \ \isacommand{proof}\isamarkupfalse%
\ {\isacharparenleft}cases\ {\isachardoublequoteopen}{\isasymmu}\ c\ s\ v{\isachardoublequoteclose}{\isacharparenright}\isanewline
\ \ \ \ \isacommand{show}\isamarkupfalse%
\ {\isachardoublequoteopen}{\isasymAnd}r{\isachardot}\ {\isasymmu}\ c\ s\ v\ {\isacharequal}\ ereal\ r\ {\isasymLongrightarrow}\ False{\isachardoublequoteclose}\isanewline
\ \ \ \ \isacommand{proof}\isamarkupfalse%
\ {\isacharminus}\isanewline
\ \ \ \ \ \ \isacommand{fix}\isamarkupfalse%
\ r\ \isacommand{assume}\isamarkupfalse%
\ r{\isacharunderscore}asm{\isacharcolon}\ {\isachardoublequoteopen}{\isasymmu}\ c\ s\ v\ {\isacharequal}\ ereal\ r{\isachardoublequoteclose}\isanewline
\ \ \ \ \ \ \isacommand{hence}\isamarkupfalse%
\ sv{\isacharcolon}\ {\isachardoublequoteopen}s\ {\isasymrightarrow}\isactrlsup {\isacharasterisk}\ v{\isachardoublequoteclose}\isanewline
\ \ \ \ \ \ \ \ \isacommand{using}\isamarkupfalse%
\ shortest{\isacharunderscore}path{\isacharunderscore}inf{\isacharbrackleft}\isakeyword{where}\ u{\isacharequal}s\ \isakeyword{and}\ v{\isacharequal}v\ \isakeyword{and}\ f{\isacharequal}c{\isacharbrackright}\ \isacommand{by}\isamarkupfalse%
\ auto\isanewline
\ \ \ \ \ \ \isacommand{obtain}\isamarkupfalse%
\ p\ \isakeyword{where}\ \isanewline
\ \ \ \ \ \ \ \ {\isachardoublequoteopen}awalk\ s\ p\ v{\isachardoublequoteclose}\ \isanewline
\ \ \ \ \ \ \ \ {\isachardoublequoteopen}{\isasymmu}\ c\ s\ v\ {\isacharequal}\ awalk{\isacharunderscore}cost\ c\ p{\isachardoublequoteclose}\isanewline
\ \ \ \ \ \ \ \ \isacommand{using}\isamarkupfalse%
\ witness{\isacharunderscore}path{\isacharbrackleft}OF\ r{\isacharunderscore}asm{\isacharbrackright}\ \isacommand{unfolding}\isamarkupfalse%
\ apath{\isacharunderscore}def\ \isacommand{by}\isamarkupfalse%
\ force\ \isanewline
\ \ \ \ \ \ \isacommand{thus}\isamarkupfalse%
\ False\ \isacommand{using}\isamarkupfalse%
\ nt\ dist{\isacharunderscore}le{\isacharunderscore}cost\ \isacommand{by}\isamarkupfalse%
\ simp\isanewline
\ \ \ \ \isacommand{qed}\isamarkupfalse%
\isanewline
\ \ \isacommand{next}\isamarkupfalse%
\isanewline
\ \ \ \ \isacommand{show}\isamarkupfalse%
\ {\isachardoublequoteopen}{\isasymmu}\ c\ s\ v\ {\isacharequal}\ {\isasyminfinity}\ {\isasymLongrightarrow}\ False{\isachardoublequoteclose}\ \isacommand{using}\isamarkupfalse%
\ nt\ \isacommand{by}\isamarkupfalse%
\ simp\isanewline
\ \ \isacommand{next}\isamarkupfalse%
\isanewline
\ \ \ \ \isacommand{show}\isamarkupfalse%
\ {\isachardoublequoteopen}{\isasymmu}\ c\ s\ v\ {\isacharequal}\ {\isacharminus}\ {\isasyminfinity}\ {\isasymLongrightarrow}\ False{\isachardoublequoteclose}\ \isanewline
\ \ \ \ \isacommand{proof}\isamarkupfalse%
\ {\isacharminus}\isanewline
\ \ \ \ \ \ \isacommand{assume}\isamarkupfalse%
\ asm{\isacharcolon}\ {\isachardoublequoteopen}{\isasymmu}\ c\ s\ v\ {\isacharequal}\ {\isacharminus}\ {\isasyminfinity}{\isachardoublequoteclose}\isanewline
\ \ \ \ \ \ \isacommand{let}\isamarkupfalse%
\ {\isacharquery}C\ {\isacharequal}\ {\isachardoublequoteopen}{\isacharparenleft}{\isasymlambda}x{\isachardot}\ ereal\ {\isacharparenleft}awalk{\isacharunderscore}cost\ c\ x{\isacharparenright}{\isacharparenright}\ {\isacharbackquote}\ {\isacharbraceleft}p{\isachardot}\ awalk\ s\ p\ v{\isacharbraceright}{\isachardoublequoteclose}\isanewline
\ \ \ \ \ \ \isacommand{have}\isamarkupfalse%
\ {\isachardoublequoteopen}{\isasymexists}x{\isasymin}\ {\isacharquery}C{\isachardot}\ x\ {\isacharless}\ dist\ v{\isachardoublequoteclose}\ \isanewline
\ \ \ \ \ \ \ \ \isacommand{using}\isamarkupfalse%
\ Inf{\isacharunderscore}ereal{\isacharunderscore}iff\ {\isacharbrackleft}\isakeyword{where}\ y\ {\isacharequal}{\isachardoublequoteopen}dist\ v{\isachardoublequoteclose}\isakeyword{and}\ X{\isacharequal}{\isachardoublequoteopen}{\isacharquery}C{\isachardoublequoteclose}\ \isakeyword{and}\ z{\isacharequal}\ {\isachardoublequoteopen}{\isacharminus}{\isasyminfinity}{\isachardoublequoteclose}{\isacharbrackright}\ \isanewline
\ \ \ \ \ \ \ \ nt\ asm\ \isacommand{unfolding}\isamarkupfalse%
\ {\isasymmu}{\isacharunderscore}def\ INF{\isacharunderscore}def\ \isacommand{by}\isamarkupfalse%
\ simp\isanewline
\ \ \ \ \ \ \isacommand{then}\isamarkupfalse%
\ \isacommand{obtain}\isamarkupfalse%
\ p\ \isakeyword{where}\ \ \isanewline
\ \ \ \ \ \ \ \ {\isachardoublequoteopen}awalk\ s\ p\ v{\isachardoublequoteclose}\ \isanewline
\ \ \ \ \ \ \ \ {\isachardoublequoteopen}awalk{\isacharunderscore}cost\ c\ p\ {\isacharless}\ dist\ v{\isachardoublequoteclose}\ \isanewline
\ \ \ \ \ \ \ \ \isacommand{by}\isamarkupfalse%
\ force\isanewline
\ \ \ \ \ \ \isacommand{thus}\isamarkupfalse%
\ False\ \isacommand{using}\isamarkupfalse%
\ dist{\isacharunderscore}le{\isacharunderscore}cost\ \isacommand{by}\isamarkupfalse%
\ force\isanewline
\ \ \ \ \isacommand{qed}\isamarkupfalse%
\isanewline
\ \ \isacommand{qed}\isamarkupfalse%
\isanewline
\isacommand{qed}\isamarkupfalse%
%
\endisatagproof
{\isafoldproof}%
%
\isadelimproof
\isanewline
%
\endisadelimproof
\isanewline
\isacommand{lemma}\isamarkupfalse%
\ {\isacharparenleft}\isakeyword{in}\ basic{\isacharunderscore}just{\isacharunderscore}sp{\isacharparenright}\ dist{\isacharunderscore}ge{\isacharunderscore}{\isasymmu}{\isacharcolon}\isanewline
\ \ \isakeyword{fixes}\ v\ {\isacharcolon}{\isacharcolon}\ {\isacharprime}a\isanewline
\ \ \isakeyword{assumes}\ {\isachardoublequoteopen}v\ {\isasymin}\ verts\ G{\isachardoublequoteclose}\isanewline
\ \ \isakeyword{assumes}\ {\isachardoublequoteopen}enum\ v\ {\isasymnoteq}\ {\isasyminfinity}{\isachardoublequoteclose}\isanewline
\ \ \isakeyword{assumes}\ {\isachardoublequoteopen}dist\ v\ {\isasymnoteq}\ {\isacharminus}{\isasyminfinity}{\isachardoublequoteclose}\isanewline
\ \ \isakeyword{assumes}\ {\isachardoublequoteopen}{\isasymmu}\ c\ s\ s\ {\isacharequal}\ ereal\ {\isadigit{0}}{\isachardoublequoteclose}\isanewline
\ \ \isakeyword{assumes}\ {\isachardoublequoteopen}dist\ s\ {\isacharequal}\ {\isadigit{0}}{\isachardoublequoteclose}\isanewline
\ \ \isakeyword{assumes}\ {\isachardoublequoteopen}{\isasymAnd}u{\isachardot}\ u{\isasymin}verts\ G\ {\isasymLongrightarrow}\ u{\isasymnoteq}s\ {\isasymLongrightarrow}\ enum\ u\ {\isasymnoteq}\ enat\ {\isadigit{0}}{\isachardoublequoteclose}\isanewline
\ \ \isakeyword{shows}\ {\isachardoublequoteopen}dist\ v\ {\isasymge}\ {\isasymmu}\ c\ s\ v{\isachardoublequoteclose}\isanewline
%
\isadelimproof
%
\endisadelimproof
%
\isatagproof
\isacommand{proof}\isamarkupfalse%
\ {\isacharminus}\isanewline
\ \ \isacommand{obtain}\isamarkupfalse%
\ n\ \isakeyword{where}\ {\isachardoublequoteopen}enat\ n\ {\isacharequal}\ enum\ v{\isachardoublequoteclose}\ \isacommand{using}\isamarkupfalse%
\ assms{\isacharparenleft}{\isadigit{2}}{\isacharparenright}\ \isacommand{by}\isamarkupfalse%
\ force\isanewline
\ \ \isacommand{thus}\isamarkupfalse%
\ {\isacharquery}thesis\ \isacommand{using}\isamarkupfalse%
\ assms\isanewline
\ \ \isacommand{proof}\isamarkupfalse%
{\isacharparenleft}induct\ n\ arbitrary{\isacharcolon}\ v{\isacharparenright}\ \isanewline
\ \ \isacommand{case}\isamarkupfalse%
\ {\isadigit{0}}\ \isacommand{thus}\isamarkupfalse%
\ {\isacharquery}case\ \isacommand{by}\isamarkupfalse%
\ {\isacharparenleft}cases\ {\isachardoublequoteopen}v{\isacharequal}s{\isachardoublequoteclose}{\isacharcomma}\ auto{\isacharparenright}\isanewline
\ \ \isacommand{next}\isamarkupfalse%
\isanewline
\ \ \isacommand{case}\isamarkupfalse%
\ {\isacharparenleft}Suc\ n{\isacharparenright}\isanewline
\ \ \ \ \isacommand{thus}\isamarkupfalse%
\ {\isacharquery}case\ \isanewline
\ \ \ \ \isacommand{proof}\isamarkupfalse%
\ {\isacharparenleft}cases\ {\isachardoublequoteopen}v{\isacharequal}s{\isachardoublequoteclose}{\isacharparenright}\isanewline
\ \ \ \ \isacommand{case}\isamarkupfalse%
\ False\isanewline
\ \ \ \ \ \ \isacommand{obtain}\isamarkupfalse%
\ e\ \isakeyword{where}\ e{\isacharunderscore}assms{\isacharcolon}\isanewline
\ \ \ \ \ \ \ \ {\isachardoublequoteopen}e\ {\isasymin}\ arcs\ G{\isachardoublequoteclose}\ \isanewline
\ \ \ \ \ \ \ \ {\isachardoublequoteopen}v\ {\isacharequal}\ head\ G\ e{\isachardoublequoteclose}\isanewline
\ \ \ \ \ \ \ \ {\isachardoublequoteopen}dist\ v\ {\isacharequal}\ dist\ {\isacharparenleft}tail\ G\ e{\isacharparenright}\ {\isacharplus}\ ereal\ {\isacharparenleft}c\ e{\isacharparenright}{\isachardoublequoteclose}\ \isanewline
\ \ \ \ \ \ \ \ {\isachardoublequoteopen}enum\ v\ {\isacharequal}\ enum\ {\isacharparenleft}tail\ G\ e{\isacharparenright}\ {\isacharplus}\ enat\ {\isadigit{1}}{\isachardoublequoteclose}\ \isanewline
\ \ \ \ \ \ \ \ \isacommand{using}\isamarkupfalse%
\ just{\isacharbrackleft}OF\ Suc{\isacharparenleft}{\isadigit{3}}{\isacharparenright}\ False\ Suc{\isacharparenleft}{\isadigit{4}}{\isacharparenright}{\isacharbrackright}\ \isacommand{by}\isamarkupfalse%
\ blast\isanewline
\ \ \ \ \ \ \isacommand{then}\isamarkupfalse%
\ \isacommand{have}\isamarkupfalse%
\ nsinf{\isacharcolon}{\isachardoublequoteopen}enum\ {\isacharparenleft}tail\ G\ e{\isacharparenright}\ {\isasymnoteq}\ {\isasyminfinity}{\isachardoublequoteclose}\ \isanewline
\ \ \ \ \ \ \ \ \isacommand{by}\isamarkupfalse%
\ {\isacharparenleft}metis\ Suc{\isacharparenleft}{\isadigit{2}}{\isacharparenright}\ enat{\isachardot}simps{\isacharparenleft}{\isadigit{3}}{\isacharparenright}\ enat{\isacharunderscore}{\isadigit{1}}\ plus{\isacharunderscore}enat{\isacharunderscore}simps{\isacharparenleft}{\isadigit{2}}{\isacharparenright}{\isacharparenright}\isanewline
\ \ \ \ \ \ \isacommand{then}\isamarkupfalse%
\ \isacommand{have}\isamarkupfalse%
\ ns{\isacharcolon}{\isachardoublequoteopen}enat\ n\ {\isacharequal}\ enum\ {\isacharparenleft}tail\ G\ e{\isacharparenright}{\isachardoublequoteclose}\ \isanewline
\ \ \ \ \ \ \ \ \isacommand{using}\isamarkupfalse%
\ e{\isacharunderscore}assms{\isacharparenleft}{\isadigit{4}}{\isacharparenright}\ Suc{\isacharparenleft}{\isadigit{2}}{\isacharparenright}\ \isacommand{by}\isamarkupfalse%
\ force\isanewline
\ \ \ \ \ \ \isacommand{have}\isamarkupfalse%
\ \ ds{\isacharcolon}\ {\isachardoublequoteopen}dist\ {\isacharparenleft}tail\ G\ e{\isacharparenright}\ {\isacharequal}\ {\isasymmu}\ c\ s\ {\isacharparenleft}tail\ G\ e{\isacharparenright}{\isachardoublequoteclose}\ \isanewline
\ \ \ \ \ \ \ \ \isacommand{using}\isamarkupfalse%
\ Suc{\isacharparenleft}{\isadigit{1}}{\isacharparenright}{\isacharbrackleft}OF\ ns\ tail{\isacharunderscore}in{\isacharunderscore}verts{\isacharbrackleft}OF\ e{\isacharunderscore}assms{\isacharparenleft}{\isadigit{1}}{\isacharparenright}{\isacharbrackright}\ nsinf{\isacharbrackright}\ \isanewline
\ \ \ \ \ \ \ \ Suc{\isacharparenleft}{\isadigit{5}}{\isacharminus}{\isadigit{8}}{\isacharparenright}\ e{\isacharunderscore}assms{\isacharparenleft}{\isadigit{3}}{\isacharparenright}\ dist{\isacharunderscore}le{\isacharunderscore}{\isasymmu}{\isacharbrackleft}OF\ tail{\isacharunderscore}in{\isacharunderscore}verts{\isacharbrackleft}OF\ e{\isacharunderscore}assms{\isacharparenleft}{\isadigit{1}}{\isacharparenright}{\isacharbrackright}{\isacharbrackright}\ \isanewline
\ \ \ \ \ \ \ \ \isacommand{by}\isamarkupfalse%
\ simp\isanewline
\ \ \ \ \ \ \isacommand{have}\isamarkupfalse%
\ dmuc{\isacharcolon}{\isachardoublequoteopen}dist\ v\ {\isacharequal}\ {\isasymmu}\ c\ s\ {\isacharparenleft}tail\ G\ e{\isacharparenright}\ {\isacharplus}\ ereal\ {\isacharparenleft}c\ e{\isacharparenright}{\isachardoublequoteclose}\isanewline
\ \ \ \ \ \ \ \ \isacommand{using}\isamarkupfalse%
\ e{\isacharunderscore}assms{\isacharparenleft}{\isadigit{3}}{\isacharparenright}\ ds\ \ \isacommand{by}\isamarkupfalse%
\ auto\isanewline
\ \ \ \ \ \ \isacommand{thus}\isamarkupfalse%
\ {\isacharquery}thesis\isanewline
\ \ \ \ \ \ \isacommand{proof}\isamarkupfalse%
\ {\isacharparenleft}cases\ {\isachardoublequoteopen}dist\ v\ {\isacharequal}\ {\isasyminfinity}{\isachardoublequoteclose}{\isacharparenright}\isanewline
\ \ \ \ \ \ \isacommand{case}\isamarkupfalse%
\ False\isanewline
\ \ \ \ \ \ \ \ \isacommand{have}\isamarkupfalse%
\ {\isachardoublequoteopen}arc{\isacharunderscore}to{\isacharunderscore}ends\ G\ e\ {\isacharequal}\ {\isacharparenleft}tail\ G\ e{\isacharcomma}\ v{\isacharparenright}{\isachardoublequoteclose}\ \isanewline
\ \ \ \ \ \ \ \ \ \ \isacommand{unfolding}\isamarkupfalse%
\ arc{\isacharunderscore}to{\isacharunderscore}ends{\isacharunderscore}def\isanewline
\ \ \ \ \ \ \ \ \ \ \isacommand{by}\isamarkupfalse%
\ {\isacharparenleft}simp\ add{\isacharcolon}\ e{\isacharunderscore}assms{\isacharparenleft}{\isadigit{2}}{\isacharparenright}{\isacharparenright}\isanewline
\ \ \ \ \ \ \ \ \isacommand{obtain}\isamarkupfalse%
\ r\ \isakeyword{where}\ \ {\isasymmu}r{\isacharcolon}\ {\isachardoublequoteopen}{\isasymmu}\ c\ s\ {\isacharparenleft}tail\ G\ e{\isacharparenright}\ {\isacharequal}\ ereal\ r{\isachardoublequoteclose}\isanewline
\ \ \ \ \ \ \ \ \ \ \ \isacommand{using}\isamarkupfalse%
\ e{\isacharunderscore}assms{\isacharparenleft}{\isadigit{3}}{\isacharparenright}\ Suc{\isacharparenleft}{\isadigit{5}}{\isacharparenright}\ ds\ False\isanewline
\ \ \ \ \ \ \ \ \ \ \ \isacommand{by}\isamarkupfalse%
\ {\isacharparenleft}cases\ {\isachardoublequoteopen}{\isasymmu}\ c\ s\ {\isacharparenleft}tail\ G\ e{\isacharparenright}{\isachardoublequoteclose}{\isacharcomma}\ auto{\isacharparenright}\isanewline
\ \ \ \ \ \ \ \ \isacommand{obtain}\isamarkupfalse%
\ p\ \isakeyword{where}\ \isanewline
\ \ \ \ \ \ \ \ \ \ {\isachardoublequoteopen}awalk\ s\ p\ {\isacharparenleft}tail\ G\ e{\isacharparenright}{\isachardoublequoteclose}\ \isakeyword{and}\isanewline
\ \ \ \ \ \ \ \ \ \ {\isasymmu}s{\isacharcolon}\ {\isachardoublequoteopen}{\isasymmu}\ c\ s\ {\isacharparenleft}tail\ G\ e{\isacharparenright}\ {\isacharequal}\ ereal\ {\isacharparenleft}awalk{\isacharunderscore}cost\ c\ p{\isacharparenright}{\isachardoublequoteclose}\isanewline
\ \ \ \ \ \ \ \ \ \ \isacommand{using}\isamarkupfalse%
\ witness{\isacharunderscore}path{\isacharbrackleft}OF\ {\isasymmu}r{\isacharbrackright}\ \isacommand{unfolding}\isamarkupfalse%
\ apath{\isacharunderscore}def\ \isanewline
\ \ \ \ \ \ \ \ \ \ \isacommand{by}\isamarkupfalse%
\ blast\isanewline
\ \ \ \ \ \ \ \ \isacommand{then}\isamarkupfalse%
\ \isacommand{have}\isamarkupfalse%
\ pe{\isacharcolon}\ {\isachardoublequoteopen}awalk\ s\ {\isacharparenleft}p\ {\isacharat}\ {\isacharbrackleft}e{\isacharbrackright}{\isacharparenright}\ v{\isachardoublequoteclose}\ \isanewline
\ \ \ \ \ \ \ \ \ \ \isacommand{using}\isamarkupfalse%
\ e{\isacharunderscore}assms{\isacharparenleft}{\isadigit{1}}{\isacharcomma}{\isadigit{2}}{\isacharparenright}\ \isacommand{by}\isamarkupfalse%
\ {\isacharparenleft}auto\ simp{\isacharcolon}\ awalk{\isacharunderscore}simps\ awlast{\isacharunderscore}of{\isacharunderscore}awalk{\isacharparenright}\isanewline
\ \ \ \ \ \ \ \ \isacommand{hence}\isamarkupfalse%
\ muc{\isacharcolon}{\isachardoublequoteopen}{\isasymmu}\ c\ s\ v\ {\isasymle}\ {\isasymmu}\ c\ s\ {\isacharparenleft}tail\ G\ e{\isacharparenright}\ {\isacharplus}\ ereal\ {\isacharparenleft}c\ e{\isacharparenright}{\isachardoublequoteclose}\ \isanewline
\ \ \ \ \ \ \ \ \isacommand{using}\isamarkupfalse%
\ {\isasymmu}s\ min{\isacharunderscore}cost{\isacharunderscore}le{\isacharunderscore}walk{\isacharunderscore}cost{\isacharbrackleft}OF\ pe{\isacharbrackright}\ \isacommand{by}\isamarkupfalse%
\ simp\ \isanewline
\ \ \ \ \ \ \ \ \isacommand{thus}\isamarkupfalse%
\ \ {\isachardoublequoteopen}dist\ v\ {\isasymge}\ {\isasymmu}\ c\ s\ v{\isachardoublequoteclose}\ \ \isacommand{using}\isamarkupfalse%
\ dmuc\ \isacommand{by}\isamarkupfalse%
\ simp\isanewline
\ \ \ \ \ \ \isacommand{qed}\isamarkupfalse%
\ simp\isanewline
\ \ \ \ \isacommand{qed}\isamarkupfalse%
\ {\isacharparenleft}simp\ add{\isacharcolon}\ Suc{\isacharparenleft}{\isadigit{6}}{\isacharcomma}{\isadigit{7}}{\isacharparenright}{\isacharparenright}\isanewline
\ \ \isacommand{qed}\isamarkupfalse%
\isanewline
\isacommand{qed}\isamarkupfalse%
%
\endisatagproof
{\isafoldproof}%
%
\isadelimproof
\isanewline
%
\endisadelimproof
\isanewline
\isacommand{lemma}\isamarkupfalse%
\ {\isacharparenleft}\isakeyword{in}\ shortest{\isacharunderscore}path{\isacharunderscore}non{\isacharunderscore}neg{\isacharunderscore}cost{\isacharparenright}\ tail{\isacharunderscore}value{\isacharunderscore}check{\isacharcolon}\ \isanewline
\ \ \isakeyword{fixes}\ u\ {\isacharcolon}{\isacharcolon}\ {\isacharprime}a\isanewline
\ \ \isakeyword{assumes}\ {\isachardoublequoteopen}s\ {\isasymin}\ verts\ G{\isachardoublequoteclose}\isanewline
\ \ \isakeyword{shows}\ {\isachardoublequoteopen}{\isasymmu}\ c\ s\ s\ {\isacharequal}\ ereal\ {\isadigit{0}}{\isachardoublequoteclose}\isanewline
%
\isadelimproof
%
\endisadelimproof
%
\isatagproof
\isacommand{proof}\isamarkupfalse%
\ {\isacharminus}\isanewline
\ \ \isacommand{have}\isamarkupfalse%
\ {\isacharasterisk}{\isacharcolon}\ {\isachardoublequoteopen}awalk\ s\ {\isacharbrackleft}{\isacharbrackright}\ s{\isachardoublequoteclose}\ \isacommand{using}\isamarkupfalse%
\ assms\ \isacommand{unfolding}\isamarkupfalse%
\ awalk{\isacharunderscore}def\ \isacommand{by}\isamarkupfalse%
\ simp\isanewline
\ \ \isacommand{hence}\isamarkupfalse%
\ {\isachardoublequoteopen}{\isasymmu}\ c\ s\ s\ {\isasymle}\ ereal\ {\isadigit{0}}{\isachardoublequoteclose}\ \isacommand{using}\isamarkupfalse%
\ min{\isacharunderscore}cost{\isacharunderscore}le{\isacharunderscore}walk{\isacharunderscore}cost{\isacharbrackleft}OF\ {\isacharasterisk}{\isacharbrackright}\ \isacommand{by}\isamarkupfalse%
\ simp\isanewline
\ \ \isacommand{moreover}\isamarkupfalse%
\isanewline
\ \ \isacommand{have}\isamarkupfalse%
\ {\isachardoublequoteopen}{\isacharparenleft}{\isasymAnd}p{\isachardot}\ awalk\ s\ p\ s\ {\isasymLongrightarrow}\ ereal{\isacharparenleft}awalk{\isacharunderscore}cost\ c\ p{\isacharparenright}\ {\isasymge}\ ereal\ {\isadigit{0}}{\isacharparenright}{\isachardoublequoteclose}\isanewline
\ \ \ \isacommand{using}\isamarkupfalse%
\ non{\isacharunderscore}neg{\isacharunderscore}cost\ pos{\isacharunderscore}cost{\isacharunderscore}pos{\isacharunderscore}awalk{\isacharunderscore}cost\ \isacommand{by}\isamarkupfalse%
\ auto\isanewline
\ \ \isacommand{hence}\isamarkupfalse%
\ {\isachardoublequoteopen}{\isasymmu}\ c\ s\ s\ {\isasymge}\ ereal\ {\isadigit{0}}{\isachardoublequoteclose}\ \isanewline
\ \ \ \ \isacommand{unfolding}\isamarkupfalse%
\ {\isasymmu}{\isacharunderscore}def\ \isacommand{by}\isamarkupfalse%
\ {\isacharparenleft}blast\ intro{\isacharcolon}\ INF{\isacharunderscore}greatest{\isacharparenright}\isanewline
\ \ \isacommand{ultimately}\isamarkupfalse%
\isanewline
\ \ \isacommand{show}\isamarkupfalse%
\ {\isacharquery}thesis\ \isacommand{by}\isamarkupfalse%
\ simp\isanewline
\isacommand{qed}\isamarkupfalse%
%
\endisatagproof
{\isafoldproof}%
%
\isadelimproof
\isanewline
%
\endisadelimproof
\isanewline
\isacommand{lemma}\isamarkupfalse%
\ {\isacharparenleft}\isakeyword{in}\ shortest{\isacharunderscore}path{\isacharunderscore}non{\isacharunderscore}neg{\isacharunderscore}cost{\isacharparenright}\ enum{\isacharunderscore}not{\isadigit{0}}{\isacharcolon}\isanewline
\ \ \isakeyword{fixes}\ v\ {\isacharcolon}{\isacharcolon}\ {\isacharprime}a\isanewline
\ \ \isakeyword{assumes}\ {\isachardoublequoteopen}v\ {\isasymin}\ verts\ G{\isachardoublequoteclose}\isanewline
\ \ \isakeyword{assumes}\ {\isachardoublequoteopen}v\ {\isasymnoteq}\ s{\isachardoublequoteclose}\isanewline
\ \ \isanewline
\ \ \isakeyword{shows}\ {\isachardoublequoteopen}enum\ v\ {\isasymnoteq}\ enat\ {\isadigit{0}}{\isachardoublequoteclose}\isanewline
%
\isadelimproof
%
\endisadelimproof
%
\isatagproof
\isacommand{proof}\isamarkupfalse%
\ {\isacharparenleft}cases\ {\isachardoublequoteopen}enum\ v\ {\isasymnoteq}\ {\isasyminfinity}{\isachardoublequoteclose}{\isacharparenright}\isanewline
\isacommand{case}\isamarkupfalse%
\ True\isanewline
\ \ \isacommand{then}\isamarkupfalse%
\ \isacommand{obtain}\isamarkupfalse%
\ ku\ \isakeyword{where}\ {\isachardoublequoteopen}enum\ v\ {\isacharequal}\ ku\ {\isacharplus}\ enat\ {\isadigit{1}}{\isachardoublequoteclose}\ \isanewline
\ \ \ \ \isacommand{using}\isamarkupfalse%
\ assms\ just\ \isacommand{by}\isamarkupfalse%
\ blast\isanewline
\ \ \isacommand{thus}\isamarkupfalse%
\ {\isacharquery}thesis\ \ \isacommand{by}\isamarkupfalse%
\ {\isacharparenleft}induct\ ku{\isacharparenright}\ auto\isanewline
\isacommand{qed}\isamarkupfalse%
\ fast%
\endisatagproof
{\isafoldproof}%
%
\isadelimproof
\isanewline
%
\endisadelimproof
\isanewline
\isacommand{lemma}\isamarkupfalse%
\ {\isacharparenleft}\isakeyword{in}\ shortest{\isacharunderscore}path{\isacharunderscore}non{\isacharunderscore}neg{\isacharunderscore}cost{\isacharparenright}\ dist{\isacharunderscore}ne{\isacharunderscore}ninf{\isacharcolon}\isanewline
\ \ \isakeyword{fixes}\ v\ {\isacharcolon}{\isacharcolon}\ {\isacharprime}a\isanewline
\ \ \isakeyword{assumes}\ {\isachardoublequoteopen}v\ {\isasymin}\ verts\ G{\isachardoublequoteclose}\isanewline
\ \ \isakeyword{shows}\ {\isachardoublequoteopen}dist\ v\ {\isasymnoteq}\ {\isacharminus}{\isasyminfinity}{\isachardoublequoteclose}\isanewline
%
\isadelimproof
%
\endisadelimproof
%
\isatagproof
\isacommand{proof}\isamarkupfalse%
\ {\isacharparenleft}cases\ {\isachardoublequoteopen}enum\ v\ {\isacharequal}\ {\isasyminfinity}{\isachardoublequoteclose}{\isacharparenright}\isanewline
\isacommand{case}\isamarkupfalse%
\ False\ \isanewline
\ \ \isacommand{obtain}\isamarkupfalse%
\ n\ \isakeyword{where}\ {\isachardoublequoteopen}enat\ n\ {\isacharequal}\ enum\ v{\isachardoublequoteclose}\isanewline
\ \ \ \ \isacommand{using}\isamarkupfalse%
\ False\ \isacommand{by}\isamarkupfalse%
\ force\isanewline
\ \ \isacommand{thus}\isamarkupfalse%
\ {\isacharquery}thesis\ \isacommand{using}\isamarkupfalse%
\ assms\ False\isanewline
\ \ \isacommand{proof}\isamarkupfalse%
{\isacharparenleft}induct\ n\ arbitrary{\isacharcolon}\ v{\isacharparenright}\ \isanewline
\ \ \isacommand{case}\isamarkupfalse%
\ {\isadigit{0}}\ \isacommand{thus}\isamarkupfalse%
\ {\isacharquery}case\ \isanewline
\ \ \ \ \isacommand{using}\isamarkupfalse%
\ enum{\isacharunderscore}not{\isadigit{0}}\ source{\isacharunderscore}val\ \isacommand{by}\isamarkupfalse%
\ {\isacharparenleft}cases\ {\isachardoublequoteopen}v{\isacharequal}s{\isachardoublequoteclose}{\isacharcomma}\ auto{\isacharparenright}\ \isanewline
\ \ \isacommand{next}\isamarkupfalse%
\isanewline
\ \ \isacommand{case}\isamarkupfalse%
\ {\isacharparenleft}Suc\ n{\isacharparenright}\isanewline
\ \ \ \ \isacommand{thus}\isamarkupfalse%
\ {\isacharquery}case\ \isanewline
\ \ \ \ \isacommand{proof}\isamarkupfalse%
\ {\isacharparenleft}cases\ {\isachardoublequoteopen}v{\isacharequal}s{\isachardoublequoteclose}{\isacharparenright}\ \isanewline
\ \ \ \ \isacommand{case}\isamarkupfalse%
\ True\ \isanewline
\ \ \ \ \ \ \isacommand{thus}\isamarkupfalse%
\ {\isacharquery}thesis\ \isacommand{using}\isamarkupfalse%
\ source{\isacharunderscore}val\ \isacommand{by}\isamarkupfalse%
\ simp\isanewline
\ \ \ \ \isacommand{next}\isamarkupfalse%
\isanewline
\ \ \ \ \isacommand{case}\isamarkupfalse%
\ False\isanewline
\ \ \ \ \ \ \isacommand{obtain}\isamarkupfalse%
\ e\ \isakeyword{where}\ e{\isacharunderscore}assms{\isacharcolon}\isanewline
\ \ \ \ \ \ \ \ {\isachardoublequoteopen}e\ {\isasymin}\ arcs\ G{\isachardoublequoteclose}\isanewline
\ \ \ \ \ \ \ \ {\isachardoublequoteopen}dist\ v\ {\isacharequal}\ dist\ {\isacharparenleft}tail\ G\ e{\isacharparenright}\ {\isacharplus}\ ereal\ {\isacharparenleft}c\ e{\isacharparenright}{\isachardoublequoteclose}\ \isanewline
\ \ \ \ \ \ \ \ {\isachardoublequoteopen}enum\ v\ {\isacharequal}\ enum\ {\isacharparenleft}tail\ G\ e{\isacharparenright}\ {\isacharplus}\ enat\ {\isadigit{1}}{\isachardoublequoteclose}\ \isanewline
\ \ \ \ \ \ \ \ \isacommand{using}\isamarkupfalse%
\ just{\isacharbrackleft}OF\ Suc{\isacharparenleft}{\isadigit{3}}{\isacharparenright}\ False\ Suc{\isacharparenleft}{\isadigit{4}}{\isacharparenright}{\isacharbrackright}\ \isacommand{by}\isamarkupfalse%
\ blast\isanewline
\ \ \ \ \ \ \isacommand{then}\isamarkupfalse%
\ \isacommand{have}\isamarkupfalse%
\ nsinf{\isacharcolon}{\isachardoublequoteopen}enum\ {\isacharparenleft}tail\ G\ e{\isacharparenright}\ {\isasymnoteq}\ {\isasyminfinity}{\isachardoublequoteclose}\ \isanewline
\ \ \ \ \ \ \ \ \isacommand{by}\isamarkupfalse%
\ {\isacharparenleft}metis\ Suc{\isacharparenleft}{\isadigit{2}}{\isacharparenright}\ enat{\isachardot}simps{\isacharparenleft}{\isadigit{3}}{\isacharparenright}\ enat{\isacharunderscore}{\isadigit{1}}\ plus{\isacharunderscore}enat{\isacharunderscore}simps{\isacharparenleft}{\isadigit{2}}{\isacharparenright}{\isacharparenright}\isanewline
\ \ \ \ \ \ \isacommand{then}\isamarkupfalse%
\ \isacommand{have}\isamarkupfalse%
\ ns{\isacharcolon}{\isachardoublequoteopen}enat\ n\ {\isacharequal}\ enum\ {\isacharparenleft}tail\ G\ e{\isacharparenright}{\isachardoublequoteclose}\ \isanewline
\ \ \ \ \ \ \ \ \isacommand{using}\isamarkupfalse%
\ e{\isacharunderscore}assms{\isacharparenleft}{\isadigit{3}}{\isacharparenright}\ Suc{\isacharparenleft}{\isadigit{2}}{\isacharparenright}\ \isacommand{by}\isamarkupfalse%
\ force\isanewline
\ \ \ \ \ \ \isacommand{have}\isamarkupfalse%
\ {\isachardoublequoteopen}dist\ {\isacharparenleft}tail\ G\ e\ {\isacharparenright}\ {\isasymnoteq}\ {\isacharminus}\ {\isasyminfinity}{\isachardoublequoteclose}\ \isanewline
\ \ \ \ \ \ \ \ \isacommand{by}\isamarkupfalse%
\ {\isacharparenleft}rule\ Suc{\isacharparenleft}{\isadigit{1}}{\isacharparenright}\ {\isacharbrackleft}OF\ ns\ tail{\isacharunderscore}in{\isacharunderscore}verts{\isacharbrackleft}OF\ e{\isacharunderscore}assms{\isacharparenleft}{\isadigit{1}}{\isacharparenright}{\isacharbrackright}\ nsinf{\isacharbrackright}{\isacharparenright}\isanewline
\ \ \ \ \ \ \isacommand{thus}\isamarkupfalse%
\ {\isacharquery}thesis\ \isacommand{using}\isamarkupfalse%
\ e{\isacharunderscore}assms{\isacharparenleft}{\isadigit{2}}{\isacharparenright}\ \isacommand{by}\isamarkupfalse%
\ simp\isanewline
\ \ \ \ \isacommand{qed}\isamarkupfalse%
\isanewline
\ \ \isacommand{qed}\isamarkupfalse%
\isanewline
\isacommand{next}\isamarkupfalse%
\isanewline
\isacommand{case}\isamarkupfalse%
\ True\ \isanewline
\ \ \isacommand{thus}\isamarkupfalse%
\ {\isacharquery}thesis\ \isacommand{using}\isamarkupfalse%
\ no{\isacharunderscore}path{\isacharbrackleft}OF\ assms{\isacharbrackright}\ \isacommand{by}\isamarkupfalse%
\ simp\isanewline
\isacommand{qed}\isamarkupfalse%
%
\endisatagproof
{\isafoldproof}%
%
\isadelimproof
\isanewline
%
\endisadelimproof
\isanewline
\isacommand{theorem}\isamarkupfalse%
\ {\isacharparenleft}\isakeyword{in}\ shortest{\isacharunderscore}path{\isacharunderscore}non{\isacharunderscore}neg{\isacharunderscore}cost{\isacharparenright}\ correct{\isacharunderscore}shortest{\isacharunderscore}path{\isacharcolon}\isanewline
\ \ \isakeyword{fixes}\ v\ {\isacharcolon}{\isacharcolon}\ {\isacharprime}a\isanewline
\ \ \isakeyword{assumes}\ {\isachardoublequoteopen}v\ {\isasymin}\ verts\ G{\isachardoublequoteclose}\isanewline
\ \ \isakeyword{shows}\ {\isachardoublequoteopen}dist\ v\ {\isacharequal}\ {\isasymmu}\ c\ s\ v{\isachardoublequoteclose}\isanewline
%
\isadelimproof
\ \ %
\endisadelimproof
%
\isatagproof
\isacommand{using}\isamarkupfalse%
\ no{\isacharunderscore}path{\isacharbrackleft}OF\ assms{\isacharparenleft}{\isadigit{1}}{\isacharparenright}{\isacharbrackright}\ dist{\isacharunderscore}le{\isacharunderscore}{\isasymmu}{\isacharbrackleft}OF\ assms{\isacharparenleft}{\isadigit{1}}{\isacharparenright}{\isacharbrackright}\isanewline
\ \ \ \ dist{\isacharunderscore}ge{\isacharunderscore}{\isasymmu}{\isacharbrackleft}OF\ assms{\isacharparenleft}{\isadigit{1}}{\isacharparenright}\ {\isacharunderscore}\ dist{\isacharunderscore}ne{\isacharunderscore}ninf{\isacharbrackleft}OF\ assms{\isacharparenleft}{\isadigit{1}}{\isacharparenright}{\isacharbrackright}\ \isanewline
\ \ \ \ tail{\isacharunderscore}value{\isacharunderscore}check{\isacharbrackleft}OF\ s{\isacharunderscore}in{\isacharunderscore}G{\isacharbrackright}\ source{\isacharunderscore}val\ enum{\isacharunderscore}not{\isadigit{0}}{\isacharbrackright}\ \isanewline
\ \ \ \ \isacommand{by}\isamarkupfalse%
\ fastforce%
\endisatagproof
{\isafoldproof}%
%
\isadelimproof
\isanewline
%
\endisadelimproof
\isanewline
\isacommand{corollary}\isamarkupfalse%
\ {\isacharparenleft}\isakeyword{in}\ shortest{\isacharunderscore}path{\isacharunderscore}non{\isacharunderscore}neg{\isacharunderscore}cost{\isacharunderscore}pred{\isacharparenright}\ correct{\isacharunderscore}shortest{\isacharunderscore}path{\isacharunderscore}pred{\isacharcolon}\isanewline
\ \ \isakeyword{fixes}\ v\ {\isacharcolon}{\isacharcolon}\ {\isacharprime}a\isanewline
\ \ \isakeyword{assumes}\ {\isachardoublequoteopen}v\ {\isasymin}\ verts\ G{\isachardoublequoteclose}\isanewline
\ \ \isakeyword{shows}\ {\isachardoublequoteopen}dist\ v\ {\isacharequal}\ {\isasymmu}\ c\ s\ v{\isachardoublequoteclose}\isanewline
%
\isadelimproof
\ \ %
\endisadelimproof
%
\isatagproof
\isacommand{using}\isamarkupfalse%
\ correct{\isacharunderscore}shortest{\isacharunderscore}path\ assms\ \isacommand{by}\isamarkupfalse%
\ simp%
\endisatagproof
{\isafoldproof}%
%
\isadelimproof
\isanewline
%
\endisadelimproof
%
\isadelimtheory
\isanewline
%
\endisadelimtheory
%
\isatagtheory
\isacommand{end}\isamarkupfalse%
%
\endisatagtheory
{\isafoldtheory}%
%
\isadelimtheory
%
\endisadelimtheory
\end{isabellebody}%
%%% Local Variables:
%%% mode: latex
%%% TeX-master: "root"
%%% End:


%
\begin{isabellebody}%
\def\isabellecontext{Check{\isacharunderscore}Connected{\isacharunderscore}Impl}%
%
\isadelimtheory
%
\endisadelimtheory
%
\isatagtheory
\isacommand{theory}\isamarkupfalse%
\ Check{\isacharunderscore}Connected{\isacharunderscore}Impl\isanewline
\isakeyword{imports}\isanewline
\ \ {\isachardoublequoteopen}Vcg{\isachardoublequoteclose}\isanewline
\ \ {\isachardoublequoteopen}{\isachardot}{\isachardot}{\isacharslash}Witness{\isacharunderscore}Property{\isacharslash}Connected{\isacharunderscore}Components{\isachardoublequoteclose}\isanewline
\isakeyword{begin}%
\endisatagtheory
{\isafoldtheory}%
%
\isadelimtheory
\isanewline
%
\endisadelimtheory
\isanewline
\isacommand{type{\isacharunderscore}synonym}\isamarkupfalse%
\ IVertex\ {\isacharequal}\ nat\isanewline
\isacommand{type{\isacharunderscore}synonym}\isamarkupfalse%
\ IEdge{\isacharunderscore}Id\ {\isacharequal}\ nat\isanewline
\isacommand{type{\isacharunderscore}synonym}\isamarkupfalse%
\ IEdge\ {\isacharequal}\ {\isachardoublequoteopen}IVertex\ {\isasymtimes}\ IVertex{\isachardoublequoteclose}\isanewline
\isacommand{type{\isacharunderscore}synonym}\isamarkupfalse%
\ IPEdge\ {\isacharequal}\ {\isachardoublequoteopen}IVertex\ {\isasymRightarrow}\ IEdge{\isacharunderscore}Id\ option{\isachardoublequoteclose}\isanewline
\isacommand{type{\isacharunderscore}synonym}\isamarkupfalse%
\ INum\ {\isacharequal}\ {\isachardoublequoteopen}IVertex\ {\isasymRightarrow}\ nat{\isachardoublequoteclose}\isanewline
\isacommand{type{\isacharunderscore}synonym}\isamarkupfalse%
\ IGraph\ {\isacharequal}\ {\isachardoublequoteopen}nat\ {\isasymtimes}\ nat\ {\isasymtimes}\ {\isacharparenleft}IEdge{\isacharunderscore}Id\ {\isasymRightarrow}\ IEdge{\isacharparenright}{\isachardoublequoteclose}\ \isanewline
\isanewline
\isacommand{abbreviation}\isamarkupfalse%
\ ivertex{\isacharunderscore}cnt\ {\isacharcolon}{\isacharcolon}\ {\isachardoublequoteopen}IGraph\ {\isasymRightarrow}\ nat{\isachardoublequoteclose}\isanewline
\ \ \isakeyword{where}\ {\isachardoublequoteopen}ivertex{\isacharunderscore}cnt\ G\ {\isasymequiv}\ fst\ G{\isachardoublequoteclose}\isanewline
\isanewline
\isacommand{abbreviation}\isamarkupfalse%
\ iedge{\isacharunderscore}cnt\ {\isacharcolon}{\isacharcolon}\ {\isachardoublequoteopen}IGraph\ {\isasymRightarrow}\ nat{\isachardoublequoteclose}\isanewline
\ \ \isakeyword{where}\ {\isachardoublequoteopen}iedge{\isacharunderscore}cnt\ G\ {\isasymequiv}\ fst\ {\isacharparenleft}snd\ G{\isacharparenright}{\isachardoublequoteclose}\isanewline
\isanewline
\isacommand{abbreviation}\isamarkupfalse%
\ iedges\ {\isacharcolon}{\isacharcolon}\ {\isachardoublequoteopen}IGraph\ {\isasymRightarrow}\ IEdge{\isacharunderscore}Id\ {\isasymRightarrow}\ IEdge{\isachardoublequoteclose}\isanewline
\ \ \isakeyword{where}\ {\isachardoublequoteopen}iedges\ G\ {\isasymequiv}\ snd\ {\isacharparenleft}snd\ G{\isacharparenright}{\isachardoublequoteclose}\isanewline
\isanewline
\isacommand{definition}\isamarkupfalse%
\ is{\isacharunderscore}wellformed{\isacharunderscore}inv\ {\isacharcolon}{\isacharcolon}\ {\isachardoublequoteopen}IGraph\ {\isasymRightarrow}\ nat\ {\isasymRightarrow}\ bool{\isachardoublequoteclose}\ \isakeyword{where}\isanewline
\ \ {\isachardoublequoteopen}is{\isacharunderscore}wellformed{\isacharunderscore}inv\ G\ i\ {\isasymequiv}\ {\isasymforall}k\ {\isacharless}\ i{\isachardot}\ ivertex{\isacharunderscore}cnt\ G\ {\isachargreater}\ fst\ {\isacharparenleft}iedges\ G\ k{\isacharparenright}\isanewline
\ \ \ \ \ \ \ \ {\isasymand}\ ivertex{\isacharunderscore}cnt\ G\ {\isachargreater}\ snd\ {\isacharparenleft}iedges\ G\ k{\isacharparenright}{\isachardoublequoteclose}\isanewline
%
\isadelimML
%
\endisadelimML
%
\isatagML
\isacommand{ML}\isamarkupfalse%
\ {\isacharverbatimopen}\ Toplevel{\isachardot}theory\ {\isacharverbatimclose}%
\endisatagML
{\isafoldML}%
%
\isadelimML
\isanewline
%
\endisadelimML
\isacommand{procedures}\isamarkupfalse%
\ is{\isacharunderscore}wellformed\ {\isacharparenleft}G\ {\isacharcolon}{\isacharcolon}\ IGraph\ {\isacharbar}\ R\ {\isacharcolon}{\isacharcolon}\ bool{\isacharparenright}\isanewline
\ \ \isakeyword{where}\isanewline
\ \ \ \ i\ {\isacharcolon}{\isacharcolon}\ nat\isanewline
\ \ \ \ e\ {\isacharcolon}{\isacharcolon}\ IEdge\isanewline
\ \ \isakeyword{in}\ {\isachardoublequoteopen}\isanewline
\ \ \ \ ANNO\ G{\isachardot}{\isasymlbrace}\ {\isasymacute}G\ {\isacharequal}\ G\ {\isasymrbrace}\isanewline
\ \ \ \ \ \ {\isasymacute}R\ {\isacharcolon}{\isacharequal}{\isacharequal}\ True\ {\isacharsemicolon}{\isacharsemicolon}\isanewline
\ \ \ \ \ \ {\isasymacute}i\ {\isacharcolon}{\isacharequal}{\isacharequal}\ {\isadigit{0}}\ {\isacharsemicolon}{\isacharsemicolon}\isanewline
\ \ \ \ \ \ TRY\isanewline
\ \ \ \ \ \ \ \ WHILE\ {\isasymacute}i\ {\isacharless}\ iedge{\isacharunderscore}cnt\ {\isasymacute}G\isanewline
\ \ \ \ \ \ \ \ INV\ {\isasymlbrace}\ {\isasymacute}R\ {\isacharequal}\ is{\isacharunderscore}wellformed{\isacharunderscore}inv\ {\isasymacute}G\ {\isasymacute}i\ {\isasymand}\ \isanewline
\ \ \ \ \ \ \ \ \ \ \ \ \ \ {\isasymacute}i\ {\isasymle}\ iedge{\isacharunderscore}cnt\ {\isasymacute}G\ {\isasymand}\ {\isasymacute}G\ {\isacharequal}\ G\ {\isasymrbrace}\isanewline
\ \ \ \ \ \ \ \ VAR\ MEASURE\ {\isacharparenleft}iedge{\isacharunderscore}cnt\ {\isasymacute}G\ {\isacharminus}\ {\isasymacute}i{\isacharparenright}\isanewline
\ \ \ \ \ \ \ \ DO\isanewline
\ \ \ \ \ \ \ \ \ {\isasymacute}e\ {\isacharcolon}{\isacharequal}{\isacharequal}\ iedges\ {\isasymacute}G\ {\isasymacute}i\ {\isacharsemicolon}{\isacharsemicolon}\isanewline
\ \ \ \ \ \ \ \ \ IF\ ivertex{\isacharunderscore}cnt\ {\isasymacute}G\ {\isasymle}\ fst\ {\isasymacute}e\ {\isasymor}\ ivertex{\isacharunderscore}cnt\ {\isasymacute}G\ {\isasymle}\ snd\ {\isasymacute}e\ THEN\isanewline
\ \ \ \ \ \ \ \ \ \ \ {\isasymacute}R\ {\isacharcolon}{\isacharequal}{\isacharequal}\ False\ {\isacharsemicolon}{\isacharsemicolon}\isanewline
\ \ \ \ \ \ \ \ \ \ \ THROW\isanewline
\ \ \ \ \ \ \ \ \ FI\ {\isacharsemicolon}{\isacharsemicolon}\isanewline
\ \ \ \ \ \ \ \ \ {\isasymacute}i\ {\isacharcolon}{\isacharequal}{\isacharequal}\ {\isasymacute}i\ {\isacharplus}\ {\isadigit{1}}\isanewline
\ \ \ \ \ \ \ \ OD\isanewline
\ \ \ \ \ \ CATCH\ SKIP\ END\isanewline
\ \ \ \ \ \ {\isasymlbrace}\ {\isasymacute}G\ {\isacharequal}\ G\ {\isasymand}\ \isanewline
\ \ \ \ \ \ \ \ {\isasymacute}R\ {\isacharequal}\ is{\isacharunderscore}wellformed{\isacharunderscore}inv\ {\isasymacute}G\ {\isacharparenleft}iedge{\isacharunderscore}cnt\ {\isasymacute}G{\isacharparenright}\ {\isasymrbrace}{\isachardoublequoteclose}\isanewline
\isanewline
\isacommand{definition}\isamarkupfalse%
\ parent{\isacharunderscore}num{\isacharunderscore}assms{\isacharunderscore}inv\ {\isacharcolon}{\isacharcolon}\ {\isachardoublequoteopen}IGraph\ {\isasymRightarrow}\ IVertex\ {\isasymRightarrow}\ IPEdge\ {\isasymRightarrow}\ INum\ {\isasymRightarrow}\ nat\ {\isasymRightarrow}\ bool{\isachardoublequoteclose}\ \isakeyword{where}\isanewline
\ \ {\isachardoublequoteopen}parent{\isacharunderscore}num{\isacharunderscore}assms{\isacharunderscore}inv\ G\ r\ p\ n\ k\ {\isasymequiv}\ {\isasymforall}i\ {\isacharless}\ k{\isachardot}\ i\ {\isasymnoteq}\ r\ {\isasymlongrightarrow}\ {\isacharparenleft}case\ p\ i\ of\isanewline
\ \ \ \ \ \ None\ {\isasymRightarrow}\ False\isanewline
\ \ \ \ {\isacharbar}\ Some\ x\ {\isasymRightarrow}\ x\ {\isacharless}\ iedge{\isacharunderscore}cnt\ G\ {\isasymand}\ snd\ {\isacharparenleft}iedges\ G\ x{\isacharparenright}\ {\isacharequal}\ i\ {\isasymand}\ n\ i\ {\isacharequal}\ n\ {\isacharparenleft}fst\ {\isacharparenleft}iedges\ G\ x{\isacharparenright}{\isacharparenright}\ {\isacharplus}\ {\isadigit{1}}{\isacharparenright}{\isachardoublequoteclose}\isanewline
\isanewline
\isacommand{procedures}\isamarkupfalse%
\ parent{\isacharunderscore}num{\isacharunderscore}assms\ {\isacharparenleft}G\ {\isacharcolon}{\isacharcolon}\ IGraph{\isacharcomma}\ r\ {\isacharcolon}{\isacharcolon}\ IVertex{\isacharcomma}\ parent{\isacharunderscore}edge\ {\isacharcolon}{\isacharcolon}\ IPEdge{\isacharcomma}\isanewline
\ \ \ \ num\ {\isacharcolon}{\isacharcolon}\ INum\ {\isacharbar}\ R\ {\isacharcolon}{\isacharcolon}\ bool{\isacharparenright}\isanewline
\ \ \isakeyword{where}\isanewline
\ \ \ \ vertex\ {\isacharcolon}{\isacharcolon}\ IVertex\isanewline
\ \ \ \ edge{\isacharunderscore}id\ {\isacharcolon}{\isacharcolon}\ IEdge{\isacharunderscore}Id\isanewline
\ \ \isakeyword{in}\ {\isachardoublequoteopen}\isanewline
\ \ \ \ ANNO\ {\isacharparenleft}G{\isacharcomma}r{\isacharcomma}p{\isacharcomma}n{\isacharparenright}{\isachardot}\isanewline
\ \ \ \ \ \ {\isasymlbrace}\ {\isasymacute}G\ {\isacharequal}\ G\ {\isasymand}\ {\isasymacute}r\ {\isacharequal}\ r\ {\isasymand}\ {\isasymacute}parent{\isacharunderscore}edge\ {\isacharequal}\ p\ {\isasymand}\ {\isasymacute}num\ {\isacharequal}\ n\ {\isasymrbrace}\isanewline
\ \ \ \ \ \ {\isasymacute}R\ {\isacharcolon}{\isacharequal}{\isacharequal}\ True\ {\isacharsemicolon}{\isacharsemicolon}\isanewline
\ \ \ \ \ \ {\isasymacute}vertex\ {\isacharcolon}{\isacharequal}{\isacharequal}\ {\isadigit{0}}\ {\isacharsemicolon}{\isacharsemicolon}\isanewline
\ \ \ \ \ \ TRY\isanewline
\ \ \ \ \ \ \ \ WHILE\ {\isasymacute}vertex\ {\isacharless}\ ivertex{\isacharunderscore}cnt\ {\isasymacute}G\isanewline
\ \ \ \ \ \ \ \ INV\ {\isasymlbrace}\ {\isasymacute}R\ {\isacharequal}\ parent{\isacharunderscore}num{\isacharunderscore}assms{\isacharunderscore}inv\ {\isasymacute}G\ {\isasymacute}r\ {\isasymacute}parent{\isacharunderscore}edge\ {\isasymacute}num\ {\isasymacute}vertex\isanewline
\ \ \ \ \ \ \ \ \ \ {\isasymand}\ {\isasymacute}G\ {\isacharequal}\ G\ {\isasymand}\ {\isasymacute}r\ {\isacharequal}\ r\ {\isasymand}\ {\isasymacute}parent{\isacharunderscore}edge\ {\isacharequal}\ p\ {\isasymand}\ {\isasymacute}num\ {\isacharequal}\ n\isanewline
\ \ \ \ \ \ \ \ \ \ {\isasymand}\ {\isasymacute}vertex\ {\isasymle}\ ivertex{\isacharunderscore}cnt\ {\isasymacute}G{\isasymrbrace}\isanewline
\ \ \ \ \ \ \ \ VAR\ MEASURE\ {\isacharparenleft}ivertex{\isacharunderscore}cnt\ {\isasymacute}G\ {\isacharminus}\ {\isasymacute}vertex{\isacharparenright}\isanewline
\ \ \ \ \ \ \ \ DO\isanewline
\ \ \ \ \ \ \ \ \ \ IF\ {\isacharparenleft}{\isasymacute}vertex\ {\isasymnoteq}\ {\isasymacute}r{\isacharparenright}\ THEN\isanewline
\ \ \ \ \ \ \ \ \ \ \ \ IF\ {\isasymacute}parent{\isacharunderscore}edge\ {\isasymacute}vertex\ {\isacharequal}\ None\ THEN\isanewline
\ \ \ \ \ \ \ \ \ \ \ \ \ \ {\isasymacute}R\ {\isacharcolon}{\isacharequal}{\isacharequal}\ False\ {\isacharsemicolon}{\isacharsemicolon}\isanewline
\ \ \ \ \ \ \ \ \ \ \ \ \ \ THROW\isanewline
\ \ \ \ \ \ \ \ \ \ \ \ FI\ {\isacharsemicolon}{\isacharsemicolon}\isanewline
\ \ \ \ \ \ \ \ \ \ \ \ {\isasymacute}edge{\isacharunderscore}id\ {\isacharcolon}{\isacharequal}{\isacharequal}\ the\ {\isacharparenleft}{\isasymacute}parent{\isacharunderscore}edge\ {\isasymacute}vertex{\isacharparenright}\ {\isacharsemicolon}{\isacharsemicolon}\isanewline
\ \ \ \ \ \ \ \ \ \ \ \ IF\ {\isasymacute}edge{\isacharunderscore}id\ {\isasymge}\ iedge{\isacharunderscore}cnt\ {\isasymacute}G\isanewline
\ \ \ \ \ \ \ \ \ \ \ \ \ \ \ \ {\isasymor}\ snd\ {\isacharparenleft}iedges\ {\isasymacute}G\ {\isasymacute}edge{\isacharunderscore}id{\isacharparenright}\ {\isasymnoteq}\ {\isasymacute}vertex\isanewline
\ \ \ \ \ \ \ \ \ \ \ \ \ \ \ \ {\isasymor}\ {\isasymacute}num\ {\isasymacute}vertex\ {\isasymnoteq}\ {\isasymacute}num\ {\isacharparenleft}fst\ {\isacharparenleft}iedges\ {\isasymacute}G\ {\isasymacute}edge{\isacharunderscore}id{\isacharparenright}{\isacharparenright}\ {\isacharplus}\ {\isadigit{1}}\ THEN\isanewline
\ \ \ \ \ \ \ \ \ \ \ \ \ \ {\isasymacute}R\ {\isacharcolon}{\isacharequal}{\isacharequal}\ False\ {\isacharsemicolon}{\isacharsemicolon}\isanewline
\ \ \ \ \ \ \ \ \ \ \ \ \ \ THROW\isanewline
\ \ \ \ \ \ \ \ \ \ \ \ FI\isanewline
\ \ \ \ \ \ \ \ \ \ FI\ {\isacharsemicolon}{\isacharsemicolon}\isanewline
\ \ \ \ \ \ \ \ \ \ {\isasymacute}vertex\ {\isacharcolon}{\isacharequal}{\isacharequal}\ {\isasymacute}vertex\ {\isacharplus}\ {\isadigit{1}}\isanewline
\ \ \ \ \ \ \ \ OD\isanewline
\ \ \ \ \ \ CATCH\ SKIP\ END\isanewline
\ \ \ \ {\isasymlbrace}\ {\isasymacute}G\ {\isacharequal}\ G\ {\isasymand}\ {\isasymacute}r\ {\isacharequal}\ r\ {\isasymand}\ {\isasymacute}parent{\isacharunderscore}edge\ {\isacharequal}\ p\ {\isasymand}\ {\isasymacute}num\ {\isacharequal}\ n\isanewline
\ \ \ \ \ \ {\isasymand}\ {\isasymacute}R\ {\isacharequal}\ parent{\isacharunderscore}num{\isacharunderscore}assms{\isacharunderscore}inv\ {\isasymacute}G\ {\isasymacute}r\ {\isasymacute}parent{\isacharunderscore}edge\ {\isasymacute}num\ {\isacharparenleft}ivertex{\isacharunderscore}cnt\ {\isasymacute}G{\isacharparenright}{\isasymrbrace}{\isachardoublequoteclose}\isanewline
\isanewline
\isacommand{procedures}\isamarkupfalse%
\ check{\isacharunderscore}connected\ {\isacharparenleft}G\ {\isacharcolon}{\isacharcolon}\ IGraph{\isacharcomma}\ r\ {\isacharcolon}{\isacharcolon}\ IVertex{\isacharcomma}\ parent{\isacharunderscore}edge\ {\isacharcolon}{\isacharcolon}\ IPEdge{\isacharcomma}\isanewline
\ \ \ \ num\ {\isacharcolon}{\isacharcolon}\ INum\ {\isacharbar}\ R\ {\isacharcolon}{\isacharcolon}\ bool{\isacharparenright}\isanewline
\ \ \isakeyword{where}\isanewline
\ \ \ \ R{\isadigit{1}}\ {\isacharcolon}{\isacharcolon}\ bool\isanewline
\ \ \ \ R{\isadigit{2}}\ {\isacharcolon}{\isacharcolon}\ bool\isanewline
\ \ \ \ R{\isadigit{3}}\ {\isacharcolon}{\isacharcolon}\ bool\isanewline
\ \ \isakeyword{in}\ {\isachardoublequoteopen}\isanewline
\ \ \ \ {\isasymacute}R{\isadigit{1}}\ {\isacharcolon}{\isacharequal}{\isacharequal}\ CALL\ is{\isacharunderscore}wellformed{\isacharparenleft}{\isasymacute}G{\isacharparenright}\ {\isacharsemicolon}{\isacharsemicolon}\isanewline
\ \ \ \ {\isasymacute}R{\isadigit{2}}\ {\isacharcolon}{\isacharequal}{\isacharequal}\ {\isasymacute}r\ {\isacharless}\ ivertex{\isacharunderscore}cnt\ {\isasymacute}G\ {\isasymand}\ {\isasymacute}num\ {\isasymacute}r\ {\isacharequal}\ {\isadigit{0}}\ {\isasymand}\ {\isasymacute}parent{\isacharunderscore}edge\ {\isasymacute}r\ {\isacharequal}\ None\ {\isacharsemicolon}{\isacharsemicolon}\isanewline
\ \ \ \ {\isasymacute}R{\isadigit{3}}\ {\isacharcolon}{\isacharequal}{\isacharequal}\ CALL\ parent{\isacharunderscore}num{\isacharunderscore}assms{\isacharparenleft}{\isasymacute}G{\isacharcomma}\ {\isasymacute}r{\isacharcomma}\ {\isasymacute}parent{\isacharunderscore}edge{\isacharcomma}\ {\isasymacute}num{\isacharparenright}\ {\isacharsemicolon}{\isacharsemicolon}\isanewline
\ \ \ \ {\isasymacute}R\ {\isacharcolon}{\isacharequal}{\isacharequal}\ {\isasymacute}R{\isadigit{1}}\ {\isasymand}\ {\isasymacute}R{\isadigit{2}}\ {\isasymand}\ {\isasymacute}R{\isadigit{3}}{\isachardoublequoteclose}\isanewline
%
\isadelimtheory
\isanewline
%
\endisadelimtheory
%
\isatagtheory
\isacommand{end}\isamarkupfalse%
%
\endisatagtheory
{\isafoldtheory}%
%
\isadelimtheory
%
\endisadelimtheory
\end{isabellebody}%
%%% Local Variables:
%%% mode: latex
%%% TeX-master: "root"
%%% End:


%
\begin{isabellebody}%
\def\isabellecontext{Check{\isacharunderscore}Connected{\isacharunderscore}Verification}%
%
\isadelimtheory
%
\endisadelimtheory
%
\isatagtheory
\isacommand{theory}\isamarkupfalse%
\ Check{\isacharunderscore}Connected{\isacharunderscore}Verification\isanewline
\isakeyword{imports}\ Vcg\ Check{\isacharunderscore}Connected{\isacharunderscore}Impl\isanewline
\isakeyword{begin}%
\endisatagtheory
{\isafoldtheory}%
%
\isadelimtheory
\isanewline
%
\endisadelimtheory
\isanewline
\isacommand{definition}\isamarkupfalse%
\ no{\isacharunderscore}loops\ {\isacharcolon}{\isacharcolon}\ {\isachardoublequoteopen}{\isacharparenleft}{\isacharprime}a{\isacharcomma}\ {\isacharprime}b{\isacharparenright}\ pre{\isacharunderscore}digraph\ {\isasymRightarrow}\ bool{\isachardoublequoteclose}\ \isakeyword{where}\isanewline
\ \ {\isachardoublequoteopen}no{\isacharunderscore}loops\ G\ {\isasymequiv}\ {\isasymforall}e\ {\isasymin}\ arcs\ G{\isachardot}\ tail\ G\ e\ {\isasymnoteq}\ head\ G\ e{\isachardoublequoteclose}\isanewline
\isanewline
\isacommand{definition}\isamarkupfalse%
\ abs{\isacharunderscore}IGraph\ {\isacharcolon}{\isacharcolon}\ {\isachardoublequoteopen}IGraph\ {\isasymRightarrow}\ {\isacharparenleft}nat{\isacharcomma}\ nat{\isacharparenright}\ pre{\isacharunderscore}digraph{\isachardoublequoteclose}\ \isakeyword{where}\isanewline
\ \ {\isachardoublequoteopen}abs{\isacharunderscore}IGraph\ G\ {\isasymequiv}\ {\isasymlparr}\ verts\ {\isacharequal}\ {\isacharbraceleft}{\isadigit{0}}{\isachardot}{\isachardot}{\isacharless}ivertex{\isacharunderscore}cnt\ G{\isacharbraceright}{\isacharcomma}\ arcs\ {\isacharequal}\ {\isacharbraceleft}{\isadigit{0}}{\isachardot}{\isachardot}{\isacharless}iedge{\isacharunderscore}cnt\ G{\isacharbraceright}{\isacharcomma}\isanewline
\ \ \ \ tail\ {\isacharequal}\ fst\ o\ iedges\ G{\isacharcomma}\ head\ {\isacharequal}\ snd\ o\ iedges\ G\ {\isasymrparr}{\isachardoublequoteclose}\isanewline
\isanewline
\isacommand{lemma}\isamarkupfalse%
\ verts{\isacharunderscore}absI{\isacharbrackleft}simp{\isacharbrackright}{\isacharcolon}\ {\isachardoublequoteopen}verts\ {\isacharparenleft}abs{\isacharunderscore}IGraph\ G{\isacharparenright}\ {\isacharequal}\ {\isacharbraceleft}{\isadigit{0}}{\isachardot}{\isachardot}{\isacharless}ivertex{\isacharunderscore}cnt\ G{\isacharbraceright}{\isachardoublequoteclose}\isanewline
\ \ \isakeyword{and}\ arcs{\isacharunderscore}absI{\isacharbrackleft}simp{\isacharbrackright}{\isacharcolon}\ {\isachardoublequoteopen}arcs\ {\isacharparenleft}abs{\isacharunderscore}IGraph\ G{\isacharparenright}\ {\isacharequal}\ {\isacharbraceleft}{\isadigit{0}}{\isachardot}{\isachardot}{\isacharless}iedge{\isacharunderscore}cnt\ G{\isacharbraceright}{\isachardoublequoteclose}\isanewline
\ \ \isakeyword{and}\ tail{\isacharunderscore}absI{\isacharbrackleft}simp{\isacharbrackright}{\isacharcolon}\ {\isachardoublequoteopen}tail\ {\isacharparenleft}abs{\isacharunderscore}IGraph\ G{\isacharparenright}\ e\ {\isacharequal}\ fst\ {\isacharparenleft}iedges\ G\ e{\isacharparenright}{\isachardoublequoteclose}\isanewline
\ \ \isakeyword{and}\ head{\isacharunderscore}absI{\isacharbrackleft}simp{\isacharbrackright}{\isacharcolon}\ {\isachardoublequoteopen}head\ {\isacharparenleft}abs{\isacharunderscore}IGraph\ G{\isacharparenright}\ e\ {\isacharequal}\ snd\ {\isacharparenleft}iedges\ G\ e{\isacharparenright}{\isachardoublequoteclose}\isanewline
%
\isadelimproof
\ \ %
\endisadelimproof
%
\isatagproof
\isacommand{by}\isamarkupfalse%
\ {\isacharparenleft}auto\ simp{\isacharcolon}\ abs{\isacharunderscore}IGraph{\isacharunderscore}def{\isacharparenright}%
\endisatagproof
{\isafoldproof}%
%
\isadelimproof
\isanewline
%
\endisadelimproof
\isanewline
\isacommand{lemma}\isamarkupfalse%
\ is{\isacharunderscore}wellformed{\isacharunderscore}inv{\isacharunderscore}step{\isacharcolon}\isanewline
\ \ {\isachardoublequoteopen}is{\isacharunderscore}wellformed{\isacharunderscore}inv\ G\ {\isacharparenleft}Suc\ i{\isacharparenright}\ {\isasymlongleftrightarrow}\ is{\isacharunderscore}wellformed{\isacharunderscore}inv\ G\ i\isanewline
\ \ \ \ \ \ {\isasymand}\ fst\ {\isacharparenleft}iedges\ G\ i{\isacharparenright}\ {\isacharless}\ ivertex{\isacharunderscore}cnt\ G\ {\isasymand}\ snd\ {\isacharparenleft}iedges\ G\ i{\isacharparenright}\ {\isacharless}\ ivertex{\isacharunderscore}cnt\ G{\isachardoublequoteclose}\isanewline
%
\isadelimproof
\ \ %
\endisadelimproof
%
\isatagproof
\isacommand{by}\isamarkupfalse%
\ {\isacharparenleft}auto\ simp\ add{\isacharcolon}\ is{\isacharunderscore}wellformed{\isacharunderscore}inv{\isacharunderscore}def\ less{\isacharunderscore}Suc{\isacharunderscore}eq{\isacharparenright}%
\endisatagproof
{\isafoldproof}%
%
\isadelimproof
\isanewline
%
\endisadelimproof
\isanewline
\isacommand{lemma}\isamarkupfalse%
\ {\isacharparenleft}\isakeyword{in}\ is{\isacharunderscore}wellformed{\isacharunderscore}impl{\isacharparenright}\ is{\isacharunderscore}wellformed{\isacharunderscore}spec{\isacharcolon}\isanewline
\ \ {\isachardoublequoteopen}{\isasymforall}G{\isachardot}\ {\isasymGamma}\ {\isasymturnstile}\isactrlsub t\ {\isasymlbrace}{\isasymacute}G\ {\isacharequal}\ G{\isasymrbrace}\ {\isasymacute}R\ {\isacharcolon}{\isacharequal}{\isacharequal}\ PROC\ is{\isacharunderscore}wellformed{\isacharparenleft}{\isasymacute}G{\isacharparenright}\ {\isasymlbrace}{\isasymacute}R\ {\isacharequal}\ is{\isacharunderscore}wellformed{\isacharunderscore}inv\ G\ {\isacharparenleft}iedge{\isacharunderscore}cnt\ G{\isacharparenright}{\isasymrbrace}{\isachardoublequoteclose}\isanewline
%
\isadelimproof
\ \ %
\endisadelimproof
%
\isatagproof
\isacommand{apply}\isamarkupfalse%
\ vcg\isanewline
\ \ \isacommand{apply}\isamarkupfalse%
\ {\isacharparenleft}auto\ simp{\isacharcolon}\ is{\isacharunderscore}wellformed{\isacharunderscore}inv{\isacharunderscore}step{\isacharparenright}\isanewline
\ \ \isacommand{apply}\isamarkupfalse%
\ {\isacharparenleft}auto\ simp{\isacharcolon}\ is{\isacharunderscore}wellformed{\isacharunderscore}inv{\isacharunderscore}def{\isacharparenright}\isanewline
\ \ \isacommand{done}\isamarkupfalse%
%
\endisatagproof
{\isafoldproof}%
%
\isadelimproof
\isanewline
%
\endisadelimproof
\isanewline
\isacommand{lemma}\isamarkupfalse%
\ parent{\isacharunderscore}num{\isacharunderscore}assms{\isacharunderscore}inv{\isacharunderscore}step{\isacharcolon}\isanewline
\ \ {\isachardoublequoteopen}parent{\isacharunderscore}num{\isacharunderscore}assms{\isacharunderscore}inv\ G\ r\ p\ n\ {\isacharparenleft}Suc\ i{\isacharparenright}\ {\isasymlongleftrightarrow}\ parent{\isacharunderscore}num{\isacharunderscore}assms{\isacharunderscore}inv\ G\ r\ p\ n\ i\isanewline
\ \ \ \ {\isasymand}\ {\isacharparenleft}i\ {\isasymnoteq}\ r\ {\isasymlongrightarrow}\ {\isacharparenleft}case\ p\ i\ of\isanewline
\ \ \ \ \ \ None\ {\isasymRightarrow}\ False\isanewline
\ \ \ \ {\isacharbar}\ Some\ x\ {\isasymRightarrow}\ x\ {\isacharless}\ iedge{\isacharunderscore}cnt\ G\ {\isasymand}\ snd\ {\isacharparenleft}iedges\ G\ x{\isacharparenright}\ {\isacharequal}\ i\ {\isasymand}\ n\ i\ {\isacharequal}\ n\ {\isacharparenleft}fst\ {\isacharparenleft}iedges\ G\ x{\isacharparenright}{\isacharparenright}\ {\isacharplus}\ {\isadigit{1}}{\isacharparenright}{\isacharparenright}{\isachardoublequoteclose}\isanewline
%
\isadelimproof
\ \ %
\endisadelimproof
%
\isatagproof
\isacommand{by}\isamarkupfalse%
\ {\isacharparenleft}auto\ simp{\isacharcolon}\ parent{\isacharunderscore}num{\isacharunderscore}assms{\isacharunderscore}inv{\isacharunderscore}def\ less{\isacharunderscore}Suc{\isacharunderscore}eq{\isacharparenright}%
\endisatagproof
{\isafoldproof}%
%
\isadelimproof
\isanewline
%
\endisadelimproof
\isanewline
\isacommand{lemma}\isamarkupfalse%
\ {\isacharparenleft}\isakeyword{in}\ parent{\isacharunderscore}num{\isacharunderscore}assms{\isacharunderscore}impl{\isacharparenright}\ parent{\isacharunderscore}num{\isacharunderscore}assms{\isacharunderscore}spec{\isacharcolon}\isanewline
\ \ {\isachardoublequoteopen}{\isasymforall}G\ r\ p\ n{\isachardot}\ {\isasymGamma}\ {\isasymturnstile}\isactrlsub t\ {\isasymlbrace}\ {\isasymacute}G\ {\isacharequal}\ G\ {\isasymand}\ {\isasymacute}r\ {\isacharequal}\ r\ {\isasymand}\ {\isasymacute}parent{\isacharunderscore}edge\ {\isacharequal}\ p\ {\isasymand}\ {\isasymacute}num\ {\isacharequal}\ n{\isasymrbrace}\isanewline
\ \ \ \ {\isasymacute}R\ {\isacharcolon}{\isacharequal}{\isacharequal}\ PROC\ parent{\isacharunderscore}num{\isacharunderscore}assms{\isacharparenleft}{\isasymacute}G{\isacharcomma}\ {\isasymacute}r{\isacharcomma}\ {\isasymacute}parent{\isacharunderscore}edge{\isacharcomma}\ {\isasymacute}num{\isacharparenright}\isanewline
\ \ \ \ {\isasymlbrace}\ {\isasymacute}R\ {\isacharequal}\ parent{\isacharunderscore}num{\isacharunderscore}assms{\isacharunderscore}inv\ G\ r\ p\ n\ {\isacharparenleft}ivertex{\isacharunderscore}cnt\ G{\isacharparenright}{\isasymrbrace}{\isachardoublequoteclose}\isanewline
%
\isadelimproof
\ \ %
\endisadelimproof
%
\isatagproof
\isacommand{apply}\isamarkupfalse%
\ vcg\isanewline
\ \ \isacommand{apply}\isamarkupfalse%
\ {\isacharparenleft}simp{\isacharunderscore}all\ add{\isacharcolon}\ parent{\isacharunderscore}num{\isacharunderscore}assms{\isacharunderscore}inv{\isacharunderscore}step{\isacharparenright}\isanewline
\ \ \isacommand{apply}\isamarkupfalse%
\ {\isacharparenleft}auto\ simp{\isacharcolon}\ parent{\isacharunderscore}num{\isacharunderscore}assms{\isacharunderscore}inv{\isacharunderscore}def{\isacharparenright}\isanewline
\ \ \isacommand{done}\isamarkupfalse%
%
\endisatagproof
{\isafoldproof}%
%
\isadelimproof
\isanewline
%
\endisadelimproof
\isanewline
\isacommand{lemma}\isamarkupfalse%
\ connected{\isacharunderscore}components{\isacharunderscore}locale{\isacharunderscore}eq{\isacharunderscore}invariants{\isacharcolon}\isanewline
{\isachardoublequoteopen}{\isasymAnd}G\ r\ p\ n{\isachardot}\ \isanewline
\ \ connected{\isacharunderscore}components{\isacharunderscore}locale\ {\isacharparenleft}abs{\isacharunderscore}IGraph\ G{\isacharparenright}\ n\ p\ r\ {\isacharequal}\ \isanewline
\ \ \ \ {\isacharparenleft}is{\isacharunderscore}wellformed{\isacharunderscore}inv\ G\ {\isacharparenleft}iedge{\isacharunderscore}cnt\ G{\isacharparenright}\ {\isasymand}\ \isanewline
\ \ \ \ r\ {\isacharless}\ ivertex{\isacharunderscore}cnt\ G\ {\isasymand}\ n\ r\ {\isacharequal}\ {\isadigit{0}}\ {\isasymand}\ p\ r\ {\isacharequal}\ None\ {\isasymand}\ \isanewline
\ \ \ \ parent{\isacharunderscore}num{\isacharunderscore}assms{\isacharunderscore}inv\ G\ r\ p\ n\ {\isacharparenleft}ivertex{\isacharunderscore}cnt\ G{\isacharparenright}{\isacharparenright}{\isachardoublequoteclose}\ \isanewline
%
\isadelimproof
%
\endisadelimproof
%
\isatagproof
\isacommand{proof}\isamarkupfalse%
\ {\isacharminus}\isanewline
\ \ \isacommand{fix}\isamarkupfalse%
\ G\ r\ p\ n\isanewline
\ \ \isacommand{let}\isamarkupfalse%
\ {\isacharquery}aG\ {\isacharequal}\ {\isachardoublequoteopen}abs{\isacharunderscore}IGraph\ G{\isachardoublequoteclose}\isanewline
\ \ \isacommand{have}\isamarkupfalse%
\ {\isachardoublequoteopen}is{\isacharunderscore}wellformed{\isacharunderscore}inv\ G\ {\isacharparenleft}iedge{\isacharunderscore}cnt\ G{\isacharparenright}\ {\isacharequal}\ fin{\isacharunderscore}digraph\ {\isacharquery}aG{\isachardoublequoteclose}\isanewline
\ \ \ \ \isacommand{unfolding}\isamarkupfalse%
\ is{\isacharunderscore}wellformed{\isacharunderscore}inv{\isacharunderscore}def\ fin{\isacharunderscore}digraph{\isacharunderscore}def\ fin{\isacharunderscore}digraph{\isacharunderscore}axioms{\isacharunderscore}def\isanewline
\ \ \ \ \ \ wf{\isacharunderscore}digraph{\isacharunderscore}def\ \isanewline
\ \ \ \ \ \ \isacommand{by}\isamarkupfalse%
\ auto\isanewline
\isacommand{moreover}\isamarkupfalse%
\isanewline
\ \ \isacommand{have}\isamarkupfalse%
\ {\isachardoublequoteopen}{\isacharparenleft}{\isasymforall}v\ {\isasymin}\ verts\ {\isacharquery}aG{\isachardot}\ v\ {\isasymnoteq}\ r\ {\isasymlongrightarrow}\isanewline
\ \ \ \ {\isacharparenleft}{\isasymexists}e\ {\isasymin}\ arcs\ {\isacharquery}aG{\isachardot}\ p\ v\ {\isacharequal}\ Some\ e\ {\isasymand}\ \isanewline
\ \ \ \ head\ {\isacharquery}aG\ e\ {\isacharequal}\ v\ {\isasymand}\ \isanewline
\ \ \ \ n\ v\ {\isacharequal}\ \ n\ {\isacharparenleft}tail\ {\isacharquery}aG\ e{\isacharparenright}\ {\isacharplus}\ {\isadigit{1}}{\isacharparenright}{\isacharparenright}\ \isanewline
\ \ \ \ {\isacharequal}\ parent{\isacharunderscore}num{\isacharunderscore}assms{\isacharunderscore}inv\ G\ r\ p\ n\ {\isacharparenleft}ivertex{\isacharunderscore}cnt\ G{\isacharparenright}{\isachardoublequoteclose}\isanewline
\ \ \isacommand{proof}\isamarkupfalse%
\ {\isacharminus}\isanewline
\ \ \ \ \isacommand{{\isacharbraceleft}}\isamarkupfalse%
\ \isacommand{fix}\isamarkupfalse%
\ i\ \isacommand{assume}\isamarkupfalse%
\ {\isachardoublequoteopen}{\isacharparenleft}case\ p\ i\ of\ None\ {\isasymRightarrow}\ False\isanewline
\ \ \ \ \ \ \ \ {\isacharbar}\ Some\ x\ {\isasymRightarrow}\ x\ {\isacharless}\ iedge{\isacharunderscore}cnt\ G\ {\isasymand}\ snd\ {\isacharparenleft}iedges\ G\ x{\isacharparenright}\ {\isacharequal}\ i\ {\isasymand}\ n\ i\ {\isacharequal}\ n\ {\isacharparenleft}fst\ {\isacharparenleft}iedges\ G\ x{\isacharparenright}{\isacharparenright}\ {\isacharplus}\ {\isadigit{1}}{\isacharparenright}{\isachardoublequoteclose}\isanewline
\ \ \ \ \ \ \isacommand{then}\isamarkupfalse%
\ \isacommand{have}\isamarkupfalse%
\ {\isachardoublequoteopen}{\isasymexists}x{\isasymin}{\isacharbraceleft}{\isadigit{0}}{\isachardot}{\isachardot}{\isacharless}iedge{\isacharunderscore}cnt\ G{\isacharbraceright}{\isachardot}\ p\ i\ {\isacharequal}\ Some\ x\ {\isasymand}\ snd\ {\isacharparenleft}iedges\ G\ x{\isacharparenright}\ {\isacharequal}\ i\ {\isasymand}\ n\ i\ {\isacharequal}\ n\ {\isacharparenleft}fst\ {\isacharparenleft}iedges\ G\ x{\isacharparenright}{\isacharparenright}\ {\isacharplus}\ {\isadigit{1}}{\isachardoublequoteclose}\isanewline
\ \ \ \ \ \ \isacommand{by}\isamarkupfalse%
\ {\isacharparenleft}case{\isacharunderscore}tac\ {\isachardoublequoteopen}p\ i{\isachardoublequoteclose}{\isacharparenright}\ auto\ \isacommand{{\isacharbraceright}}\isamarkupfalse%
\isanewline
\ \ \ \ \isacommand{then}\isamarkupfalse%
\ \isacommand{show}\isamarkupfalse%
\ {\isacharquery}thesis\isanewline
\ \ \ \ \ \ \isacommand{by}\isamarkupfalse%
\ {\isacharparenleft}auto\ simp{\isacharcolon}\ parent{\isacharunderscore}num{\isacharunderscore}assms{\isacharunderscore}inv{\isacharunderscore}def{\isacharparenright}\isanewline
\ \ \isacommand{qed}\isamarkupfalse%
\isanewline
\isacommand{ultimately}\isamarkupfalse%
\isanewline
\isacommand{show}\isamarkupfalse%
\ \ {\isachardoublequoteopen}{\isacharquery}thesis\ G\ r\ p\ n{\isachardoublequoteclose}\isanewline
\ \ \isacommand{unfolding}\isamarkupfalse%
\ connected{\isacharunderscore}components{\isacharunderscore}locale{\isacharunderscore}def\ \isanewline
\ \ connected{\isacharunderscore}components{\isacharunderscore}locale{\isacharunderscore}axioms{\isacharunderscore}def\ \isacommand{by}\isamarkupfalse%
\ auto\isanewline
\isacommand{qed}\isamarkupfalse%
%
\endisatagproof
{\isafoldproof}%
%
\isadelimproof
\isanewline
%
\endisadelimproof
\isanewline
\isacommand{theorem}\isamarkupfalse%
\ {\isacharparenleft}\isakeyword{in}\ check{\isacharunderscore}connected{\isacharunderscore}impl{\isacharparenright}\ check{\isacharunderscore}connected{\isacharunderscore}eq{\isacharunderscore}locale{\isacharcolon}\isanewline
\ \ {\isachardoublequoteopen}{\isasymforall}G\ r\ p\ n{\isachardot}\ {\isasymGamma}\ {\isasymturnstile}\isactrlsub t\ {\isasymlbrace}\ {\isasymacute}G\ {\isacharequal}\ G\ {\isasymand}\ {\isasymacute}r\ {\isacharequal}\ r\ {\isasymand}\ {\isasymacute}parent{\isacharunderscore}edge\ {\isacharequal}\ p\ {\isasymand}\ {\isasymacute}num\ {\isacharequal}\ n\ {\isasymrbrace}\isanewline
\ \ \ \ {\isasymacute}R\ {\isacharcolon}{\isacharequal}{\isacharequal}\ PROC\ check{\isacharunderscore}connected\ {\isacharparenleft}{\isasymacute}G{\isacharcomma}\ {\isasymacute}r{\isacharcomma}\ {\isasymacute}parent{\isacharunderscore}edge{\isacharcomma}\ {\isasymacute}num{\isacharparenright}\isanewline
\ \ \ \ {\isasymlbrace}\ {\isasymacute}R\ {\isacharequal}\ connected{\isacharunderscore}components{\isacharunderscore}locale\ {\isacharparenleft}abs{\isacharunderscore}IGraph\ G{\isacharparenright}\ n\ p\ r{\isasymrbrace}{\isachardoublequoteclose}\isanewline
%
\isadelimproof
%
\endisadelimproof
%
\isatagproof
\isacommand{by}\isamarkupfalse%
\ vcg\ {\isacharparenleft}auto\ simp{\isacharcolon}\ connected{\isacharunderscore}components{\isacharunderscore}locale{\isacharunderscore}eq{\isacharunderscore}invariants{\isacharparenright}%
\endisatagproof
{\isafoldproof}%
%
\isadelimproof
\isanewline
%
\endisadelimproof
\isanewline
\isacommand{lemma}\isamarkupfalse%
\ connected{\isacharunderscore}components{\isacharunderscore}locale{\isacharunderscore}imp{\isacharunderscore}correct{\isacharcolon}\isanewline
\ \ \isakeyword{assumes}\ {\isachardoublequoteopen}connected{\isacharunderscore}components{\isacharunderscore}locale\ {\isacharparenleft}abs{\isacharunderscore}IGraph\ G{\isacharparenright}n\ p\ r{\isachardoublequoteclose}\isanewline
\ \ \isakeyword{assumes}\ {\isachardoublequoteopen}u\ {\isasymin}\ pverts\ {\isacharparenleft}mk{\isacharunderscore}symmetric\ {\isacharparenleft}abs{\isacharunderscore}IGraph\ G{\isacharparenright}{\isacharparenright}{\isachardoublequoteclose}\isanewline
\ \ \isakeyword{assumes}\ {\isachardoublequoteopen}v\ {\isasymin}\ pverts\ {\isacharparenleft}mk{\isacharunderscore}symmetric\ {\isacharparenleft}abs{\isacharunderscore}IGraph\ G{\isacharparenright}{\isacharparenright}{\isachardoublequoteclose}\isanewline
\ \ \isakeyword{shows}\ {\isachardoublequoteopen}{\isasymexists}p{\isachardot}\ pre{\isacharunderscore}digraph{\isachardot}apath\ {\isacharparenleft}mk{\isacharunderscore}symmetric\ {\isacharparenleft}abs{\isacharunderscore}IGraph\ G{\isacharparenright}{\isacharparenright}\ u\ p\ v{\isachardoublequoteclose}\isanewline
%
\isadelimproof
%
\endisadelimproof
%
\isatagproof
\isacommand{proof}\isamarkupfalse%
\ {\isacharminus}\isanewline
\ \ \isacommand{interpret}\isamarkupfalse%
\ S{\isacharcolon}\ pair{\isacharunderscore}wf{\isacharunderscore}digraph\ {\isachardoublequoteopen}mk{\isacharunderscore}symmetric\ {\isacharparenleft}abs{\isacharunderscore}IGraph\ G{\isacharparenright}{\isachardoublequoteclose}\isanewline
\ \ \ \ \isacommand{by}\isamarkupfalse%
\ {\isacharparenleft}intro\ wf{\isacharunderscore}digraph{\isachardot}wellformed{\isacharunderscore}mk{\isacharunderscore}symmetric\isanewline
\ \ \ \ \ \ \ \ connected{\isacharunderscore}components{\isacharunderscore}locale{\isachardot}ccl{\isacharunderscore}wellformed{\isacharbrackleft}OF\ assms{\isacharparenleft}{\isadigit{1}}{\isacharparenright}{\isacharbrackright}{\isacharparenright}\isanewline
\ \ \isacommand{show}\isamarkupfalse%
\ {\isacharquery}thesis\isanewline
\ \ \ \ \isacommand{using}\isamarkupfalse%
\ connected{\isacharunderscore}components{\isacharunderscore}locale{\isachardot}connected{\isacharunderscore}by{\isacharunderscore}path{\isacharbrackleft}OF\ assms{\isacharbrackright}\isanewline
\ \ \ \ \isacommand{by}\isamarkupfalse%
\ {\isacharparenleft}simp\ only{\isacharcolon}\ S{\isachardot}reachable{\isacharunderscore}apath{\isacharparenright}\isanewline
\isacommand{qed}\isamarkupfalse%
%
\endisatagproof
{\isafoldproof}%
%
\isadelimproof
\isanewline
%
\endisadelimproof
\isanewline
\isacommand{theorem}\isamarkupfalse%
\ {\isacharparenleft}\isakeyword{in}\ check{\isacharunderscore}connected{\isacharunderscore}impl{\isacharparenright}\ check{\isacharunderscore}connected{\isacharunderscore}spec{\isacharcolon}\isanewline
\ \ {\isachardoublequoteopen}{\isasymAnd}G\ r\ p\ n{\isachardot}\ {\isasymGamma}\ {\isasymturnstile}\isactrlsub t\ {\isasymlbrace}\ {\isasymacute}G\ {\isacharequal}\ G\ {\isasymand}\ {\isasymacute}r\ {\isacharequal}\ r\ {\isasymand}\ {\isasymacute}parent{\isacharunderscore}edge\ {\isacharequal}\ p\ {\isasymand}\ {\isasymacute}num\ {\isacharequal}\ n\ {\isasymrbrace}\isanewline
\ \ \ \ {\isasymacute}R\ {\isacharcolon}{\isacharequal}{\isacharequal}\ PROC\ check{\isacharunderscore}connected{\isacharparenleft}{\isasymacute}G{\isacharcomma}\ {\isasymacute}r{\isacharcomma}\ {\isasymacute}parent{\isacharunderscore}edge{\isacharcomma}\ {\isasymacute}num{\isacharparenright}\isanewline
\ \ \ \ {\isasymlbrace}\ {\isasymacute}R\ {\isasymlongrightarrow}\isanewline
\ \ \ \ \ \ \ \ {\isacharparenleft}{\isasymforall}u\ {\isasymin}\ pverts\ {\isacharparenleft}mk{\isacharunderscore}symmetric\ {\isacharparenleft}abs{\isacharunderscore}IGraph\ G{\isacharparenright}{\isacharparenright}{\isachardot}\isanewline
\ \ \ \ \ \ \ \ \ \ {\isasymforall}v\ {\isasymin}\ pverts\ {\isacharparenleft}mk{\isacharunderscore}symmetric\ {\isacharparenleft}abs{\isacharunderscore}IGraph\ G{\isacharparenright}{\isacharparenright}{\isachardot}\ \isanewline
\ \ \ \ \ \ \ \ \ \ {\isasymexists}p{\isachardot}\ pre{\isacharunderscore}digraph{\isachardot}apath\ {\isacharparenleft}mk{\isacharunderscore}symmetric\ {\isacharparenleft}abs{\isacharunderscore}IGraph\ G{\isacharparenright}{\isacharparenright}\ u\ p\ v{\isacharparenright}{\isasymrbrace}{\isachardoublequoteclose}\isanewline
%
\isadelimproof
%
\endisadelimproof
%
\isatagproof
\isacommand{using}\isamarkupfalse%
\ connected{\isacharunderscore}components{\isacharunderscore}locale{\isacharunderscore}eq{\isacharunderscore}invariants\isanewline
\ \ \ \ \ \ connected{\isacharunderscore}components{\isacharunderscore}locale{\isacharunderscore}imp{\isacharunderscore}correct\ \isanewline
\isacommand{by}\isamarkupfalse%
\ vcg\ metis%
\endisatagproof
{\isafoldproof}%
%
\isadelimproof
\isanewline
%
\endisadelimproof
%
\isadelimtheory
\isanewline
%
\endisadelimtheory
%
\isatagtheory
\isacommand{end}\isamarkupfalse%
%
\endisatagtheory
{\isafoldtheory}%
%
\isadelimtheory
%
\endisadelimtheory
\end{isabellebody}%
%%% Local Variables:
%%% mode: latex
%%% TeX-master: "root"
%%% End:


%
\begin{isabellebody}%
\def\isabellecontext{Check{\isacharunderscore}Shortest{\isacharunderscore}Path{\isacharunderscore}Impl}%
%
\isadelimtheory
%
\endisadelimtheory
%
\isatagtheory
\isacommand{theory}\isamarkupfalse%
\ Check{\isacharunderscore}Shortest{\isacharunderscore}Path{\isacharunderscore}Impl\isanewline
\isakeyword{imports}\isanewline
\ \ {\isachardoublequoteopen}Vcg{\isachardoublequoteclose}\isanewline
\ \ {\isachardoublequoteopen}{\isachardot}{\isachardot}{\isacharslash}Witness{\isacharunderscore}Property{\isacharslash}Shortest{\isacharunderscore}Path{\isacharunderscore}Theory{\isachardoublequoteclose}\isanewline
{\isachardoublequoteopen}{\isachartilde}{\isachartilde}{\isacharslash}src{\isacharslash}HOL{\isacharslash}Statespace{\isacharslash}StateSpaceLocale{\isachardoublequoteclose}\isanewline
\isakeyword{begin}%
\endisatagtheory
{\isafoldtheory}%
%
\isadelimtheory
\isanewline
%
\endisadelimtheory
\isanewline
\isacommand{type{\isacharunderscore}synonym}\isamarkupfalse%
\ IVertex\ {\isacharequal}\ nat\isanewline
\isacommand{type{\isacharunderscore}synonym}\isamarkupfalse%
\ IEdge{\isacharunderscore}Id\ {\isacharequal}\ nat\isanewline
\isacommand{type{\isacharunderscore}synonym}\isamarkupfalse%
\ IEdge\ {\isacharequal}\ {\isachardoublequoteopen}IVertex\ {\isasymtimes}\ IVertex{\isachardoublequoteclose}\isanewline
\isacommand{type{\isacharunderscore}synonym}\isamarkupfalse%
\ ICost\ {\isacharequal}\ {\isachardoublequoteopen}IEdge{\isacharunderscore}Id\ {\isasymRightarrow}\ nat{\isachardoublequoteclose}\isanewline
\isacommand{type{\isacharunderscore}synonym}\isamarkupfalse%
\ IDist\ {\isacharequal}\ {\isachardoublequoteopen}IVertex\ {\isasymRightarrow}\ ereal{\isachardoublequoteclose}\isanewline
\isacommand{type{\isacharunderscore}synonym}\isamarkupfalse%
\ IPEdge\ {\isacharequal}\ {\isachardoublequoteopen}IVertex\ {\isasymRightarrow}\ IEdge{\isacharunderscore}Id\ option{\isachardoublequoteclose}\isanewline
\isacommand{type{\isacharunderscore}synonym}\isamarkupfalse%
\ INum\ {\isacharequal}\ {\isachardoublequoteopen}IVertex\ {\isasymRightarrow}\ enat{\isachardoublequoteclose}\isanewline
\isacommand{type{\isacharunderscore}synonym}\isamarkupfalse%
\ IGraph\ {\isacharequal}\ {\isachardoublequoteopen}nat\ {\isasymtimes}\ nat\ {\isasymtimes}\ {\isacharparenleft}IEdge{\isacharunderscore}Id\ {\isasymRightarrow}\ IEdge{\isacharparenright}{\isachardoublequoteclose}\ \isanewline
\isanewline
\isacommand{abbreviation}\isamarkupfalse%
\ ivertex{\isacharunderscore}cnt\ {\isacharcolon}{\isacharcolon}\ {\isachardoublequoteopen}IGraph\ {\isasymRightarrow}\ nat{\isachardoublequoteclose}\isanewline
\ \ \isakeyword{where}\ {\isachardoublequoteopen}ivertex{\isacharunderscore}cnt\ G\ {\isasymequiv}\ fst\ G{\isachardoublequoteclose}\isanewline
\isanewline
\isacommand{abbreviation}\isamarkupfalse%
\ iedge{\isacharunderscore}cnt\ {\isacharcolon}{\isacharcolon}\ {\isachardoublequoteopen}IGraph\ {\isasymRightarrow}\ nat{\isachardoublequoteclose}\isanewline
\ \ \isakeyword{where}\ {\isachardoublequoteopen}iedge{\isacharunderscore}cnt\ G\ {\isasymequiv}\ fst\ {\isacharparenleft}snd\ G{\isacharparenright}{\isachardoublequoteclose}\isanewline
\isanewline
\isacommand{abbreviation}\isamarkupfalse%
\ iarcs\ {\isacharcolon}{\isacharcolon}\ {\isachardoublequoteopen}IGraph\ {\isasymRightarrow}\ IEdge{\isacharunderscore}Id\ {\isasymRightarrow}\ IEdge{\isachardoublequoteclose}\isanewline
\ \ \isakeyword{where}\ {\isachardoublequoteopen}iarcs\ G\ {\isasymequiv}\ snd\ {\isacharparenleft}snd\ G{\isacharparenright}{\isachardoublequoteclose}\isanewline
\isanewline
\isacommand{definition}\isamarkupfalse%
\ is{\isacharunderscore}wellformed{\isacharunderscore}inv\ {\isacharcolon}{\isacharcolon}\ {\isachardoublequoteopen}IGraph\ {\isasymRightarrow}\ nat\ {\isasymRightarrow}\ bool{\isachardoublequoteclose}\ \isakeyword{where}\isanewline
\ \ {\isachardoublequoteopen}is{\isacharunderscore}wellformed{\isacharunderscore}inv\ G\ i\ {\isasymequiv}\ {\isasymforall}k\ {\isacharless}\ i{\isachardot}\ ivertex{\isacharunderscore}cnt\ G\ {\isachargreater}\ fst\ {\isacharparenleft}iarcs\ G\ k{\isacharparenright}\isanewline
\ \ \ \ \ \ \ \ {\isasymand}\ ivertex{\isacharunderscore}cnt\ G\ {\isachargreater}\ snd\ {\isacharparenleft}iarcs\ G\ k{\isacharparenright}{\isachardoublequoteclose}\isanewline
\isanewline
\isacommand{procedures}\isamarkupfalse%
\ is{\isacharunderscore}wellformed\ {\isacharparenleft}G\ {\isacharcolon}{\isacharcolon}\ IGraph\ {\isacharbar}\ R\ {\isacharcolon}{\isacharcolon}\ bool{\isacharparenright}\isanewline
\ \ \isakeyword{where}\isanewline
\ \ \ \ i\ {\isacharcolon}{\isacharcolon}\ nat\isanewline
\ \ \ \ e\ {\isacharcolon}{\isacharcolon}\ IEdge\isanewline
\ \ \isakeyword{in}\ {\isachardoublequoteopen}\isanewline
\ \ \ \ ANNO\ G{\isachardot}\isanewline
\ \ \ \ \ \ {\isasymlbrace}\ {\isasymacute}G\ {\isacharequal}\ G\ {\isasymrbrace}\isanewline
\ \ \ \ \ \ {\isasymacute}R\ {\isacharcolon}{\isacharequal}{\isacharequal}\ True\ {\isacharsemicolon}{\isacharsemicolon}\isanewline
\ \ \ \ \ \ {\isasymacute}i\ {\isacharcolon}{\isacharequal}{\isacharequal}\ {\isadigit{0}}\ {\isacharsemicolon}{\isacharsemicolon}\isanewline
\ \ \ \ \ \ TRY\isanewline
\ \ \ \ \ \ \ \ WHILE\ {\isasymacute}i\ {\isacharless}\ iedge{\isacharunderscore}cnt\ {\isasymacute}G\isanewline
\ \ \ \ \ \ \ \ INV\ {\isasymlbrace}\ {\isasymacute}R\ {\isacharequal}\ is{\isacharunderscore}wellformed{\isacharunderscore}inv\ {\isasymacute}G\ {\isasymacute}i\ {\isasymand}\ {\isasymacute}i\ {\isasymle}\ iedge{\isacharunderscore}cnt\ {\isasymacute}G\ {\isasymand}\ {\isasymacute}G\ {\isacharequal}\ G\ {\isasymrbrace}\isanewline
\ \ \ \ \ \ \ \ VAR\ MEASURE\ {\isacharparenleft}iedge{\isacharunderscore}cnt\ {\isasymacute}G\ {\isacharminus}\ {\isasymacute}i{\isacharparenright}\isanewline
\ \ \ \ \ \ \ \ DO\isanewline
\ \ \ \ \ \ \ \ \ {\isasymacute}e\ {\isacharcolon}{\isacharequal}{\isacharequal}\ iarcs\ {\isasymacute}G\ {\isasymacute}i\ {\isacharsemicolon}{\isacharsemicolon}\isanewline
\ \ \ \ \ \ \ \ \ IF\ ivertex{\isacharunderscore}cnt\ {\isasymacute}G\ {\isasymle}\ fst\ {\isasymacute}e\ {\isasymor}\ ivertex{\isacharunderscore}cnt\ {\isasymacute}G\ {\isasymle}\ snd\ {\isasymacute}e\ THEN\isanewline
\ \ \ \ \ \ \ \ \ \ \ {\isasymacute}R\ {\isacharcolon}{\isacharequal}{\isacharequal}\ False\ {\isacharsemicolon}{\isacharsemicolon}\isanewline
\ \ \ \ \ \ \ \ \ \ \ THROW\isanewline
\ \ \ \ \ \ \ \ \ FI\ {\isacharsemicolon}{\isacharsemicolon}\isanewline
\ \ \ \ \ \ \ \ \ {\isasymacute}i\ {\isacharcolon}{\isacharequal}{\isacharequal}\ {\isasymacute}i\ {\isacharplus}\ {\isadigit{1}}\isanewline
\ \ \ \ \ \ \ \ OD\isanewline
\ \ \ \ \ \ CATCH\ SKIP\ END\isanewline
\ \ \ \ \ \ {\isasymlbrace}\ {\isasymacute}G\ {\isacharequal}\ G\ {\isasymand}\ {\isasymacute}R\ {\isacharequal}\ is{\isacharunderscore}wellformed{\isacharunderscore}inv\ {\isasymacute}G\ {\isacharparenleft}iedge{\isacharunderscore}cnt\ {\isasymacute}G{\isacharparenright}\ {\isasymrbrace}\isanewline
\ \ \ \ {\isachardoublequoteclose}\isanewline
\isanewline
\isacommand{definition}\isamarkupfalse%
\ trian{\isacharunderscore}inv\ {\isacharcolon}{\isacharcolon}\ {\isachardoublequoteopen}IGraph\ {\isasymRightarrow}\ IDist\ {\isasymRightarrow}\ ICost\ {\isasymRightarrow}\ nat\ {\isasymRightarrow}\ bool{\isachardoublequoteclose}\ \isakeyword{where}\isanewline
\ \ {\isachardoublequoteopen}trian{\isacharunderscore}inv\ G\ d\ c\ m\ {\isasymequiv}\ \isanewline
\ \ \ \ {\isasymforall}i\ {\isacharless}\ m{\isachardot}\ d\ {\isacharparenleft}snd\ {\isacharparenleft}iarcs\ G\ i{\isacharparenright}{\isacharparenright}\ {\isasymle}\ d\ {\isacharparenleft}fst\ {\isacharparenleft}iarcs\ G\ i{\isacharparenright}{\isacharparenright}\ {\isacharplus}\ ereal\ {\isacharparenleft}c\ i{\isacharparenright}{\isachardoublequoteclose}\isanewline
\isanewline
\isacommand{procedures}\isamarkupfalse%
\ trian\ {\isacharparenleft}G\ {\isacharcolon}{\isacharcolon}\ IGraph{\isacharcomma}\ dist\ {\isacharcolon}{\isacharcolon}\ IDist{\isacharcomma}\ c\ {\isacharcolon}{\isacharcolon}\ ICost\ {\isacharbar}\ R\ {\isacharcolon}{\isacharcolon}\ bool{\isacharparenright}\isanewline
\ \ \isakeyword{where}\isanewline
\ \ \ \ edge{\isacharunderscore}id\ {\isacharcolon}{\isacharcolon}\ IEdge{\isacharunderscore}Id\isanewline
\ \ \isakeyword{in}\ {\isachardoublequoteopen}\isanewline
\ \ \ \ ANNO\ {\isacharparenleft}G{\isacharcomma}dist{\isacharcomma}c{\isacharparenright}{\isachardot}\isanewline
\ \ \ \ \ \ {\isasymlbrace}\ {\isasymacute}G\ {\isacharequal}\ G\ {\isasymand}\ {\isasymacute}dist\ {\isacharequal}\ dist\ {\isasymand}\ {\isasymacute}c\ {\isacharequal}\ c\ {\isasymrbrace}\isanewline
\ \ \ \ \ \ {\isasymacute}R\ {\isacharcolon}{\isacharequal}{\isacharequal}\ True\ {\isacharsemicolon}{\isacharsemicolon}\isanewline
\ \ \ \ \ \ {\isasymacute}edge{\isacharunderscore}id\ {\isacharcolon}{\isacharequal}{\isacharequal}\ {\isadigit{0}}\ {\isacharsemicolon}{\isacharsemicolon}\isanewline
\ \ \ \ \ \ TRY\isanewline
\ \ \ \ \ \ \ \ WHILE\ {\isasymacute}edge{\isacharunderscore}id\ {\isacharless}\ iedge{\isacharunderscore}cnt\ {\isasymacute}G\isanewline
\ \ \ \ \ \ \ \ INV\ {\isasymlbrace}\ {\isasymacute}R\ {\isacharequal}\ trian{\isacharunderscore}inv\ {\isasymacute}G\ {\isasymacute}dist\ {\isasymacute}c\ {\isasymacute}edge{\isacharunderscore}id\isanewline
\ \ \ \ \ \ \ \ \ \ {\isasymand}\ {\isasymacute}G\ {\isacharequal}\ G\ {\isasymand}\ {\isasymacute}dist\ {\isacharequal}\ dist\ {\isasymand}\ {\isasymacute}c\ {\isacharequal}\ c\isanewline
\ \ \ \ \ \ \ \ \ \ {\isasymand}\ {\isasymacute}edge{\isacharunderscore}id\ {\isasymle}\ iedge{\isacharunderscore}cnt\ {\isasymacute}G{\isasymrbrace}\isanewline
\ \ \ \ \ \ \ \ VAR\ MEASURE\ {\isacharparenleft}iedge{\isacharunderscore}cnt\ {\isasymacute}G\ {\isacharminus}\ {\isasymacute}edge{\isacharunderscore}id{\isacharparenright}\isanewline
\ \ \ \ \ \ \ \ DO\isanewline
\ \ \ \ \ \ \ \ \ \ IF\ \ {\isasymacute}dist\ {\isacharparenleft}snd\ {\isacharparenleft}iarcs\ {\isasymacute}G\ {\isasymacute}edge{\isacharunderscore}id{\isacharparenright}{\isacharparenright}\ {\isachargreater}\ \isanewline
\ \ \ \ \ \ \ \ \ \ \ \ \ \ {\isasymacute}dist\ {\isacharparenleft}fst\ {\isacharparenleft}iarcs\ {\isasymacute}G\ {\isasymacute}edge{\isacharunderscore}id{\isacharparenright}{\isacharparenright}\ {\isacharplus}\ \isanewline
\ \ \ \ \ \ \ \ \ \ \ \ \ \ ereal\ {\isacharparenleft}{\isasymacute}c\ {\isasymacute}edge{\isacharunderscore}id{\isacharparenright}\ THEN\isanewline
\ \ \ \ \ \ \ \ \ \ \ \ {\isasymacute}R\ {\isacharcolon}{\isacharequal}{\isacharequal}\ False\ {\isacharsemicolon}{\isacharsemicolon}\isanewline
\ \ \ \ \ \ \ \ \ \ \ \ THROW\isanewline
\ \ \ \ \ \ \ \ \ \ FI\ {\isacharsemicolon}{\isacharsemicolon}\isanewline
\ \ \ \ \ \ \ \ \ \ {\isasymacute}edge{\isacharunderscore}id\ {\isacharcolon}{\isacharequal}{\isacharequal}\ {\isasymacute}edge{\isacharunderscore}id\ {\isacharplus}\ {\isadigit{1}}\isanewline
\ \ \ \ \ \ \ \ OD\isanewline
\ \ \ \ \ \ CATCH\ SKIP\ END\isanewline
\ \ \ \ \ \ {\isasymlbrace}\ {\isasymacute}G\ {\isacharequal}\ G\ {\isasymand}\ {\isasymacute}dist\ {\isacharequal}\ dist\ {\isasymand}\ {\isasymacute}c\ {\isacharequal}\ c\isanewline
\ \ \ \ \ \ {\isasymand}\ {\isasymacute}R\ {\isacharequal}\ trian{\isacharunderscore}inv\ {\isasymacute}G\ {\isasymacute}dist\ {\isasymacute}c\ {\isacharparenleft}iedge{\isacharunderscore}cnt\ {\isasymacute}G{\isacharparenright}\ {\isasymrbrace}\isanewline
\ \ \ \ {\isachardoublequoteclose}\isanewline
\isanewline
\isacommand{definition}\isamarkupfalse%
\ just{\isacharunderscore}inv\ {\isacharcolon}{\isacharcolon}\ \isanewline
\ \ {\isachardoublequoteopen}IGraph\ {\isasymRightarrow}\ IDist\ {\isasymRightarrow}\ ICost\ {\isasymRightarrow}\ IVertex\ {\isasymRightarrow}\ INum\ {\isasymRightarrow}\ IPEdge\ {\isasymRightarrow}\ nat\ {\isasymRightarrow}\ bool{\isachardoublequoteclose}\ \isakeyword{where}\isanewline
\ \ {\isachardoublequoteopen}just{\isacharunderscore}inv\ G\ d\ c\ s\ n\ p\ k\ {\isasymequiv}\ \isanewline
\ \ \ \ {\isasymforall}v\ {\isacharless}\ k{\isachardot}\ v\ {\isasymnoteq}\ s\ {\isasymand}\ n\ v\ {\isasymnoteq}\ {\isasyminfinity}\ {\isasymlongrightarrow}\ \isanewline
\ \ \ \ \ \ {\isacharparenleft}{\isasymexists}\ e{\isachardot}\ e\ {\isacharequal}\ the\ {\isacharparenleft}p\ v{\isacharparenright}\ {\isasymand}\ e\ {\isacharless}\ iedge{\isacharunderscore}cnt\ G\ {\isasymand}\ \isanewline
\ \ \ \ \ \ \ \ v\ {\isacharequal}\ snd\ {\isacharparenleft}iarcs\ G\ e{\isacharparenright}\ {\isasymand}\isanewline
\ \ \ \ \ \ \ \ d\ v\ {\isacharequal}\ d\ {\isacharparenleft}fst\ {\isacharparenleft}iarcs\ G\ e{\isacharparenright}{\isacharparenright}\ {\isacharplus}\ ereal\ {\isacharparenleft}c\ e{\isacharparenright}\ {\isasymand}\isanewline
\ \ \ \ \ \ \ \ n\ v\ {\isacharequal}\ n\ {\isacharparenleft}fst\ {\isacharparenleft}iarcs\ G\ e{\isacharparenright}{\isacharparenright}\ {\isacharplus}\ {\isacharparenleft}enat\ {\isadigit{1}}{\isacharparenright}{\isacharparenright}{\isachardoublequoteclose}\isanewline
\isanewline
\isacommand{procedures}\isamarkupfalse%
\ just\ {\isacharparenleft}G\ {\isacharcolon}{\isacharcolon}\ IGraph{\isacharcomma}\ dist\ {\isacharcolon}{\isacharcolon}\ IDist{\isacharcomma}\ c\ {\isacharcolon}{\isacharcolon}\ ICost{\isacharcomma}\ \isanewline
\ \ \ \ s\ {\isacharcolon}{\isacharcolon}\ IVertex{\isacharcomma}\ enum\ {\isacharcolon}{\isacharcolon}\ INum{\isacharcomma}\ pred\ {\isacharcolon}{\isacharcolon}\ IPEdge\ {\isacharbar}\ R\ {\isacharcolon}{\isacharcolon}\ bool{\isacharparenright}\isanewline
\ \ \isakeyword{where}\isanewline
\ \ \ \ v\ {\isacharcolon}{\isacharcolon}\ IVertex\isanewline
\ \ \ \ edge{\isacharunderscore}id\ {\isacharcolon}{\isacharcolon}\ IEdge{\isacharunderscore}Id\isanewline
\ \ \isakeyword{in}\ {\isachardoublequoteopen}\isanewline
\ \ \ \ ANNO\ {\isacharparenleft}G{\isacharcomma}dist{\isacharcomma}\ c{\isacharcomma}\ s\ {\isacharcomma}enum{\isacharcomma}\ pred{\isacharparenright}{\isachardot}\isanewline
\ \ \ \ \ \ {\isasymlbrace}\ {\isasymacute}G\ {\isacharequal}\ G\ {\isasymand}\ {\isasymacute}dist\ {\isacharequal}\ dist\ {\isasymand}\ {\isasymacute}c\ {\isacharequal}\ c\ {\isasymand}\ {\isasymacute}s\ {\isacharequal}\ s\ {\isasymand}\ {\isasymacute}enum\ {\isacharequal}\ enum\ {\isasymand}\ {\isasymacute}pred\ {\isacharequal}\ pred{\isasymrbrace}\isanewline
\ \ \ \ \ \ {\isasymacute}R\ {\isacharcolon}{\isacharequal}{\isacharequal}\ True\ {\isacharsemicolon}{\isacharsemicolon}\isanewline
\ \ \ \ \ \ {\isasymacute}v\ {\isacharcolon}{\isacharequal}{\isacharequal}\ {\isadigit{0}}\ {\isacharsemicolon}{\isacharsemicolon}\isanewline
\ \ \ \ \ \ TRY\isanewline
\ \ \ \ \ \ \ \ WHILE\ {\isasymacute}v\ {\isacharless}\ ivertex{\isacharunderscore}cnt\ {\isasymacute}G\isanewline
\ \ \ \ \ \ \ \ INV\ {\isasymlbrace}\ {\isasymacute}R\ {\isacharequal}\ just{\isacharunderscore}inv\ {\isasymacute}G\ {\isasymacute}dist\ {\isasymacute}c\ {\isasymacute}s\ {\isasymacute}enum\ {\isasymacute}pred\ {\isasymacute}v\isanewline
\ \ \ \ \ \ \ \ \ \ {\isasymand}\ {\isasymacute}G\ {\isacharequal}\ G\ {\isasymand}\ {\isasymacute}c\ {\isacharequal}\ c\ {\isasymand}\ {\isasymacute}s\ {\isacharequal}\ s\ {\isasymand}\ {\isasymacute}dist\ {\isacharequal}\ dist\ \isanewline
\ \ \ \ \ \ \ \ \ \ {\isasymand}\ {\isasymacute}enum\ {\isacharequal}\ enum\ {\isasymand}\ {\isasymacute}pred\ {\isacharequal}\ pred\isanewline
\ \ \ \ \ \ \ \ \ \ {\isasymand}\ {\isasymacute}v\ {\isasymle}\ ivertex{\isacharunderscore}cnt\ {\isasymacute}G{\isasymrbrace}\isanewline
\ \ \ \ \ \ \ \ VAR\ MEASURE\ {\isacharparenleft}ivertex{\isacharunderscore}cnt\ {\isasymacute}G\ {\isacharminus}\ {\isasymacute}v{\isacharparenright}\isanewline
\ \ \ \ \ \ \ \ DO\isanewline
\ \ \ \ \ \ \ \ \ \ {\isasymacute}edge{\isacharunderscore}id\ {\isacharcolon}{\isacharequal}{\isacharequal}\ the\ {\isacharparenleft}{\isasymacute}pred\ {\isasymacute}v{\isacharparenright}\ {\isacharsemicolon}{\isacharsemicolon}\isanewline
\ \ \ \ \ \ \ \ \ \ IF\ {\isacharparenleft}{\isasymacute}v\ {\isasymnoteq}\ {\isasymacute}s{\isacharparenright}\ {\isasymand}\ \ {\isasymacute}enum\ {\isasymacute}v\ {\isasymnoteq}\ {\isasyminfinity}\ {\isasymand}\isanewline
\ \ \ \ \ \ \ \ \ \ \ \ \ {\isacharparenleft}{\isasymacute}edge{\isacharunderscore}id\ {\isasymge}\ iedge{\isacharunderscore}cnt\ {\isasymacute}G\ \isanewline
\ \ \ \ \ \ \ \ \ \ \ \ \ \ {\isasymor}\ snd\ {\isacharparenleft}iarcs\ {\isasymacute}G\ {\isasymacute}edge{\isacharunderscore}id{\isacharparenright}\ {\isasymnoteq}\ {\isasymacute}v\isanewline
\ \ \ \ \ \ \ \ \ \ \ \ \ \ {\isasymor}\ {\isasymacute}dist\ {\isasymacute}v\ {\isasymnoteq}\ \isanewline
\ \ \ \ \ \ \ \ \ \ \ \ \ \ \ \ {\isasymacute}dist\ {\isacharparenleft}fst\ {\isacharparenleft}iarcs\ {\isasymacute}G\ {\isasymacute}edge{\isacharunderscore}id{\isacharparenright}{\isacharparenright}\ {\isacharplus}\ ereal\ {\isacharparenleft}{\isasymacute}c\ {\isasymacute}edge{\isacharunderscore}id{\isacharparenright}\isanewline
\ \ \ \ \ \ \ \ \ \ \ \ \ \ {\isasymor}\ {\isasymacute}enum\ {\isasymacute}v\ {\isasymnoteq}\ {\isasymacute}enum\ {\isacharparenleft}fst\ {\isacharparenleft}iarcs\ {\isasymacute}G\ {\isasymacute}edge{\isacharunderscore}id{\isacharparenright}{\isacharparenright}\ {\isacharplus}\ {\isacharparenleft}enat\ {\isadigit{1}}{\isacharparenright}{\isacharparenright}\ THEN\isanewline
\ \ \ \ \ \ \ \ \ \ \ \ {\isasymacute}R\ {\isacharcolon}{\isacharequal}{\isacharequal}\ False\ {\isacharsemicolon}{\isacharsemicolon}\isanewline
\ \ \ \ \ \ \ \ \ \ \ \ THROW\isanewline
\ \ \ \ \ \ \ \ \ \ FI{\isacharsemicolon}{\isacharsemicolon}\isanewline
\ \ \ \ \ \ \ \ \ \ {\isasymacute}v\ {\isacharcolon}{\isacharequal}{\isacharequal}\ {\isasymacute}v\ {\isacharplus}\ {\isadigit{1}}\isanewline
\ \ \ \ \ \ \ \ OD\isanewline
\ \ \ \ \ \ CATCH\ SKIP\ END\isanewline
\ \ \ \ {\isasymlbrace}\ {\isasymacute}G\ {\isacharequal}\ G\ {\isasymand}\ {\isasymacute}dist\ {\isacharequal}\ dist\ {\isasymand}\ {\isasymacute}c\ {\isacharequal}\ c\ {\isasymand}\ {\isasymacute}s\ {\isacharequal}\ s\ {\isasymand}\ {\isasymacute}enum\ {\isacharequal}\ enum\ {\isasymand}\ {\isasymacute}pred\ {\isacharequal}\ pred\isanewline
\ \ \ \ \ \ {\isasymand}\ {\isasymacute}R\ {\isacharequal}\ just{\isacharunderscore}inv\ {\isasymacute}G\ {\isasymacute}dist\ {\isasymacute}c\ {\isasymacute}s\ {\isasymacute}enum\ {\isasymacute}pred\ {\isacharparenleft}ivertex{\isacharunderscore}cnt\ {\isasymacute}G{\isacharparenright}\ {\isasymrbrace}\isanewline
\ \ {\isachardoublequoteclose}\isanewline
\isanewline
\isacommand{definition}\isamarkupfalse%
\ no{\isacharunderscore}path{\isacharunderscore}inv\ {\isacharcolon}{\isacharcolon}\ {\isachardoublequoteopen}IGraph\ {\isasymRightarrow}\ IDist\ {\isasymRightarrow}\ INum\ {\isasymRightarrow}\ nat\ {\isasymRightarrow}\ bool{\isachardoublequoteclose}\ \isakeyword{where}\isanewline
\ \ {\isachardoublequoteopen}no{\isacharunderscore}path{\isacharunderscore}inv\ G\ d\ n\ k\ {\isasymequiv}\ \ {\isasymforall}v\ {\isacharless}\ k{\isachardot}\ {\isacharparenleft}d\ v\ {\isacharequal}\ {\isasyminfinity}\ {\isasymlongleftrightarrow}\ n\ v\ {\isacharequal}\ {\isasyminfinity}{\isacharparenright}{\isachardoublequoteclose}\isanewline
\isanewline
\isacommand{procedures}\isamarkupfalse%
\ no{\isacharunderscore}path\ {\isacharparenleft}G\ {\isacharcolon}{\isacharcolon}\ IGraph{\isacharcomma}\ dist\ {\isacharcolon}{\isacharcolon}\ IDist{\isacharcomma}\ enum\ {\isacharcolon}{\isacharcolon}\ INum\ {\isacharbar}\ R\ {\isacharcolon}{\isacharcolon}\ bool{\isacharparenright}\isanewline
\ \ \isakeyword{where}\isanewline
\ \ \ \ v\ {\isacharcolon}{\isacharcolon}\ IVertex\isanewline
\ \ \isakeyword{in}\ {\isachardoublequoteopen}\isanewline
\ \ \ \ ANNO\ {\isacharparenleft}G{\isacharcomma}dist{\isacharcomma}enum{\isacharparenright}{\isachardot}\isanewline
\ \ \ \ \ \ {\isasymlbrace}\ {\isasymacute}G\ {\isacharequal}\ G\ {\isasymand}\ {\isasymacute}dist\ {\isacharequal}\ dist\ {\isasymand}\ {\isasymacute}enum\ {\isacharequal}\ enum\ {\isasymrbrace}\isanewline
\ \ \ \ \ \ {\isasymacute}R\ {\isacharcolon}{\isacharequal}{\isacharequal}\ True\ {\isacharsemicolon}{\isacharsemicolon}\isanewline
\ \ \ \ \ \ {\isasymacute}v\ {\isacharcolon}{\isacharequal}{\isacharequal}\ {\isadigit{0}}\ {\isacharsemicolon}{\isacharsemicolon}\isanewline
\ \ \ \ \ \ TRY\isanewline
\ \ \ \ \ \ \ \ WHILE\ {\isasymacute}v\ {\isacharless}\ ivertex{\isacharunderscore}cnt\ {\isasymacute}G\isanewline
\ \ \ \ \ \ \ \ INV\ {\isasymlbrace}\ {\isasymacute}R\ {\isacharequal}\ no{\isacharunderscore}path{\isacharunderscore}inv\ {\isasymacute}G\ {\isasymacute}dist\ {\isasymacute}enum\ {\isasymacute}v\isanewline
\ \ \ \ \ \ \ \ \ \ {\isasymand}\ {\isasymacute}G\ {\isacharequal}\ G\ {\isasymand}\ {\isasymacute}dist\ {\isacharequal}\ dist\ {\isasymand}\ {\isasymacute}enum\ {\isacharequal}\ enum\isanewline
\ \ \ \ \ \ \ \ \ \ {\isasymand}\ {\isasymacute}v\ {\isasymle}\ ivertex{\isacharunderscore}cnt\ {\isasymacute}G{\isasymrbrace}\isanewline
\ \ \ \ \ \ \ \ VAR\ MEASURE\ {\isacharparenleft}ivertex{\isacharunderscore}cnt\ {\isasymacute}G\ {\isacharminus}\ {\isasymacute}v{\isacharparenright}\isanewline
\ \ \ \ \ \ \ \ DO\isanewline
\ \ \ \ \ \ \ \ \ \ IF\ \ {\isasymnot}{\isacharparenleft}{\isasymacute}dist\ {\isasymacute}v\ {\isacharequal}\ {\isasyminfinity}\ {\isasymlongleftrightarrow}\ {\isasymacute}enum\ {\isasymacute}v\ {\isacharequal}\ {\isasyminfinity}{\isacharparenright}\ THEN\isanewline
\ \ \ \ \ \ \ \ \ \ \ \ {\isasymacute}R\ {\isacharcolon}{\isacharequal}{\isacharequal}\ False\ {\isacharsemicolon}{\isacharsemicolon}\isanewline
\ \ \ \ \ \ \ \ \ \ \ \ THROW\isanewline
\ \ \ \ \ \ \ \ \ \ FI\ {\isacharsemicolon}{\isacharsemicolon}\isanewline
\ \ \ \ \ \ \ \ \ \ {\isasymacute}v\ {\isacharcolon}{\isacharequal}{\isacharequal}\ {\isasymacute}v\ {\isacharplus}\ {\isadigit{1}}\isanewline
\ \ \ \ \ \ \ \ OD\isanewline
\ \ \ \ \ \ CATCH\ SKIP\ END\isanewline
\ \ \ \ \ \ {\isasymlbrace}\ {\isasymacute}G\ {\isacharequal}\ G\ {\isasymand}\ {\isasymacute}dist\ {\isacharequal}\ dist\ {\isasymand}\ {\isasymacute}enum\ {\isacharequal}\ enum\isanewline
\ \ \ \ \ \ {\isasymand}\ {\isasymacute}R\ {\isacharequal}\ no{\isacharunderscore}path{\isacharunderscore}inv\ {\isasymacute}G\ {\isasymacute}dist\ {\isasymacute}enum\ {\isacharparenleft}ivertex{\isacharunderscore}cnt\ {\isasymacute}G{\isacharparenright}\ {\isasymrbrace}\isanewline
\ \ \ \ {\isachardoublequoteclose}\isanewline
\isanewline
\isacommand{definition}\isamarkupfalse%
\ non{\isacharunderscore}neg{\isacharunderscore}cost{\isacharunderscore}inv\ {\isacharcolon}{\isacharcolon}\ {\isachardoublequoteopen}IGraph\ {\isasymRightarrow}\ ICost\ {\isasymRightarrow}\ nat\ {\isasymRightarrow}\ bool{\isachardoublequoteclose}\ \isakeyword{where}\isanewline
\ \ {\isachardoublequoteopen}non{\isacharunderscore}neg{\isacharunderscore}cost{\isacharunderscore}inv\ G\ c\ m\ {\isasymequiv}\ \ {\isasymforall}e\ {\isacharless}\ m{\isachardot}\ c\ e\ {\isasymge}\ {\isadigit{0}}{\isachardoublequoteclose}\isanewline
\isanewline
\isacommand{procedures}\isamarkupfalse%
\ non{\isacharunderscore}neg{\isacharunderscore}cost\ {\isacharparenleft}G\ {\isacharcolon}{\isacharcolon}\ IGraph{\isacharcomma}\ c\ {\isacharcolon}{\isacharcolon}\ ICost\ {\isacharbar}\ R\ {\isacharcolon}{\isacharcolon}\ bool{\isacharparenright}\isanewline
\ \ \isakeyword{where}\isanewline
\ \ \ \ edge{\isacharunderscore}id\ {\isacharcolon}{\isacharcolon}\ IEdge{\isacharunderscore}Id\isanewline
\ \ \isakeyword{in}\ {\isachardoublequoteopen}\isanewline
\ \ \ \ ANNO\ {\isacharparenleft}G{\isacharcomma}c{\isacharparenright}{\isachardot}\isanewline
\ \ \ \ \ \ {\isasymlbrace}\ {\isasymacute}G\ {\isacharequal}\ G\ {\isasymand}\ {\isasymacute}c\ {\isacharequal}\ c\ {\isasymrbrace}\isanewline
\ \ \ \ \ \ {\isasymacute}R\ {\isacharcolon}{\isacharequal}{\isacharequal}\ True\ {\isacharsemicolon}{\isacharsemicolon}\isanewline
\ \ \ \ \ \ {\isasymacute}edge{\isacharunderscore}id\ {\isacharcolon}{\isacharequal}{\isacharequal}\ {\isadigit{0}}\ {\isacharsemicolon}{\isacharsemicolon}\isanewline
\ \ \ \ \ \ TRY\isanewline
\ \ \ \ \ \ \ \ WHILE\ {\isasymacute}edge{\isacharunderscore}id\ {\isacharless}\ iedge{\isacharunderscore}cnt\ {\isasymacute}G\isanewline
\ \ \ \ \ \ \ \ INV\ {\isasymlbrace}\ {\isasymacute}R\ {\isacharequal}\ non{\isacharunderscore}neg{\isacharunderscore}cost{\isacharunderscore}inv\ {\isasymacute}G\ {\isasymacute}c\ {\isasymacute}edge{\isacharunderscore}id\isanewline
\ \ \ \ \ \ \ \ \ \ {\isasymand}\ {\isasymacute}G\ {\isacharequal}\ G\ {\isasymand}\ {\isasymacute}c\ {\isacharequal}\ c\isanewline
\ \ \ \ \ \ \ \ \ \ {\isasymand}\ {\isasymacute}edge{\isacharunderscore}id\ {\isasymle}\ iedge{\isacharunderscore}cnt\ {\isasymacute}G{\isasymrbrace}\isanewline
\ \ \ \ \ \ \ \ VAR\ MEASURE\ {\isacharparenleft}iedge{\isacharunderscore}cnt\ {\isasymacute}G\ {\isacharminus}\ {\isasymacute}edge{\isacharunderscore}id{\isacharparenright}\isanewline
\ \ \ \ \ \ \ \ DO\isanewline
\ \ \ \ \ \ \ \ \ \ IF\ {\isasymacute}c\ {\isasymacute}edge{\isacharunderscore}id\ {\isacharless}\ {\isadigit{0}}\ THEN\isanewline
\ \ \ \ \ \ \ \ \ \ \ \ {\isasymacute}R\ {\isacharcolon}{\isacharequal}{\isacharequal}\ False\ {\isacharsemicolon}{\isacharsemicolon}\isanewline
\ \ \ \ \ \ \ \ \ \ \ \ THROW\isanewline
\ \ \ \ \ \ \ \ \ \ FI\ {\isacharsemicolon}{\isacharsemicolon}\isanewline
\ \ \ \ \ \ \ \ \ \ {\isasymacute}edge{\isacharunderscore}id\ {\isacharcolon}{\isacharequal}{\isacharequal}\ {\isasymacute}edge{\isacharunderscore}id\ {\isacharplus}\ {\isadigit{1}}\isanewline
\ \ \ \ \ \ \ \ OD\isanewline
\ \ \ \ \ \ CATCH\ SKIP\ END\isanewline
\ \ \ \ \ \ {\isasymlbrace}\ {\isasymacute}G\ {\isacharequal}\ G\ {\isasymand}\ {\isasymacute}c\ {\isacharequal}\ c\isanewline
\ \ \ \ \ \ {\isasymand}\ {\isasymacute}R\ {\isacharequal}\ non{\isacharunderscore}neg{\isacharunderscore}cost{\isacharunderscore}inv\ {\isasymacute}G\ {\isasymacute}c\ {\isacharparenleft}iedge{\isacharunderscore}cnt\ {\isasymacute}G{\isacharparenright}\ {\isasymrbrace}\isanewline
\ \ \ \ {\isachardoublequoteclose}\isanewline
\isanewline
\isacommand{procedures}\isamarkupfalse%
\ check{\isacharunderscore}basic{\isacharunderscore}just{\isacharunderscore}sp\ {\isacharparenleft}G\ {\isacharcolon}{\isacharcolon}\ IGraph{\isacharcomma}\ dist\ {\isacharcolon}{\isacharcolon}\ IDist{\isacharcomma}\ c\ {\isacharcolon}{\isacharcolon}\ ICost{\isacharcomma}\ \isanewline
\ \ \ \ s\ {\isacharcolon}{\isacharcolon}\ IVertex{\isacharcomma}\ enum\ {\isacharcolon}{\isacharcolon}\ INum{\isacharcomma}\ pred\ {\isacharcolon}{\isacharcolon}\ IPEdge\ {\isacharbar}\ R\ {\isacharcolon}{\isacharcolon}\ bool{\isacharparenright}\isanewline
\ \ \isakeyword{where}\isanewline
\ \ \ \ R{\isadigit{1}}\ {\isacharcolon}{\isacharcolon}\ bool\isanewline
\ \ \ \ R{\isadigit{2}}\ {\isacharcolon}{\isacharcolon}\ bool\isanewline
\ \ \ \ R{\isadigit{3}}\ {\isacharcolon}{\isacharcolon}\ bool\isanewline
\ \ \ \ R{\isadigit{4}}\ {\isacharcolon}{\isacharcolon}\ bool\isanewline
\ \ \isakeyword{in}\ {\isachardoublequoteopen}\isanewline
\ \ \ \ {\isasymacute}R{\isadigit{1}}\ {\isacharcolon}{\isacharequal}{\isacharequal}\ CALL\ is{\isacharunderscore}wellformed\ {\isacharparenleft}{\isasymacute}G{\isacharparenright}\ {\isacharsemicolon}{\isacharsemicolon}\isanewline
\ \ \ \ {\isasymacute}R{\isadigit{2}}\ {\isacharcolon}{\isacharequal}{\isacharequal}\ {\isasymacute}dist\ {\isasymacute}s\ {\isasymle}\ {\isadigit{0}}\ {\isacharsemicolon}{\isacharsemicolon}\isanewline
\ \ \ \ {\isasymacute}R{\isadigit{3}}\ {\isacharcolon}{\isacharequal}{\isacharequal}\ CALL\ trian\ {\isacharparenleft}{\isasymacute}G{\isacharcomma}\ {\isasymacute}dist{\isacharcomma}\ {\isasymacute}c{\isacharparenright}\ {\isacharsemicolon}{\isacharsemicolon}\isanewline
\ \ \ \ {\isasymacute}R{\isadigit{4}}\ {\isacharcolon}{\isacharequal}{\isacharequal}\ CALL\ just\ {\isacharparenleft}{\isasymacute}G{\isacharcomma}\ {\isasymacute}dist{\isacharcomma}\ {\isasymacute}c{\isacharcomma}\ {\isasymacute}s{\isacharcomma}\ {\isasymacute}enum{\isacharcomma}\ {\isasymacute}pred{\isacharparenright}\ {\isacharsemicolon}{\isacharsemicolon}\isanewline
\ \ \ \ {\isasymacute}R\ {\isacharcolon}{\isacharequal}{\isacharequal}\ {\isasymacute}R{\isadigit{1}}\ {\isasymand}\ {\isasymacute}R{\isadigit{2}}\ {\isasymand}\ {\isasymacute}R{\isadigit{3}}\ {\isasymand}\ {\isasymacute}R{\isadigit{4}}\isanewline
\ \ {\isachardoublequoteclose}\isanewline
\isanewline
\isacommand{procedures}\isamarkupfalse%
\ check{\isacharunderscore}sp\ {\isacharparenleft}G\ {\isacharcolon}{\isacharcolon}\ IGraph{\isacharcomma}\ dist\ {\isacharcolon}{\isacharcolon}\ IDist{\isacharcomma}\ c\ {\isacharcolon}{\isacharcolon}\ ICost{\isacharcomma}\ \isanewline
\ \ \ \ s\ {\isacharcolon}{\isacharcolon}\ IVertex{\isacharcomma}\ enum\ {\isacharcolon}{\isacharcolon}\ INum{\isacharcomma}\ pred\ {\isacharcolon}{\isacharcolon}\ IPEdge\ {\isacharbar}\ R\ {\isacharcolon}{\isacharcolon}\ bool{\isacharparenright}\isanewline
\ \ \isakeyword{where}\isanewline
\ \ \ \ R{\isadigit{1}}\ {\isacharcolon}{\isacharcolon}\ bool\isanewline
\ \ \ \ R{\isadigit{2}}\ {\isacharcolon}{\isacharcolon}\ bool\isanewline
\ \ \ \ R{\isadigit{3}}\ {\isacharcolon}{\isacharcolon}\ bool\isanewline
\ \ \ \ R{\isadigit{4}}\ {\isacharcolon}{\isacharcolon}\ bool\isanewline
\ \ \isakeyword{in}\ {\isachardoublequoteopen}\isanewline
\ \ \ \ {\isasymacute}R{\isadigit{1}}\ {\isacharcolon}{\isacharequal}{\isacharequal}\ CALL\ check{\isacharunderscore}basic{\isacharunderscore}just{\isacharunderscore}sp\ {\isacharparenleft}{\isasymacute}G{\isacharcomma}\ {\isasymacute}dist{\isacharcomma}\ {\isasymacute}c{\isacharcomma}\ {\isasymacute}s{\isacharcomma}\ {\isasymacute}enum{\isacharcomma}\ {\isasymacute}pred{\isacharparenright}\ {\isacharsemicolon}{\isacharsemicolon}\isanewline
\ \ \ \ {\isasymacute}R{\isadigit{2}}\ {\isacharcolon}{\isacharequal}{\isacharequal}\ {\isasymacute}s\ {\isacharless}\ ivertex{\isacharunderscore}cnt\ {\isasymacute}G\ {\isasymand}\ {\isasymacute}dist\ {\isasymacute}s\ {\isacharequal}\ {\isadigit{0}}\ {\isacharsemicolon}{\isacharsemicolon}\isanewline
\ \ \ \ {\isasymacute}R{\isadigit{3}}\ {\isacharcolon}{\isacharequal}{\isacharequal}\ CALL\ no{\isacharunderscore}path\ {\isacharparenleft}{\isasymacute}G{\isacharcomma}\ {\isasymacute}dist{\isacharcomma}\ {\isasymacute}enum{\isacharparenright}\ {\isacharsemicolon}{\isacharsemicolon}\isanewline
\ \ \ \ {\isasymacute}R{\isadigit{4}}\ {\isacharcolon}{\isacharequal}{\isacharequal}\ CALL\ non{\isacharunderscore}neg{\isacharunderscore}cost\ {\isacharparenleft}{\isasymacute}G{\isacharcomma}\ {\isasymacute}c{\isacharparenright}\ {\isacharsemicolon}{\isacharsemicolon}\isanewline
\ \ \ \ {\isasymacute}R\ {\isacharcolon}{\isacharequal}{\isacharequal}\ {\isasymacute}R{\isadigit{1}}\ {\isasymand}\ {\isasymacute}R{\isadigit{2}}\ {\isasymand}\ {\isasymacute}R{\isadigit{3}}\ {\isasymand}\ {\isasymacute}R{\isadigit{4}}\isanewline
\ \ {\isachardoublequoteclose}\isanewline
%
\isadelimtheory
\isanewline
%
\endisadelimtheory
%
\isatagtheory
\isacommand{end}\isamarkupfalse%
%
\endisatagtheory
{\isafoldtheory}%
%
\isadelimtheory
%
\endisadelimtheory
\end{isabellebody}%
%%% Local Variables:
%%% mode: latex
%%% TeX-master: "root"
%%% End:


%
\begin{isabellebody}%
\def\isabellecontext{Check{\isacharunderscore}Shortest{\isacharunderscore}Path{\isacharunderscore}Verification}%
%
\isadelimtheory
%
\endisadelimtheory
%
\isatagtheory
\isacommand{theory}\isamarkupfalse%
\ Check{\isacharunderscore}Shortest{\isacharunderscore}Path{\isacharunderscore}Verification\isanewline
\isakeyword{imports}\isanewline
\ \ {\isachardoublequoteopen}Vcg{\isachardoublequoteclose}\isanewline
\ \ {\isachardoublequoteopen}{\isachardot}{\isachardot}{\isacharslash}Simpl{\isacharunderscore}Verification{\isacharslash}Check{\isacharunderscore}Shortest{\isacharunderscore}Path{\isacharunderscore}Impl{\isachardoublequoteclose}\isanewline
\isanewline
\isakeyword{begin}%
\endisatagtheory
{\isafoldtheory}%
%
\isadelimtheory
\isanewline
%
\endisadelimtheory
\isanewline
\isacommand{definition}\isamarkupfalse%
\ no{\isacharunderscore}loops\ {\isacharcolon}{\isacharcolon}\ {\isachardoublequoteopen}{\isacharparenleft}{\isacharprime}a{\isacharcomma}\ {\isacharprime}b{\isacharparenright}\ pre{\isacharunderscore}digraph\ {\isasymRightarrow}\ bool{\isachardoublequoteclose}\ \isakeyword{where}\isanewline
\ \ {\isachardoublequoteopen}no{\isacharunderscore}loops\ G\ {\isasymequiv}\ {\isasymforall}e\ {\isasymin}\ arcs\ G{\isachardot}\ tail\ G\ e\ {\isasymnoteq}\ head\ G\ e{\isachardoublequoteclose}\isanewline
\isanewline
\isacommand{definition}\isamarkupfalse%
\ abs{\isacharunderscore}IGraph\ {\isacharcolon}{\isacharcolon}\ {\isachardoublequoteopen}IGraph\ {\isasymRightarrow}\ {\isacharparenleft}nat{\isacharcomma}\ nat{\isacharparenright}\ pre{\isacharunderscore}digraph{\isachardoublequoteclose}\ \isakeyword{where}\isanewline
\ \ {\isachardoublequoteopen}abs{\isacharunderscore}IGraph\ G\ {\isasymequiv}\ {\isasymlparr}\ verts\ {\isacharequal}\ {\isacharbraceleft}{\isadigit{0}}{\isachardot}{\isachardot}{\isacharless}ivertex{\isacharunderscore}cnt\ G{\isacharbraceright}{\isacharcomma}\ arcs\ {\isacharequal}\ {\isacharbraceleft}{\isadigit{0}}{\isachardot}{\isachardot}{\isacharless}iedge{\isacharunderscore}cnt\ G{\isacharbraceright}{\isacharcomma}\isanewline
\ \ \ \ tail\ {\isacharequal}\ fst\ o\ iarcs\ G{\isacharcomma}\ head\ {\isacharequal}\ snd\ o\ iarcs\ G\ {\isasymrparr}{\isachardoublequoteclose}\isanewline
\isanewline
\isacommand{lemma}\isamarkupfalse%
\ verts{\isacharunderscore}absI{\isacharbrackleft}simp{\isacharbrackright}{\isacharcolon}\ {\isachardoublequoteopen}verts\ {\isacharparenleft}abs{\isacharunderscore}IGraph\ G{\isacharparenright}\ {\isacharequal}\ {\isacharbraceleft}{\isadigit{0}}{\isachardot}{\isachardot}{\isacharless}ivertex{\isacharunderscore}cnt\ G{\isacharbraceright}{\isachardoublequoteclose}\isanewline
\ \ \isakeyword{and}\ arcs{\isacharunderscore}absI{\isacharbrackleft}simp{\isacharbrackright}{\isacharcolon}\ {\isachardoublequoteopen}arcs\ {\isacharparenleft}abs{\isacharunderscore}IGraph\ G{\isacharparenright}\ {\isacharequal}\ {\isacharbraceleft}{\isadigit{0}}{\isachardot}{\isachardot}{\isacharless}iedge{\isacharunderscore}cnt\ G{\isacharbraceright}{\isachardoublequoteclose}\isanewline
\ \ \isakeyword{and}\ tail{\isacharunderscore}absI{\isacharbrackleft}simp{\isacharbrackright}{\isacharcolon}\ {\isachardoublequoteopen}tail\ {\isacharparenleft}abs{\isacharunderscore}IGraph\ G{\isacharparenright}\ e\ {\isacharequal}\ fst\ {\isacharparenleft}iarcs\ G\ e{\isacharparenright}{\isachardoublequoteclose}\isanewline
\ \ \isakeyword{and}\ head{\isacharunderscore}absI{\isacharbrackleft}simp{\isacharbrackright}{\isacharcolon}\ {\isachardoublequoteopen}head\ {\isacharparenleft}abs{\isacharunderscore}IGraph\ G{\isacharparenright}\ e\ {\isacharequal}\ snd\ {\isacharparenleft}iarcs\ G\ e{\isacharparenright}{\isachardoublequoteclose}\isanewline
%
\isadelimproof
\ \ %
\endisadelimproof
%
\isatagproof
\isacommand{by}\isamarkupfalse%
\ {\isacharparenleft}auto\ simp{\isacharcolon}\ abs{\isacharunderscore}IGraph{\isacharunderscore}def{\isacharparenright}%
\endisatagproof
{\isafoldproof}%
%
\isadelimproof
\isanewline
%
\endisadelimproof
\isanewline
\isacommand{lemma}\isamarkupfalse%
\ is{\isacharunderscore}wellformed{\isacharunderscore}inv{\isacharunderscore}step{\isacharcolon}\isanewline
\ \ {\isachardoublequoteopen}is{\isacharunderscore}wellformed{\isacharunderscore}inv\ G\ {\isacharparenleft}Suc\ i{\isacharparenright}\ {\isasymlongleftrightarrow}\ is{\isacharunderscore}wellformed{\isacharunderscore}inv\ G\ i\isanewline
\ \ \ \ \ \ {\isasymand}\ fst\ {\isacharparenleft}iarcs\ G\ i{\isacharparenright}\ {\isacharless}\ ivertex{\isacharunderscore}cnt\ G\ {\isasymand}\ snd\ {\isacharparenleft}iarcs\ G\ i{\isacharparenright}\ {\isacharless}\ ivertex{\isacharunderscore}cnt\ G{\isachardoublequoteclose}\isanewline
%
\isadelimproof
\ \ %
\endisadelimproof
%
\isatagproof
\isacommand{by}\isamarkupfalse%
\ {\isacharparenleft}auto\ simp\ add{\isacharcolon}\ is{\isacharunderscore}wellformed{\isacharunderscore}inv{\isacharunderscore}def\ less{\isacharunderscore}Suc{\isacharunderscore}eq{\isacharparenright}%
\endisatagproof
{\isafoldproof}%
%
\isadelimproof
\isanewline
%
\endisadelimproof
\isanewline
\isacommand{lemma}\isamarkupfalse%
\ {\isacharparenleft}\isakeyword{in}\ is{\isacharunderscore}wellformed{\isacharunderscore}impl{\isacharparenright}\ is{\isacharunderscore}wellformed{\isacharunderscore}spec{\isacharcolon}\isanewline
\ \ {\isachardoublequoteopen}{\isasymforall}G{\isachardot}\ {\isasymGamma}\ {\isasymturnstile}\isactrlsub t\ {\isasymlbrace}{\isasymacute}G\ {\isacharequal}\ G{\isasymrbrace}\ {\isasymacute}R\ {\isacharcolon}{\isacharequal}{\isacharequal}\ PROC\ is{\isacharunderscore}wellformed{\isacharparenleft}{\isasymacute}G{\isacharparenright}\ {\isasymlbrace}{\isasymacute}R\ {\isacharequal}\ is{\isacharunderscore}wellformed{\isacharunderscore}inv\ G\ {\isacharparenleft}iedge{\isacharunderscore}cnt\ G{\isacharparenright}{\isasymrbrace}{\isachardoublequoteclose}\isanewline
%
\isadelimproof
\ \ %
\endisadelimproof
%
\isatagproof
\isacommand{apply}\isamarkupfalse%
\ vcg\isanewline
\ \ \isacommand{apply}\isamarkupfalse%
\ {\isacharparenleft}auto\ simp{\isacharcolon}\ is{\isacharunderscore}wellformed{\isacharunderscore}inv{\isacharunderscore}step{\isacharparenright}\isanewline
\ \ \isacommand{apply}\isamarkupfalse%
\ {\isacharparenleft}auto\ simp{\isacharcolon}\ is{\isacharunderscore}wellformed{\isacharunderscore}inv{\isacharunderscore}def{\isacharparenright}\ \isanewline
\isacommand{done}\isamarkupfalse%
%
\endisatagproof
{\isafoldproof}%
%
\isadelimproof
\isanewline
%
\endisadelimproof
\isanewline
\isacommand{lemma}\isamarkupfalse%
\ trian{\isacharunderscore}inv{\isacharunderscore}step{\isacharcolon}\isanewline
\ \ {\isachardoublequoteopen}trian{\isacharunderscore}inv\ G\ d\ c\ {\isacharparenleft}Suc\ i{\isacharparenright}\ {\isasymlongleftrightarrow}\ trian{\isacharunderscore}inv\ G\ d\ c\ i\isanewline
\ \ \ \ {\isasymand}\ d\ {\isacharparenleft}snd\ {\isacharparenleft}iarcs\ G\ i{\isacharparenright}{\isacharparenright}\ {\isasymle}\ d\ {\isacharparenleft}fst\ {\isacharparenleft}iarcs\ G\ i{\isacharparenright}{\isacharparenright}\ {\isacharplus}\ c\ i{\isachardoublequoteclose}\isanewline
%
\isadelimproof
\ \ %
\endisadelimproof
%
\isatagproof
\isacommand{by}\isamarkupfalse%
\ {\isacharparenleft}auto\ simp{\isacharcolon}\ trian{\isacharunderscore}inv{\isacharunderscore}def\ less{\isacharunderscore}Suc{\isacharunderscore}eq{\isacharparenright}%
\endisatagproof
{\isafoldproof}%
%
\isadelimproof
\isanewline
%
\endisadelimproof
\isanewline
\isacommand{lemma}\isamarkupfalse%
\ {\isacharparenleft}\isakeyword{in}\ trian{\isacharunderscore}impl{\isacharparenright}\ trian{\isacharunderscore}spec{\isacharcolon}\isanewline
\ \ {\isachardoublequoteopen}{\isasymforall}G\ d\ c{\isachardot}\ {\isasymGamma}\ {\isasymturnstile}\isactrlsub t\ {\isasymlbrace}\ {\isasymacute}G\ {\isacharequal}\ G\ {\isasymand}\ {\isasymacute}dist\ {\isacharequal}\ d\ {\isasymand}\ {\isasymacute}c\ {\isacharequal}\ c{\isasymrbrace}\isanewline
\ \ \ \ {\isasymacute}R\ {\isacharcolon}{\isacharequal}{\isacharequal}\ PROC\ trian{\isacharparenleft}{\isasymacute}G{\isacharcomma}\ {\isasymacute}dist{\isacharcomma}\ {\isasymacute}c{\isacharparenright}\isanewline
\ \ \ \ {\isasymlbrace}\ {\isasymacute}R\ {\isacharequal}\ trian{\isacharunderscore}inv\ G\ d\ c\ {\isacharparenleft}iedge{\isacharunderscore}cnt\ G{\isacharparenright}{\isasymrbrace}{\isachardoublequoteclose}\isanewline
%
\isadelimproof
\ \ %
\endisadelimproof
%
\isatagproof
\isacommand{apply}\isamarkupfalse%
\ vcg\ \ \ \isanewline
\ \ \isacommand{apply}\isamarkupfalse%
\ {\isacharparenleft}auto\ simp\ add{\isacharcolon}\ trian{\isacharunderscore}inv{\isacharunderscore}step{\isacharparenright}\isanewline
\ \ \isacommand{apply}\isamarkupfalse%
\ {\isacharparenleft}auto\ simp{\isacharcolon}\ trian{\isacharunderscore}inv{\isacharunderscore}def{\isacharparenright}\ \isanewline
\isacommand{done}\isamarkupfalse%
%
\endisatagproof
{\isafoldproof}%
%
\isadelimproof
\isanewline
%
\endisadelimproof
\isanewline
\isacommand{lemma}\isamarkupfalse%
\ just{\isacharunderscore}inv{\isacharunderscore}step{\isacharcolon}\isanewline
\ \ {\isachardoublequoteopen}just{\isacharunderscore}inv\ G\ d\ c\ s\ n\ p\ {\isacharparenleft}Suc\ v{\isacharparenright}\ {\isasymlongleftrightarrow}\ just{\isacharunderscore}inv\ G\ d\ c\ s\ n\ p\ v\isanewline
\ \ \ \ {\isasymand}\ {\isacharparenleft}v\ {\isasymnoteq}\ s\ {\isasymand}\ n\ v\ {\isasymnoteq}\ {\isasyminfinity}\ {\isasymlongrightarrow}\ \isanewline
\ \ \ \ \ \ {\isacharparenleft}{\isasymexists}\ e{\isachardot}\ e\ {\isacharequal}\ the\ {\isacharparenleft}p\ v{\isacharparenright}\ {\isasymand}\ e\ {\isacharless}\ iedge{\isacharunderscore}cnt\ G\ {\isasymand}\ \isanewline
\ \ \ \ \ \ \ \ v\ {\isacharequal}\ snd\ {\isacharparenleft}iarcs\ G\ e{\isacharparenright}\ {\isasymand}\isanewline
\ \ \ \ \ \ \ \ d\ v\ {\isacharequal}\ d\ {\isacharparenleft}fst\ {\isacharparenleft}iarcs\ G\ e{\isacharparenright}{\isacharparenright}\ {\isacharplus}\ ereal\ {\isacharparenleft}c\ e{\isacharparenright}\ {\isasymand}\isanewline
\ \ \ \ \ \ \ \ n\ v\ {\isacharequal}\ n\ {\isacharparenleft}fst\ {\isacharparenleft}iarcs\ G\ e{\isacharparenright}{\isacharparenright}\ {\isacharplus}\ {\isacharparenleft}enat\ {\isadigit{1}}{\isacharparenright}{\isacharparenright}{\isacharparenright}{\isachardoublequoteclose}\isanewline
%
\isadelimproof
\ \ %
\endisadelimproof
%
\isatagproof
\isacommand{by}\isamarkupfalse%
\ {\isacharparenleft}auto\ simp{\isacharcolon}\ just{\isacharunderscore}inv{\isacharunderscore}def\ less{\isacharunderscore}Suc{\isacharunderscore}eq{\isacharparenright}%
\endisatagproof
{\isafoldproof}%
%
\isadelimproof
\isanewline
%
\endisadelimproof
\isanewline
\isacommand{lemma}\isamarkupfalse%
\ just{\isacharunderscore}inv{\isacharunderscore}le{\isacharcolon}\isanewline
\ \ \isakeyword{assumes}\ {\isachardoublequoteopen}j\ {\isasymle}\ i{\isachardoublequoteclose}\ {\isachardoublequoteopen}just{\isacharunderscore}inv\ G\ d\ c\ s\ n\ p\ i{\isachardoublequoteclose}\isanewline
\ \ \isakeyword{shows}\ {\isachardoublequoteopen}just{\isacharunderscore}inv\ G\ d\ c\ s\ n\ p\ j{\isachardoublequoteclose}\isanewline
%
\isadelimproof
\ \ %
\endisadelimproof
%
\isatagproof
\isacommand{using}\isamarkupfalse%
\ assms\ \isacommand{by}\isamarkupfalse%
\ {\isacharparenleft}induct\ rule{\isacharcolon}\ dec{\isacharunderscore}induct{\isacharparenright}\ {\isacharparenleft}auto\ simp{\isacharcolon}\ just{\isacharunderscore}inv{\isacharunderscore}step{\isacharparenright}%
\endisatagproof
{\isafoldproof}%
%
\isadelimproof
\isanewline
%
\endisadelimproof
\isanewline
\isacommand{lemma}\isamarkupfalse%
\ not{\isacharunderscore}just{\isacharunderscore}verts{\isacharcolon}\isanewline
\ \ \isakeyword{fixes}\ G\ R\ c\ d\ n\ p\ s\ v\isanewline
\ \ \isakeyword{assumes}\ {\isachardoublequoteopen}v\ {\isacharless}\ ivertex{\isacharunderscore}cnt\ G{\isachardoublequoteclose}\isanewline
\ \ \isakeyword{assumes}\ {\isachardoublequoteopen}v\ {\isasymnoteq}\ s\ {\isasymand}\ n\ v\ {\isasymnoteq}\ {\isasyminfinity}\ {\isasymand}\isanewline
\ \ \ \ \ \ \ \ {\isacharparenleft}iedge{\isacharunderscore}cnt\ G\ {\isasymle}\ the\ {\isacharparenleft}p\ v{\isacharparenright}\ {\isasymor}\isanewline
\ \ \ \ \ \ \ \ snd\ {\isacharparenleft}iarcs\ G\ {\isacharparenleft}the\ {\isacharparenleft}p\ v{\isacharparenright}{\isacharparenright}{\isacharparenright}\ {\isasymnoteq}\ v\ {\isasymor}\ \isanewline
\ \ \ \ \ \ \ \ d\ v\ {\isasymnoteq}\ \isanewline
\ \ \ \ \ \ \ \ \ \ d\ {\isacharparenleft}fst\ {\isacharparenleft}iarcs\ G\ {\isacharparenleft}the\ {\isacharparenleft}p\ v{\isacharparenright}{\isacharparenright}{\isacharparenright}{\isacharparenright}\ {\isacharplus}\ ereal\ {\isacharparenleft}c\ {\isacharparenleft}the\ {\isacharparenleft}p\ v{\isacharparenright}{\isacharparenright}{\isacharparenright}\ {\isasymor}\ \isanewline
\ \ \ \ \ \ \ \ n\ v\ {\isasymnoteq}\ n\ {\isacharparenleft}fst\ {\isacharparenleft}iarcs\ G\ {\isacharparenleft}the\ {\isacharparenleft}p\ v{\isacharparenright}{\isacharparenright}{\isacharparenright}{\isacharparenright}\ {\isacharplus}\ enat\ {\isadigit{1}}{\isacharparenright}{\isachardoublequoteclose}\isanewline
\ \ \isakeyword{shows}\ {\isachardoublequoteopen}{\isasymnot}\ just{\isacharunderscore}inv\ G\ d\ c\ s\ n\ p\ {\isacharparenleft}ivertex{\isacharunderscore}cnt\ G{\isacharparenright}{\isachardoublequoteclose}\isanewline
%
\isadelimproof
%
\endisadelimproof
%
\isatagproof
\isacommand{proof}\isamarkupfalse%
\ {\isacharparenleft}rule\ notI{\isacharparenright}\isanewline
\ \ \isacommand{assume}\isamarkupfalse%
\ jv{\isacharcolon}\ {\isachardoublequoteopen}just{\isacharunderscore}inv\ G\ d\ c\ s\ n\ p\ {\isacharparenleft}ivertex{\isacharunderscore}cnt\ G{\isacharparenright}{\isachardoublequoteclose}\isanewline
\ \ \isacommand{have}\isamarkupfalse%
\ {\isachardoublequoteopen}just{\isacharunderscore}inv\ G\ d\ c\ s\ n\ p\ {\isacharparenleft}Suc\ v{\isacharparenright}{\isachardoublequoteclose}\isanewline
\ \ \ \ \isacommand{using}\isamarkupfalse%
\ just{\isacharunderscore}inv{\isacharunderscore}le{\isacharbrackleft}OF\ {\isacharunderscore}\ jv{\isacharbrackright}\ assms{\isacharparenleft}{\isadigit{1}}{\isacharparenright}\ \isacommand{by}\isamarkupfalse%
\ simp\isanewline
\ \ \isacommand{then}\isamarkupfalse%
\ \isacommand{have}\isamarkupfalse%
\ {\isachardoublequoteopen}{\isacharparenleft}v\ {\isasymnoteq}\ s\ {\isasymand}\ n\ v\ {\isasymnoteq}\ {\isasyminfinity}\ {\isasymlongrightarrow}\ \isanewline
\ \ \ \ \ \ {\isacharparenleft}{\isasymexists}\ e{\isachardot}\ e\ {\isacharequal}\ the\ {\isacharparenleft}p\ v{\isacharparenright}\ {\isasymand}\ e\ {\isacharless}\ iedge{\isacharunderscore}cnt\ G\ {\isasymand}\ \isanewline
\ \ \ \ \ \ \ \ v\ {\isacharequal}\ snd\ {\isacharparenleft}iarcs\ G\ e{\isacharparenright}\ {\isasymand}\isanewline
\ \ \ \ \ \ \ \ d\ v\ {\isacharequal}\ d\ {\isacharparenleft}fst\ {\isacharparenleft}iarcs\ G\ e{\isacharparenright}{\isacharparenright}\ {\isacharplus}\ ereal\ {\isacharparenleft}c\ e{\isacharparenright}\ {\isasymand}\isanewline
\ \ \ \ \ \ \ \ n\ v\ {\isacharequal}\ n\ {\isacharparenleft}fst\ {\isacharparenleft}iarcs\ G\ e{\isacharparenright}{\isacharparenright}\ {\isacharplus}\ {\isacharparenleft}enat\ {\isadigit{1}}{\isacharparenright}{\isacharparenright}{\isacharparenright}{\isachardoublequoteclose}\isanewline
\ \ \ \ \ \ \ \ \isacommand{by}\isamarkupfalse%
\ {\isacharparenleft}auto\ simp{\isacharcolon}\ just{\isacharunderscore}inv{\isacharunderscore}step{\isacharparenright}\isanewline
\ \ \isacommand{with}\isamarkupfalse%
\ assms\ \isacommand{show}\isamarkupfalse%
\ False\ \ \isacommand{by}\isamarkupfalse%
\ force\isanewline
\isacommand{qed}\isamarkupfalse%
%
\endisatagproof
{\isafoldproof}%
%
\isadelimproof
\isanewline
%
\endisadelimproof
\isanewline
\isacommand{lemma}\isamarkupfalse%
\ {\isacharparenleft}\isakeyword{in}\ just{\isacharunderscore}impl{\isacharparenright}\ just{\isacharunderscore}spec{\isacharcolon}\isanewline
\ \ {\isachardoublequoteopen}{\isasymforall}G\ d\ c\ s\ n\ p{\isachardot}\ \isanewline
\ \ \ \ {\isasymGamma}\ \ {\isasymturnstile}\isactrlsub t\ {\isasymlbrace}{\isasymacute}G\ {\isacharequal}\ G\ {\isasymand}\ {\isasymacute}dist\ {\isacharequal}\ d\ {\isasymand}\ \isanewline
\ \ \ \ {\isasymacute}c\ {\isacharequal}\ c\ {\isasymand}\ {\isasymacute}s\ {\isacharequal}\ s\ {\isasymand}\ {\isasymacute}enum\ {\isacharequal}\ n\ {\isasymand}\ {\isasymacute}pred\ {\isacharequal}\ p{\isasymrbrace}\isanewline
\ \ \ \ {\isasymacute}R\ {\isacharcolon}{\isacharequal}{\isacharequal}\ PROC\ just{\isacharparenleft}{\isasymacute}G{\isacharcomma}\ {\isasymacute}dist{\isacharcomma}\ {\isasymacute}c{\isacharcomma}\ {\isasymacute}s{\isacharcomma}\ {\isasymacute}enum{\isacharcomma}\ {\isasymacute}pred{\isacharparenright}\isanewline
\ \ \ \ {\isasymlbrace}\ {\isasymacute}R\ {\isacharequal}\ just{\isacharunderscore}inv\ \ G\ d\ c\ s\ n\ p\ {\isacharparenleft}ivertex{\isacharunderscore}cnt\ G{\isacharparenright}{\isasymrbrace}{\isachardoublequoteclose}\isanewline
%
\isadelimproof
\ \ %
\endisadelimproof
%
\isatagproof
\isacommand{apply}\isamarkupfalse%
\ vcg\ \isanewline
\ \ \isacommand{apply}\isamarkupfalse%
\ {\isacharparenleft}auto\ simp{\isacharcolon}\ not{\isacharunderscore}just{\isacharunderscore}verts\ just{\isacharunderscore}inv{\isacharunderscore}step{\isacharparenright}\isanewline
\ \ \isacommand{apply}\isamarkupfalse%
\ {\isacharparenleft}simp\ add{\isacharcolon}\ just{\isacharunderscore}inv{\isacharunderscore}def{\isacharparenright}\ \isanewline
\isacommand{done}\isamarkupfalse%
%
\endisatagproof
{\isafoldproof}%
%
\isadelimproof
\isanewline
%
\endisadelimproof
\isanewline
\isacommand{lemma}\isamarkupfalse%
\ no{\isacharunderscore}path{\isacharunderscore}inv{\isacharunderscore}step{\isacharcolon}\isanewline
\ \ {\isachardoublequoteopen}no{\isacharunderscore}path{\isacharunderscore}inv\ G\ d\ n\ {\isacharparenleft}Suc\ v{\isacharparenright}\ {\isasymlongleftrightarrow}\ no{\isacharunderscore}path{\isacharunderscore}inv\ G\ d\ n\ v\isanewline
\ \ \ \ {\isasymand}\ {\isacharparenleft}d\ v\ {\isacharequal}\ {\isasyminfinity}\ {\isasymlongleftrightarrow}\ n\ v\ {\isacharequal}\ {\isasyminfinity}{\isacharparenright}{\isachardoublequoteclose}\isanewline
%
\isadelimproof
\ \ %
\endisadelimproof
%
\isatagproof
\isacommand{by}\isamarkupfalse%
\ {\isacharparenleft}auto\ simp\ add{\isacharcolon}\ no{\isacharunderscore}path{\isacharunderscore}inv{\isacharunderscore}def\ less{\isacharunderscore}Suc{\isacharunderscore}eq{\isacharparenright}%
\endisatagproof
{\isafoldproof}%
%
\isadelimproof
\isanewline
%
\endisadelimproof
\isanewline
\isacommand{lemma}\isamarkupfalse%
\ {\isacharparenleft}\isakeyword{in}\ no{\isacharunderscore}path{\isacharunderscore}impl{\isacharparenright}\ no{\isacharunderscore}path{\isacharunderscore}spec{\isacharcolon}\isanewline
\ \ {\isachardoublequoteopen}{\isasymforall}G\ d\ n{\isachardot}\ {\isasymGamma}\ {\isasymturnstile}\isactrlsub t\ {\isasymlbrace}\ {\isasymacute}G\ {\isacharequal}\ G\ {\isasymand}\ {\isasymacute}dist\ {\isacharequal}\ d\ {\isasymand}\ {\isasymacute}enum\ {\isacharequal}\ n{\isasymrbrace}\isanewline
\ \ \ \ {\isasymacute}R\ {\isacharcolon}{\isacharequal}{\isacharequal}\ PROC\ no{\isacharunderscore}path{\isacharparenleft}{\isasymacute}G{\isacharcomma}\ {\isasymacute}dist{\isacharcomma}\ {\isasymacute}enum{\isacharparenright}\isanewline
\ \ \ \ {\isasymlbrace}\ {\isasymacute}R\ {\isacharequal}\ no{\isacharunderscore}path{\isacharunderscore}inv\ G\ d\ n\ {\isacharparenleft}ivertex{\isacharunderscore}cnt\ G{\isacharparenright}{\isasymrbrace}{\isachardoublequoteclose}\isanewline
%
\isadelimproof
\ \ %
\endisadelimproof
%
\isatagproof
\isacommand{apply}\isamarkupfalse%
\ vcg\isanewline
\ \ \isacommand{apply}\isamarkupfalse%
\ {\isacharparenleft}simp{\isacharunderscore}all\ add{\isacharcolon}\ no{\isacharunderscore}path{\isacharunderscore}inv{\isacharunderscore}step{\isacharparenright}\isanewline
\ \ \isacommand{apply}\isamarkupfalse%
\ {\isacharparenleft}auto\ simp{\isacharcolon}\ no{\isacharunderscore}path{\isacharunderscore}inv{\isacharunderscore}def{\isacharparenright}\isanewline
\isacommand{done}\isamarkupfalse%
%
\endisatagproof
{\isafoldproof}%
%
\isadelimproof
\isanewline
%
\endisadelimproof
\isanewline
\isacommand{lemma}\isamarkupfalse%
\ non{\isacharunderscore}neg{\isacharunderscore}cost{\isacharunderscore}inv{\isacharunderscore}step{\isacharcolon}\isanewline
\ \ {\isachardoublequoteopen}non{\isacharunderscore}neg{\isacharunderscore}cost{\isacharunderscore}inv\ G\ c\ {\isacharparenleft}Suc\ i{\isacharparenright}\ {\isasymlongleftrightarrow}\ non{\isacharunderscore}neg{\isacharunderscore}cost{\isacharunderscore}inv\ G\ c\ i\isanewline
\ \ \ \ {\isasymand}\ c\ i\ {\isasymge}\ {\isadigit{0}}{\isachardoublequoteclose}\isanewline
%
\isadelimproof
\ \ %
\endisadelimproof
%
\isatagproof
\isacommand{by}\isamarkupfalse%
\ {\isacharparenleft}auto\ simp\ add{\isacharcolon}\ non{\isacharunderscore}neg{\isacharunderscore}cost{\isacharunderscore}inv{\isacharunderscore}def\ less{\isacharunderscore}Suc{\isacharunderscore}eq{\isacharparenright}%
\endisatagproof
{\isafoldproof}%
%
\isadelimproof
\isanewline
%
\endisadelimproof
\isanewline
\isacommand{lemma}\isamarkupfalse%
\ {\isacharparenleft}\isakeyword{in}\ non{\isacharunderscore}neg{\isacharunderscore}cost{\isacharunderscore}impl{\isacharparenright}\ non{\isacharunderscore}neg{\isacharunderscore}cost{\isacharunderscore}spec{\isacharcolon}\isanewline
\ \ {\isachardoublequoteopen}{\isasymforall}G\ c{\isachardot}\ {\isasymGamma}\ {\isasymturnstile}\isactrlsub t\ {\isasymlbrace}\ {\isasymacute}G\ {\isacharequal}\ G\ {\isasymand}\ {\isasymacute}c\ {\isacharequal}\ c{\isasymrbrace}\isanewline
\ \ \ \ {\isasymacute}R\ {\isacharcolon}{\isacharequal}{\isacharequal}\ PROC\ non{\isacharunderscore}neg{\isacharunderscore}cost{\isacharparenleft}{\isasymacute}G{\isacharcomma}\ {\isasymacute}c{\isacharparenright}\isanewline
\ \ \ \ {\isasymlbrace}\ {\isasymacute}R\ {\isacharequal}\ non{\isacharunderscore}neg{\isacharunderscore}cost{\isacharunderscore}inv\ G\ c\ {\isacharparenleft}iedge{\isacharunderscore}cnt\ G{\isacharparenright}{\isasymrbrace}{\isachardoublequoteclose}\isanewline
%
\isadelimproof
\ \ %
\endisadelimproof
%
\isatagproof
\isacommand{apply}\isamarkupfalse%
\ vcg\isanewline
\ \ \isacommand{apply}\isamarkupfalse%
\ {\isacharparenleft}simp{\isacharunderscore}all\ add{\isacharcolon}\ non{\isacharunderscore}neg{\isacharunderscore}cost{\isacharunderscore}inv{\isacharunderscore}step{\isacharparenright}\isanewline
\ \ \isacommand{apply}\isamarkupfalse%
\ {\isacharparenleft}auto\ simp{\isacharcolon}\ non{\isacharunderscore}neg{\isacharunderscore}cost{\isacharunderscore}inv{\isacharunderscore}def{\isacharparenright}\isanewline
\isacommand{done}\isamarkupfalse%
%
\endisatagproof
{\isafoldproof}%
%
\isadelimproof
\isanewline
%
\endisadelimproof
\isanewline
\isacommand{lemma}\isamarkupfalse%
\ basic{\isacharunderscore}just{\isacharunderscore}sp{\isacharunderscore}eq{\isacharunderscore}invariants{\isacharcolon}\isanewline
{\isachardoublequoteopen}{\isasymAnd}G\ dist\ c\ s\ enum\ pred{\isachardot}\ \isanewline
\ \ basic{\isacharunderscore}just{\isacharunderscore}sp{\isacharunderscore}pred\ {\isacharparenleft}abs{\isacharunderscore}IGraph\ G{\isacharparenright}\ dist\ c\ s\ enum\ pred\ {\isasymlongleftrightarrow}\ \isanewline
\ \ \ \ {\isacharparenleft}is{\isacharunderscore}wellformed{\isacharunderscore}inv\ G\ {\isacharparenleft}iedge{\isacharunderscore}cnt\ G{\isacharparenright}\ {\isasymand}\ \isanewline
\ \ \ \ dist\ s\ {\isasymle}\ {\isadigit{0}}\ {\isasymand}\ \isanewline
\ \ \ \ trian{\isacharunderscore}inv\ G\ dist\ c\ {\isacharparenleft}iedge{\isacharunderscore}cnt\ G{\isacharparenright}\ {\isasymand}\ \isanewline
\ \ \ \ just{\isacharunderscore}inv\ G\ dist\ c\ s\ enum\ pred\ {\isacharparenleft}ivertex{\isacharunderscore}cnt\ G{\isacharparenright}{\isacharparenright}{\isachardoublequoteclose}\isanewline
%
\isadelimproof
%
\endisadelimproof
%
\isatagproof
\isacommand{proof}\isamarkupfalse%
\ {\isacharminus}\isanewline
\ \ \isacommand{fix}\isamarkupfalse%
\ G\ d\ c\ s\ n\ p\ \isanewline
\ \ \isacommand{let}\isamarkupfalse%
\ {\isacharquery}aG\ {\isacharequal}\ {\isachardoublequoteopen}abs{\isacharunderscore}IGraph\ G{\isachardoublequoteclose}\isanewline
\ \ \isacommand{have}\isamarkupfalse%
\ {\isachardoublequoteopen}fin{\isacharunderscore}digraph\ {\isacharparenleft}abs{\isacharunderscore}IGraph\ G{\isacharparenright}\ {\isasymlongleftrightarrow}\ is{\isacharunderscore}wellformed{\isacharunderscore}inv\ G\ {\isacharparenleft}iedge{\isacharunderscore}cnt\ G{\isacharparenright}{\isachardoublequoteclose}\isanewline
\ \ \ \ \isacommand{unfolding}\isamarkupfalse%
\ is{\isacharunderscore}wellformed{\isacharunderscore}inv{\isacharunderscore}def\ fin{\isacharunderscore}digraph{\isacharunderscore}def\ fin{\isacharunderscore}digraph{\isacharunderscore}axioms{\isacharunderscore}def\isanewline
\ \ \ \ \ \ wf{\isacharunderscore}digraph{\isacharunderscore}def\ no{\isacharunderscore}loops{\isacharunderscore}def\ \isanewline
\ \ \ \ \ \ \isacommand{by}\isamarkupfalse%
\ auto\isanewline
\isacommand{moreover}\isamarkupfalse%
\isanewline
\ \ \isacommand{have}\isamarkupfalse%
\ {\isachardoublequoteopen}trian{\isacharunderscore}inv\ G\ d\ c\ {\isacharparenleft}iedge{\isacharunderscore}cnt\ G{\isacharparenright}\ {\isacharequal}\ \isanewline
\ \ \ \ {\isacharparenleft}{\isasymforall}e{\isachardot}\ e\ {\isasymin}\ arcs\ {\isacharparenleft}abs{\isacharunderscore}IGraph\ G{\isacharparenright}\ {\isasymlongrightarrow}\ \isanewline
\ \ \ {\isacharparenleft}d\ {\isacharparenleft}head\ {\isacharquery}aG\ e{\isacharparenright}\ {\isasymle}\ d\ {\isacharparenleft}tail\ {\isacharquery}aG\ e{\isacharparenright}\ {\isacharplus}\ ereal\ {\isacharparenleft}c\ e{\isacharparenright}{\isacharparenright}{\isacharparenright}{\isachardoublequoteclose}\isanewline
\ \ \ \ \isacommand{by}\isamarkupfalse%
\ {\isacharparenleft}simp\ add{\isacharcolon}\ trian{\isacharunderscore}inv{\isacharunderscore}def{\isacharparenright}\isanewline
\isacommand{moreover}\isamarkupfalse%
\isanewline
\ \ \isacommand{have}\isamarkupfalse%
\ {\isachardoublequoteopen}just{\isacharunderscore}inv\ \ G\ d\ c\ s\ n\ p\ {\isacharparenleft}ivertex{\isacharunderscore}cnt\ G{\isacharparenright}\ {\isacharequal}\isanewline
\ \ \ \ {\isacharparenleft}{\isasymforall}v{\isachardot}\ v\ {\isasymin}\ verts\ {\isacharquery}aG\ {\isasymlongrightarrow}\isanewline
\ \ \ \ \ \ v\ {\isasymnoteq}\ s\ {\isasymlongrightarrow}\ n\ v\ {\isasymnoteq}\ {\isasyminfinity}\ {\isasymlongrightarrow}\ \isanewline
\ \ \ \ \ \ {\isacharparenleft}{\isasymexists}e{\isasymin}arcs\ {\isacharquery}aG{\isachardot}\ e\ {\isacharequal}\ the\ {\isacharparenleft}p\ v{\isacharparenright}\ {\isasymand}\isanewline
\ \ \ \ \ \ v\ {\isacharequal}\ head\ {\isacharquery}aG\ e\ {\isasymand}\ \isanewline
\ \ \ \ \ \ d\ v\ {\isacharequal}\ d\ {\isacharparenleft}tail\ {\isacharquery}aG\ e{\isacharparenright}\ {\isacharplus}\ ereal\ {\isacharparenleft}c\ e{\isacharparenright}\ {\isasymand}\ \isanewline
\ \ \ \ \ n\ v\ {\isacharequal}\ n\ {\isacharparenleft}tail\ {\isacharquery}aG\ e{\isacharparenright}\ {\isacharplus}\ enat\ {\isadigit{1}}{\isacharparenright}{\isacharparenright}{\isachardoublequoteclose}\isanewline
\ \ \ \ \ \ \isacommand{unfolding}\isamarkupfalse%
\ just{\isacharunderscore}inv{\isacharunderscore}def\ \isacommand{by}\isamarkupfalse%
\ fastforce\isanewline
\isacommand{ultimately}\isamarkupfalse%
\isanewline
\ \ \ \isacommand{show}\isamarkupfalse%
\ {\isachardoublequoteopen}{\isacharquery}thesis\ G\ d\ c\ s\ n\ p{\isachardoublequoteclose}\isanewline
\ \ \ \isacommand{unfolding}\isamarkupfalse%
\ \isanewline
\ \ \ \ basic{\isacharunderscore}just{\isacharunderscore}sp{\isacharunderscore}pred{\isacharunderscore}def\ \isanewline
\ \ \ \ basic{\isacharunderscore}just{\isacharunderscore}sp{\isacharunderscore}pred{\isacharunderscore}axioms{\isacharunderscore}def\ \isanewline
\ \ \ \ basic{\isacharunderscore}sp{\isacharunderscore}def\ basic{\isacharunderscore}sp{\isacharunderscore}axioms{\isacharunderscore}def\isanewline
\ \ \ \isacommand{by}\isamarkupfalse%
\ presburger\isanewline
\isacommand{qed}\isamarkupfalse%
%
\endisatagproof
{\isafoldproof}%
%
\isadelimproof
\isanewline
%
\endisadelimproof
\isanewline
\isacommand{lemma}\isamarkupfalse%
\ {\isacharparenleft}\isakeyword{in}\ check{\isacharunderscore}basic{\isacharunderscore}just{\isacharunderscore}sp{\isacharunderscore}impl{\isacharparenright}\ check{\isacharunderscore}basic{\isacharunderscore}just{\isacharunderscore}sp{\isacharunderscore}imp{\isacharunderscore}locale{\isacharcolon}\isanewline
\ \ {\isachardoublequoteopen}{\isasymforall}\ G\ d\ c\ s\ n\ p\ {\isachardot}\ {\isasymGamma}\ {\isasymturnstile}\isactrlsub t\ {\isasymlbrace}\ {\isasymacute}G\ {\isacharequal}\ G\ {\isasymand}\ {\isasymacute}dist\ {\isacharequal}\ d\ {\isasymand}\ {\isasymacute}c\ {\isacharequal}\ c\ {\isasymand}\ {\isasymacute}s\ {\isacharequal}\ s\ {\isasymand}\ {\isasymacute}enum\ {\isacharequal}\ n\ {\isasymand}\ {\isasymacute}pred\ {\isacharequal}\ p\ {\isasymrbrace}\isanewline
\ \ \ \ {\isasymacute}R\ {\isacharcolon}{\isacharequal}{\isacharequal}\ PROC\ check{\isacharunderscore}basic{\isacharunderscore}just{\isacharunderscore}sp\ {\isacharparenleft}{\isasymacute}G{\isacharcomma}\ {\isasymacute}dist{\isacharcomma}\ {\isasymacute}c{\isacharcomma}\ {\isasymacute}s{\isacharcomma}\ {\isasymacute}enum{\isacharcomma}\ {\isasymacute}pred{\isacharparenright}\isanewline
\ \ \ \ {\isasymlbrace}\ {\isasymacute}R\ {\isacharequal}\ \ basic{\isacharunderscore}just{\isacharunderscore}sp{\isacharunderscore}pred\ {\isacharparenleft}abs{\isacharunderscore}IGraph\ G{\isacharparenright}\ d\ c\ s\ n\ p{\isasymrbrace}{\isachardoublequoteclose}\isanewline
%
\isadelimproof
\ \ \ \ %
\endisadelimproof
%
\isatagproof
\isacommand{by}\isamarkupfalse%
\ vcg\ {\isacharparenleft}simp\ add{\isacharcolon}\ basic{\isacharunderscore}just{\isacharunderscore}sp{\isacharunderscore}eq{\isacharunderscore}invariants{\isacharparenright}%
\endisatagproof
{\isafoldproof}%
%
\isadelimproof
\isanewline
%
\endisadelimproof
\isanewline
\isanewline
\isacommand{lemma}\isamarkupfalse%
\ shortest{\isacharunderscore}path{\isacharunderscore}non{\isacharunderscore}neg{\isacharunderscore}cost{\isacharunderscore}eq{\isacharunderscore}invariants{\isacharcolon}\isanewline
{\isachardoublequoteopen}{\isasymAnd}G\ d\ c\ s\ n\ p\ {\isachardot}\ \isanewline
\ \ shortest{\isacharunderscore}path{\isacharunderscore}non{\isacharunderscore}neg{\isacharunderscore}cost{\isacharunderscore}pred\ {\isacharparenleft}abs{\isacharunderscore}IGraph\ G{\isacharparenright}\ d\ c\ s\ n\ p\ {\isasymlongleftrightarrow}\ \isanewline
\ \ \ \ {\isacharparenleft}is{\isacharunderscore}wellformed{\isacharunderscore}inv\ G\ {\isacharparenleft}iedge{\isacharunderscore}cnt\ G{\isacharparenright}\ {\isasymand}\ \isanewline
\ \ \ \ d\ s\ {\isasymle}\ {\isadigit{0}}\ {\isasymand}\ \isanewline
\ \ \ \ trian{\isacharunderscore}inv\ G\ d\ c\ {\isacharparenleft}iedge{\isacharunderscore}cnt\ G{\isacharparenright}\ {\isasymand}\ \isanewline
\ \ \ \ just{\isacharunderscore}inv\ G\ d\ c\ s\ n\ p\ {\isacharparenleft}ivertex{\isacharunderscore}cnt\ G{\isacharparenright}\ {\isasymand}\isanewline
\ \ \ \ s\ {\isacharless}\ ivertex{\isacharunderscore}cnt\ G\ {\isasymand}\ d\ s\ {\isacharequal}\ {\isadigit{0}}\ {\isasymand}\ \isanewline
\ \ \ \ no{\isacharunderscore}path{\isacharunderscore}inv\ G\ d\ n\ {\isacharparenleft}ivertex{\isacharunderscore}cnt\ G{\isacharparenright}\ {\isasymand}\isanewline
\ \ \ \ non{\isacharunderscore}neg{\isacharunderscore}cost{\isacharunderscore}inv\ G\ c\ {\isacharparenleft}iedge{\isacharunderscore}cnt\ G{\isacharparenright}{\isacharparenright}{\isachardoublequoteclose}\isanewline
%
\isadelimproof
%
\endisadelimproof
%
\isatagproof
\isacommand{proof}\isamarkupfalse%
\ {\isacharminus}\isanewline
\ \ \isacommand{fix}\isamarkupfalse%
\ G\ d\ c\ s\ n\ p\ \isanewline
\ \ \isacommand{let}\isamarkupfalse%
\ {\isacharquery}aG\ {\isacharequal}\ {\isachardoublequoteopen}abs{\isacharunderscore}IGraph\ G{\isachardoublequoteclose}\isanewline
\ \ \isacommand{have}\isamarkupfalse%
\ {\isachardoublequoteopen}no{\isacharunderscore}path{\isacharunderscore}inv\ G\ d\ n\ {\isacharparenleft}ivertex{\isacharunderscore}cnt\ G{\isacharparenright}\ {\isasymlongleftrightarrow}\ \isanewline
\ \ \ \ {\isacharparenleft}{\isasymforall}v{\isachardot}\ v\ {\isasymin}\ verts\ {\isacharquery}aG\ {\isasymlongrightarrow}\ {\isacharparenleft}d\ v\ {\isacharequal}\ {\isasyminfinity}{\isacharparenright}\ {\isacharequal}\ {\isacharparenleft}n\ v\ {\isacharequal}\ {\isasyminfinity}{\isacharparenright}{\isacharparenright}{\isachardoublequoteclose}\isanewline
\ \ \ \ \isacommand{by}\isamarkupfalse%
\ {\isacharparenleft}simp\ add{\isacharcolon}\ no{\isacharunderscore}path{\isacharunderscore}inv{\isacharunderscore}def{\isacharparenright}\isanewline
\isacommand{moreover}\isamarkupfalse%
\isanewline
\ \ \isacommand{have}\isamarkupfalse%
\ {\isachardoublequoteopen}non{\isacharunderscore}neg{\isacharunderscore}cost{\isacharunderscore}inv\ G\ c\ {\isacharparenleft}iedge{\isacharunderscore}cnt\ G{\isacharparenright}\ {\isasymlongleftrightarrow}\ \isanewline
\ \ \ \ {\isacharparenleft}{\isasymforall}e{\isachardot}\ e\ {\isasymin}\ arcs\ {\isacharquery}aG\ {\isasymlongrightarrow}\ {\isadigit{0}}\ {\isasymle}\ c\ e{\isacharparenright}{\isachardoublequoteclose}\isanewline
\ \ \ \ \isacommand{by}\isamarkupfalse%
\ {\isacharparenleft}simp\ add{\isacharcolon}\ non{\isacharunderscore}neg{\isacharunderscore}cost{\isacharunderscore}inv{\isacharunderscore}def{\isacharparenright}\isanewline
\isacommand{ultimately}\isamarkupfalse%
\isanewline
\ \ \ \isacommand{show}\isamarkupfalse%
\ {\isachardoublequoteopen}{\isacharquery}thesis\ G\ d\ c\ s\ n\ p{\isachardoublequoteclose}\isanewline
\ \ \ \isacommand{unfolding}\isamarkupfalse%
\ shortest{\isacharunderscore}path{\isacharunderscore}non{\isacharunderscore}neg{\isacharunderscore}cost{\isacharunderscore}pred{\isacharunderscore}def\ \isanewline
\ \ \ \ shortest{\isacharunderscore}path{\isacharunderscore}non{\isacharunderscore}neg{\isacharunderscore}cost{\isacharunderscore}pred{\isacharunderscore}axioms{\isacharunderscore}def\isanewline
\ \ \ \isacommand{using}\isamarkupfalse%
\ basic{\isacharunderscore}just{\isacharunderscore}sp{\isacharunderscore}eq{\isacharunderscore}invariants\ \isacommand{by}\isamarkupfalse%
\ simp\isanewline
\isacommand{qed}\isamarkupfalse%
%
\endisatagproof
{\isafoldproof}%
%
\isadelimproof
\isanewline
%
\endisadelimproof
\isanewline
\isacommand{theorem}\isamarkupfalse%
\ {\isacharparenleft}\isakeyword{in}\ check{\isacharunderscore}sp{\isacharunderscore}impl{\isacharparenright}\ check{\isacharunderscore}sp{\isacharunderscore}eq{\isacharunderscore}locale{\isacharcolon}\isanewline
\ \ {\isachardoublequoteopen}{\isasymforall}\ G\ d\ c\ s\ n\ p\ {\isachardot}\ {\isasymGamma}\ {\isasymturnstile}\isactrlsub t\ {\isasymlbrace}\ {\isasymacute}G\ {\isacharequal}\ G\ {\isasymand}\ {\isasymacute}dist\ {\isacharequal}\ d\ {\isasymand}\ {\isasymacute}c\ {\isacharequal}\ c\ {\isasymand}\ {\isasymacute}s\ {\isacharequal}\ s\ {\isasymand}\ {\isasymacute}enum\ {\isacharequal}\ n\ {\isasymand}\ {\isasymacute}pred\ {\isacharequal}\ p\ {\isasymrbrace}\isanewline
\ \ \ \ {\isasymacute}R\ {\isacharcolon}{\isacharequal}{\isacharequal}\ PROC\ check{\isacharunderscore}sp{\isacharparenleft}{\isasymacute}G{\isacharcomma}\ {\isasymacute}dist{\isacharcomma}\ {\isasymacute}c{\isacharcomma}\ {\isasymacute}s{\isacharcomma}\ {\isasymacute}enum{\isacharcomma}\ {\isasymacute}pred{\isacharparenright}\isanewline
\ \ \ \ {\isasymlbrace}\ {\isasymacute}R\ {\isacharequal}\ shortest{\isacharunderscore}path{\isacharunderscore}non{\isacharunderscore}neg{\isacharunderscore}cost{\isacharunderscore}pred\ {\isacharparenleft}abs{\isacharunderscore}IGraph\ G{\isacharparenright}\ d\ c\ s\ n\ p{\isasymrbrace}{\isachardoublequoteclose}\isanewline
%
\isadelimproof
\ \ \ \ %
\endisadelimproof
%
\isatagproof
\isacommand{by}\isamarkupfalse%
\ vcg\ {\isacharparenleft}auto\ simp\ add{\isacharcolon}\ shortest{\isacharunderscore}path{\isacharunderscore}non{\isacharunderscore}neg{\isacharunderscore}cost{\isacharunderscore}eq{\isacharunderscore}invariants{\isacharparenright}%
\endisatagproof
{\isafoldproof}%
%
\isadelimproof
\isanewline
%
\endisadelimproof
\isanewline
\isacommand{lemma}\isamarkupfalse%
\ shortest{\isacharunderscore}path{\isacharunderscore}non{\isacharunderscore}neg{\isacharunderscore}cost{\isacharunderscore}imp{\isacharunderscore}correct{\isacharcolon}\isanewline
{\isachardoublequoteopen}{\isasymAnd}G\ d\ c\ s\ n\ p\ {\isachardot}\ \isanewline
\ \ shortest{\isacharunderscore}path{\isacharunderscore}non{\isacharunderscore}neg{\isacharunderscore}cost{\isacharunderscore}pred\ {\isacharparenleft}abs{\isacharunderscore}IGraph\ G{\isacharparenright}\ d\ c\ s\ n\ p\ {\isasymlongrightarrow}\isanewline
\ \ \ {\isacharparenleft}{\isasymforall}v\ {\isasymin}\ verts\ {\isacharparenleft}abs{\isacharunderscore}IGraph\ G{\isacharparenright}{\isachardot}\ \isanewline
\ \ \ d\ v\ {\isacharequal}\ wf{\isacharunderscore}digraph{\isachardot}{\isasymmu}\ {\isacharparenleft}abs{\isacharunderscore}IGraph\ G{\isacharparenright}\ c\ s\ v{\isacharparenright}{\isachardoublequoteclose}\isanewline
%
\isadelimproof
%
\endisadelimproof
%
\isatagproof
\isacommand{using}\isamarkupfalse%
\ shortest{\isacharunderscore}path{\isacharunderscore}non{\isacharunderscore}neg{\isacharunderscore}cost{\isacharunderscore}pred{\isachardot}correct{\isacharunderscore}shortest{\isacharunderscore}path{\isacharunderscore}pred\ \isacommand{by}\isamarkupfalse%
\ fast%
\endisatagproof
{\isafoldproof}%
%
\isadelimproof
\isanewline
%
\endisadelimproof
\isanewline
\isacommand{theorem}\isamarkupfalse%
\ {\isacharparenleft}\isakeyword{in}\ check{\isacharunderscore}sp{\isacharunderscore}impl{\isacharparenright}\ check{\isacharunderscore}sp{\isacharunderscore}spec{\isacharcolon}\isanewline
\ \ {\isachardoublequoteopen}{\isasymforall}\ G\ d\ c\ s\ n\ p\ {\isachardot}\ {\isasymGamma}\ {\isasymturnstile}\isactrlsub t\ {\isasymlbrace}\ {\isasymacute}G\ {\isacharequal}\ G\ {\isasymand}\ {\isasymacute}dist\ {\isacharequal}\ d\ {\isasymand}\ {\isasymacute}c\ {\isacharequal}\ c\ {\isasymand}\ {\isasymacute}s\ {\isacharequal}\ s\ {\isasymand}\ {\isasymacute}enum\ {\isacharequal}\ n\ {\isasymand}\ {\isasymacute}pred\ {\isacharequal}\ p\ {\isasymrbrace}\isanewline
\ \ \ \ {\isasymacute}R\ {\isacharcolon}{\isacharequal}{\isacharequal}\ PROC\ check{\isacharunderscore}sp{\isacharparenleft}{\isasymacute}G{\isacharcomma}\ {\isasymacute}dist{\isacharcomma}\ {\isasymacute}c{\isacharcomma}\ {\isasymacute}s{\isacharcomma}\ {\isasymacute}enum{\isacharcomma}\ {\isasymacute}pred{\isacharparenright}\isanewline
\ \ \ \ {\isasymlbrace}\ {\isasymacute}R\ {\isasymlongrightarrow}\ \ {\isacharparenleft}{\isasymforall}v\ {\isasymin}\ verts\ {\isacharparenleft}abs{\isacharunderscore}IGraph\ G{\isacharparenright}{\isachardot}\ d\ v\ {\isacharequal}\ wf{\isacharunderscore}digraph{\isachardot}{\isasymmu}\ {\isacharparenleft}abs{\isacharunderscore}IGraph\ G{\isacharparenright}\ c\ s\ v{\isacharparenright}{\isasymrbrace}{\isachardoublequoteclose}\isanewline
%
\isadelimproof
%
\endisadelimproof
%
\isatagproof
\isacommand{using}\isamarkupfalse%
\ shortest{\isacharunderscore}path{\isacharunderscore}non{\isacharunderscore}neg{\isacharunderscore}cost{\isacharunderscore}eq{\isacharunderscore}invariants\isanewline
\ \ \ \ \ \ shortest{\isacharunderscore}path{\isacharunderscore}non{\isacharunderscore}neg{\isacharunderscore}cost{\isacharunderscore}imp{\isacharunderscore}correct\ \isanewline
\isacommand{by}\isamarkupfalse%
\ vcg\ blast%
\endisatagproof
{\isafoldproof}%
%
\isadelimproof
\isanewline
%
\endisadelimproof
%
\isadelimtheory
\isanewline
%
\endisadelimtheory
%
\isatagtheory
\isacommand{end}\isamarkupfalse%
%
\endisatagtheory
{\isafoldtheory}%
%
\isadelimtheory
%
\endisadelimtheory
\end{isabellebody}%
%%% Local Variables:
%%% mode: latex
%%% TeX-master: "root"
%%% End:


%
\begin{isabellebody}%
\def\isabellecontext{Graph{\isacharunderscore}Checker{\isacharunderscore}Verification{\isacharunderscore}Simpl}%
%
\isadelimtheory
\isanewline
\isanewline
%
\endisadelimtheory
%
\isatagtheory
\isacommand{theory}\isamarkupfalse%
\ Graph{\isacharunderscore}Checker{\isacharunderscore}Verification{\isacharunderscore}Simpl\isanewline
\isakeyword{imports}\isanewline
\ \ {\isachardoublequoteopen}Check{\isacharunderscore}Connected{\isacharunderscore}Impl{\isachardoublequoteclose}\isanewline
\ \ {\isachardoublequoteopen}Check{\isacharunderscore}Connected{\isacharunderscore}Verification{\isachardoublequoteclose}\isanewline
\ \ {\isachardoublequoteopen}Check{\isacharunderscore}Shortest{\isacharunderscore}Path{\isacharunderscore}Impl{\isachardoublequoteclose}\isanewline
\ \ {\isachardoublequoteopen}Check{\isacharunderscore}Shortest{\isacharunderscore}Path{\isacharunderscore}Verification{\isachardoublequoteclose}\isanewline
\isanewline
\isakeyword{begin}\isanewline
\isanewline
\isacommand{end}\isamarkupfalse%
%
\endisatagtheory
{\isafoldtheory}%
%
\isadelimtheory
%
\endisadelimtheory
\end{isabellebody}%
%%% Local Variables:
%%% mode: latex
%%% TeX-master: "root"
%%% End:


%%% Local Variables:
%%% mode: latex
%%% TeX-master: "root"
%%% End:


% optional bibliography
\bibliographystyle{abbrv}
\bibliography{root}

\end{document}

%%% Local Variables:
%%% mode: latex
%%% TeX-master: t
%%% End:
